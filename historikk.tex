\chapter*{Kort historikk}
%\addcontentsline{toc}{chapter}{Kort Historikk}

\year{1910}

\begin{itemize}
	\item ``Norges Tekniske Høgskoles Studentersamfund'' blir stiftet 1. oktober. Det velges en bestyrelse som senere blir
	til Styret.
	\item Trondhjems Studentersangforening (TSS) dannes.
  \item Studentersamfundets orkester.
  \item Studentersamfundets Interne Teater (SIT).
\end{itemize}

\year{1912}

\begin{itemize}
	\item Navneskifte -- med virkning for alltid -- til ``Studentersamfundet i Trondhjem''.
  \item Samfundet kjøper ``Cirkus'' for kr. 50.000,- og får dermed sitt eget hus. Fast ansatt forretningsfører.
\end{itemize}

\year{1913}

\begin{itemize}
  \item SIT oppfører ``Maksis''
  \item Finansstyret
\end{itemize}

\year{1914}

\begin{itemize}
  \item Under Dusken kommer ut for første gang.
  \item Studentersamfundets Frivillige Undervisning. Skal drive utadrettet undervisning. Er sammen med tilsvarende
organer i Oslo og Bergen starten for det som senere skulle bli Folkeuniversitetet og Friundervisningen.
\end{itemize}

\year{1915}

\begin{itemize}
  \item Den første økonomiske krisen
\end{itemize}

\year{1916}

\begin{itemize}
  \item Det Sorte Faars Riddersskab (DSFRS)
\end{itemize}

\year{1917}

\begin{itemize}
  \item Ny økonomisk krise som fører til at revyen "Baccarat" blir oppført. Dermed er begrepet ``Studenter-UKE'' skapt.
\end{itemize}

\year{1918}

\begin{itemize}
  \item Cirkustomten selges til kommunen. Det legges frem planer for ny Samfundsbygning.
\end{itemize}

\year{1922}

\begin{itemize}
  \item Samfundet får 100.000 fra Stortinget til nytt hus.
\end{itemize}

\year{1923}

\begin{itemize}
  \item Akademisk Radioklubb (ARK). De driver radiosendinger nordenfjelds før NRK.
\end{itemize}

\year{1925}

\begin{itemize}
  \item Trondheim overdrar Vollantomten til Samfundet gratis.
\end{itemize}

\year{1927}

\begin{itemize}
  \item HKH Kronprins Olav nedlegger grunnsteinen på Vollan.
\end{itemize}

\year{1929}

\begin{itemize}
  \item 1. oktober kl. 19.00 starter det siste Samfundsmøtet i Cirkus. Kl. 21.00 etter prosesjon over Elgeseter bru starter
det første møtet i ``Cassa Rossa''. Kronprinsparet deltar.
  \item Bodega Band.
  \item Regi.
\end{itemize}

\year{1930}

\begin{itemize}
  \item Forsterkerkomiteen (FK)
  \item Trondheims Kvinnelige Studentersangforening (TKS)
\end{itemize}

\year{1933}

\begin{itemize}
  \item Klubbstyret (KLST)
\end{itemize}

\year{1935}

\begin{itemize}
  \item 25- årsjubileum. Samfundets første historiebok kommer ut. (``Studentersamfundet i Trondhjem gjennem 25 år'' av
Brochmann.)
\end{itemize}

\year{1941 - 1945}

\begin{itemize}
  \item Bygningen taes i bruk av tyskerne. Studentersamfundet som institusjon lever videre i Sverige og England.
\end{itemize}

\year{1945}

\begin{itemize}
  \item 9. mai tas Samfundsbygningen igjen i bruk av studentene. Rekordartet aktivitet.
  \item 9. juni taler HKH Kronprins Olav i Storsalen i offiseruniform med kun DSFRS's orden på brystet.
\end{itemize}

\year{1953}

\begin{itemize}
  \item Det blir fremmet forslag om utvidelse av Samfundsbygningen
\end{itemize}

\year{1956}

\begin{itemize}
  \item HV Hertugen av Belfaar DSFRS deltar for første gang på et ordinert Samfundsmøte med sin datter HKH
prinsesse Astrid.
\end{itemize}

\year{1958}

\begin{itemize}
  \item Fotogjengen (FG).
\end{itemize}

\year{1960}

\begin{itemize}
  \item 50- årsjubileum med diverse jubileumsmøter og ny historiebok.
  \item Jubileumsinnsamling.
\end{itemize}

\year{1963}

\begin{itemize}
  \item Strindens Prommenade Orchester (SPO).
  \item Deler av bakgården gjenbygges.
\end{itemize}

\year{1965}

\begin{itemize}
  \item Et svært radikalt vårprogram fra Styret varsler en ny tid.
  \item Pirum.
\end{itemize}

\year{1966}

\begin{itemize}
  \item Kulturutvalget (KU). 
\end{itemize}

\year{1967}

\begin{itemize}
  \item "Happeningmøtet" med dansken Jørgen Nash. Studentersamfundet får stor omtale i alle landets aviser.
  \item Snaustrinda Spellemanslag.
\end{itemize}

\year{1970}

\begin{itemize}
  \item Samfundet i endring, bl.a. blir FS kastet.
  \item Politiske fronter dannes for å kjempe om styretaburetten i Samfundet. 
\end{itemize}

\year{1973}

\begin{itemize}
  \item Tillatt for kvinnelige medlemmer å ha med følge som ikke er medlem.
\end{itemize}

\year{1974}

\begin{itemize}
  \item 2565 avgitte stemmer i ledervalg.
  \item S. Møller Storband.
\end{itemize}

\year{1976}

\begin{itemize}
  \item Diversegjengen (DG).
\end{itemize}

\year{1977}

\begin{itemize}
  \item Pirum Old Boys (POB).
  \item Knauskoret
  \item Kjellerbandet.
\end{itemize}

\year{1982}

\begin{itemize}
  \item Candiss
\end{itemize}

\year{1984}

\begin{itemize}
  \item Studentersamfundets Radio.
\end{itemize}

\year{1985}

\begin{itemize}
  \item Det blir utført større om- og utbygging, som blant annet innebærer total gjenbygging av bakgården,
Byggeprosjektet blir langt dyrere enn antatt og FS må trekke seg.
  \item Ny historiebok.
\end{itemize}

\year{1988}

\begin{itemize}
  \item Samfundet er nær ved å gå konkurs. 50 personer blir sagt opp og restaurant- og serveringsvirksomheten blir satt
bort.
  \item Mediastud AS opprettes.
  \item Informasjonsgjengen (IG). 
\end{itemize}

\year{1990}

\begin{itemize}
  \item ISFIT (Internasjonal Studentfestival) blir arrangert i Samfundet etter initiativ fra tidligere styremedlemmer og
med Samfundet som en av stifterene. Studenter fra mer enn 50 land deltar.
\end{itemize}

\year{1991}

\begin{itemize}
  \item Serveringsgjengen
  \item Student-TV (STV)
\end{itemize}

\year{1992}

\begin{itemize}
  \item Trondheim Interrailcenter (TIRC) 
\end{itemize}

\year{1995}

\begin{itemize}
  \item Avtalen for Mediastud blir reforhandlet, og etter ønske fra både Studentersamfundet, Samskipnaden og
Studenttinget blir Student-TV en del av selskapet.
\end{itemize}

\year{1997}

\begin{itemize}
  \item Daglighallen Pub bygges opp i løpet av sommeren.
  \item Bodegaen bygges om og får landets tyngste bar.
\end{itemize}

\year{1998}

\begin{itemize}
  \item  TARM opprettes som en egen samfunnskontaktgjeng. (Taktisk Arbeidsgruppe for Reklame og Markedsføring). TARM blir senere omdøpt til Respons.
\end{itemize}

\year{1999}

\begin{itemize}
  \item IT-komiteen (ITK)
  \item Gjengsekretariatet (GS)
  \item Lørdagskomiteen dannes som undergruppe av Styret.
  \item Rundhallen bygges om og mister sin karakteristiske runde bar.
\end{itemize}

\year{2000}

\begin{itemize}
  \item Samfundet overtar driften av Strossa, som frem til da har vært drevet av restauratøren. 
\end{itemize}

\year{2001}

\begin{itemize}
  \item Klubben og Selskapssiden bygges om. Klubben mister sin tradisjonelle oppdeling i Birkeland og Kongsberg.
\end{itemize}

\year{2002}

\begin{itemize}
  \item Lørdagskomiteen (LK)
\end{itemize}

\year{2003}

\begin{itemize}
  \item Informasjonsgjengen og Respons slås sammen til Layout Info Market (LIM).
  \item Strossa bygges om, nytt inngangsparti til kjelleren bygges på sørsiden av bygget.
\end{itemize}

\year{2004}

\begin{itemize}
  \item Planer for nybygg på Fengselstomta godkjennes av Studentersamfundet i Trondhjem.
\end{itemize}

\year{2005}

\begin{itemize}
  \item Rekordstort medlemstall for Samfundet over 9000 medlemmer.
  \item Intensjonsavtale med NTNU om bygg på Fengselstomta signeres.
  \item Videokomiteen (VK), som tidligere undergruppe av ARK, blir egen gjeng.
  \item Bodegaen bygges om. Nye toaletter kommer til i kjelleren.
  \item Kafégjengen (KG)
\end{itemize}

\year{2006}

\begin{itemize}
  \item Samfundet åpner spisestedet Lyche.
  \item Kafégjengen tar navnet Kafégjengen (KiSS)
\end{itemize}

\year{2007}

\begin{itemize}
  \item Rundhallen er byggeposjektet til UKA
  \item Profilgruppa (PG) opprettes som ny undergruppe av Finansstyret.
\end{itemize}

Viser til følgende historiebøker om Samfundet:

\textit{Studentersamfundet gjennem 25 år}, v/Georg Brochmann\\
\textit{Studenter i den gamle stad}, v/Harald Ramm Rønneberg\\
\textit{Vår egen lille verden}, v/Idar Lind og Gunnar Strøm\\
