\chapter*{Vedvarende vedtak i Studentersamfundet}
%\addcontentsline{toc}{chapter}{Vedvarende vedtak i Studentersamfundet}

\vedtak{08.09.01}{Styret ved Studentersamfundet}{%
    Studentersamfundet i Trondhjem, samla til møte i Storsalen lørdag 8. september, krever at kampen mot narkotika må
    intensiveres gjennom økte bevilgninger til forskning på rusmidler og forebyggende arbeid. Politikere satser ofte på
    tiltak som synes, heller enn de som virker, og vi krever en satsning på tiltak med
    dokumentert virkning.
    }


\vedtak{18.08.01}{Styret ved Studentersamfundet}{%
    Studentersamfundet i Trondhjem, samlet til møte i Storsalen lørdag 18. august, krever at muligheten for frivillig
    studentengasjement ikke blir svekket gjennom den nye utdanningsreformen.
    }


\vedtak{22.09.01}{Lars Indresæter}{%
Studentersamfundet i Trondhjem, samla til møte i Storsalen lørdag 22. september, fordømmer alle terrorhandlinger,
slik som anslaget som rammet USA tirsdag 11. september. Samfundet ønsker at de ansvarlige skal stilles foran en
internasjonal rett, og er i mot en militær gjengjeldesaksjon som garantert kommer til å skader sivile.
}

\vedtak{15.11.03}{Håvard Hamnaberg}{%
Intern meningsytring fremmet av Håvard Hamnaberg:

Studentersamfundet, samlet til møte i Storsalen 15.11.03, mener at det i Studentersamfundets og tilknyttede
organisasjoners publikasjoner ikke bør forekomme diskriminering eller sensur av ytringer på bakgrunn av valg av
målform i norsk.
}


\vedtak{02.02.05}{Styret ved Studentersamfundet}{%
Studentersamfundet i Trondhjem samlet til Samfundsmøte den 2. februar 2005 stiller seg positive til videre utredelse
av planene om en samlokalisering av NTNU rundt Gløshaugenområdet, og oppfordrer Styret ved NTNU til å gjøre det
samme.
}

\vedtak{05.03.05}{Styret ved Studentersamfundet}{%
Studentersamfundet i Trondhjem samlet til Samfundsmøte den 5. mars 2005 vedtar å boikotte alle israelske varer
inntil Israel innfrir kravene skissert i ”Veikartet for fred”. Studentersamfundet oppfordrer Studentsamskipnaden i
Trondheim til å gjøre det samme.
}

\vedtak{23.04.05}{Arne Espelund}{%
Mordechai Vanunu er løslatt etter å ha sittet 18 år i fengsel i Israel, straffet for å ha røpet noe om Israels
atomvåpenplaner. Han blir fortsatt holdt i en form for husarrest. I Norge er det samlet inn 100 000 kroner for å støtte
ham, og han har søkt om å få komme til Norge. Studentersamfundet i Trondhjem ber statsråd Erna Solberg om å
innvilge denne søknaden og ikke la formelle grunner hindre dette.
}

\vedtak{25.03.06}{Styret ved Studentersamfundet}{%
Studentersamfundet i Trondhjem samlet til Samfundsmøte 25. mars 2006 ønsker å samle NTNU på en campus og
oppfordrer Styret ved NTNU til å gå inn for en flytting av fagmiljøene på Dragvoll til Gløshaugen-området.
}

\vedtak{25.0306}{Runar Jensen}{%
Studentersamfundet i Trondhjem skal tilby sine medlemmer Max Havelaar-merket kaffe og te, og benytte seg av
rettferdig handel også ellers der dette lar seg gjøre. Samfundet skal markedsføre Max Havelaar-produktene Samfundet
tilbyr sine medlemmer, og stå frem politisk for rettferdig handel.
}

\vedtak{23.09.06}{Studentradion}{%
Studentersamfundet i Trondhjem samlet i Storsalen den 23/9-06 2006 mener Studentradion i Trondheim spiller en
spesielt viktig rolle for å spre studentinformasjon og sørge for at studentenes stemme høres i samfunnslivet.
Studentradion i Trondheim er, gjennom sitt samarbeid med andre studentradiokanaler i Norge, en motvekt mot
kommersielle aktører og vil som 24-timers studentradio kunne tilby Trondheims studenter et bredere redaksjonelt
stoff enn i dag. Studentersamfundet i Trondhjem ønsker derfor å støtte Studentradion i Trondheim i sitt arbeid for å bli
en fulltids lokal kringkaster.
}

\vedtak{30.09.06}{Finansstyret}{%
Storsalen går inn for at Studentersamfundet fullfører samarbeidet med NTNU om oppføring av et nytt bygg på
Fengselstomta ”i henhold til Intensjonsavtale” av 20. april 2005.
Finansstyret ber Storsalen gi Finansstyrets leder og Daglig leder fullmakt til å inngå avtaler og skrive kontrakter med
NTNU i henhold til foran nevnte avtale. Det vil si leieavtale for arealbruk og driftsutgifter, kjøpekontrakt for
overskjøting av eiendommen.
}

\vedtak{14.10.06}{Nils Jørgen Selbekk}{%
Studentersamfundet i Trondhjem samlet til Samfundsmøte 14. oktober 2006 mener at et dyrt og dårlig kollektivtilbud
er både dårlig student- og klimapolitikk, og oppfordrer derfor Trondheim Bystyre til å holde fast ved dagens takster i
kollektivtrafikken, samt å i større grad legge til rette for fremkommeligheten for bussen.
}

\vedtak{04.11.06}{Styret ved Samfundet}{%
Studentersamfundet I Trondhjem samlet til Samfundsmøte 04. november 2006 mener regjeringen sitt forslag til
statsbudsjett legger hindringer i veien for langsiktig verdiskapning gjennom høyere utdanning og forskning.
Realkuttene i overføringene til Universitets- og Høyskolesektoren på 274 millioner viser at regjeringen ikke innser
verdien av forskning og høyere utdanning. Dette vises ved at det ikke er satt av noen midler til nye stipendiatstillinger,
overføringen til Fondet for forskning og nyskapning er for lav, og SkatteFUNN-ordningen har fått ett kutt på ca 150
millioner eller 10%.

Også på Velferdssiden er det dramatiske kutt. Avviklingen av støtten til studentbarnehagene og kuttet i midlene til
studentboligbygging er alvorlige angrep på lik rett til utdanning.

Studentersamfundet i Trondhjem ber Stortinget vurdere følgene av dette når statsbudsjett skal opp til endelig
behandling.
}

\vedtak{08.09.07}{Styret ved Samfundet}{%
Studentersamfundet i Trondhjem samlet til Samfundsmøte lørdag 8.september 2007 mener det bør innføres
rushtidsavgift i Trondheim for å redusere unødvendig bilbruk, og oppfordrer til at midlene tilføres kollektivtilbudet og
bidrar til en styrking av dette.
}



