
\begin{instruks}{Instruks for Gjengsekretariatet}{}{}

    \begin{instruksledd}{Formål}
        \begin{enumerate}
            \item Gjengsekretariatet (GS) skal drive saksforberedelse og -oppfølging for Finansstyret (FS) i forhold som vedrører
foreningsvirksomheten og gjennomføre vedtak fattet av FS i samme saksområde. GS
skal særlig virke i administrative og felles anliggender for gjengene, og i saker som ikke fanges opp av
eksisterende gjengers ansvarsområder. GS er underlagt FS.
            \item  GS skal være koordinerende, samarbeidsskapende og arbeide for å bedre kommunikasjonen
mellom gjengene, forretningsdriften og FS. GS skal fungere som ressurs- og
kompetansesenter for gjengene, og bidra til erfaringsoverføring og kontinuitet i foreningsvirksomheten.
GS vil også arbeide med organisatoriske oppdrag gitt av FS og etter eget initiativ.
        \end{enumerate}
    \end{instruksledd}

    \begin{instruksledd}{Sammensettning}
        \begin{enumerate}
            \item GS består av 4-10 aktive medlemmer, etter aktivitetsnivå og behov for å ivareta kompetanse.
Antallet medlemmer fastsettes av FS etter innstilling fra GS. GS
utlyser opptak av nye medlemmer ved behov. Medlemmene oppnevnes av FS etter innstilling fra
GS og forutgående behandling av innstillingen i Gjengsjefkollegiet. Bare medlemmer av
Studentersamfundet kan være medlemmer av GS.
            \item Medlemmer av GS skal normalt være aktive i minimum 1 år. Ved innstilling av nye
medlemmer skal GS sørge for kompetanseoverføring og at det sikres en kontinuerlig drift.
Medlemmene av GS kan innvilge et medlem permisjon, og kan også frita et medlem for sitt
medlemskap.
            \item  GS' medlemmer skal normalt rekrutteres blant nåværende eller tidligere tillitsvalgte ved
Studentersamfundet. GS' sammensetning bør gjenspeile bredden i foreningsvirksomheten og
medlemmene skal likestille alle aktiviteter som drives i Studentersamfundet.
            \item GS' kan sette ned undergrupper av begrenset varighet for særskilte formål. Gjensekretariatet
oppnevner medlemmer av slike undergrupper etter behov.
            \item GS opprettholder virksomheten under UKA
        \end{enumerate}
    \end{instruksledd}

    \begin{instruksledd}{Ansvarsområder og plikter}
        \begin{enumerate}
            \item Medlemmer i GS plikter å kjenne innholdet av og overholde
	    føgende instrukser på Huset:
                \begin{enumerate}
                    \item Generell gjenginstruks
                    \item Husorden
                    \item Branninstruks
                    \item Studentersamfundets lover
                \end{enumerate}
            \item Medlemmer av GS disponerer kontorplass etter avtale med daglig leder.
            \item Lederen av GS skal møte i FS' møter. Et annet medlem kan være sekretær i
Gjengsjefkollegiet.
            \item GS skal samarbeide med daglig leder og leder for Gjengsjefkollegiet.
        \end{enumerate}
    \end{instruksledd}

    \begin{instruksledd}{Formidling}
        \begin{enumerate}
            \item GS' leder plikter å gjøre nye medlemmer i GS kjent med denne instruks.
        \end{enumerate}
    \end{instruksledd}


\end{instruks}
