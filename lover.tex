
\chapter*{Samfundets Lover}

\begin{lovkapittel}{}

\begin{lovparagraf}{Overordnet målsetning for Studentersamfundet i Trondhjem}

    \begin{enumerate}
        \item Studentersamfundet skal være det naturlige samlingsted for studentene i Trondhjem, og gi
            et best mulig sosialt og kulturelt tilbud, samt skape forutsetninger for debatt og
            samfunnsengasjement. Studentersamfundet skal være en kulturinstitusjon i Trondheim.
        \item Studentersamfundet skal først og fremst være til for sine medlemmer. Dernest
            skal Studentersamfundet gi et tilbud til andre studenter, det akademiske miljø
            ved medlemsskolene og tidligere medlemmer. I tillegg skal Studentersamfundet
            skape et tilbud for folk i byen.
	\item Studentersamfundet skal eies og drives av sine medlemmer. Studentersamfundet skal gi så mange som mulig 
	av medlemmene anledning til å delta i arbeidet og aktivitetene til Studentersamfundet.
	\item Studentersamfundets skal drives på et forretningsmessig grunnlag. Det økonomiske målet er å ha en sunn og 
	selvstendig økonomi. Et positivt driftsresultat og UKAs overskudd skal sikre Studentersamfundets
	uavhengige posisjon.

    \end{enumerate}
\end{lovparagraf}





\begin{lovparagraf}{Medlemmer}

  \begin{enumerate}
    \item Rett til å være medlemmer har de som er over 18 år, og som:
      
      \begin{enumerate}
      \item er eller har vært studenter eller lærere ved en opplæringsinstitusjon på universitets- eller høgskolenivå
      som ligger i Trondheimsområdet, og der skolen baserer seg på heltidsstudenter.
      \item Har vært studenter ved en opplæringsinstitusjon på universitets- eller høgskolenivå og som nå er bosatt i
      Trondheims-området.
      \end{enumerate}
      
    \item Med samtykke fra Styret kan andre personer med høyere utdanning bli medlemmer av Studentersamfundet, selv om de ikke oppfyller kravene under pkt. 1.
    \item Studentersamfundet fatter vedtak om hvilke opplæringsinstitusjoner på universitets- eller høgskolenivå i 
    Trondheimsområdet som gir rett til medlemskap. Styret plikter å ajourføre en oversikt over disse.
    \item Ingen medlemmer eller deltakere i nazistiske eller andre organisasjoner som står på et klart rasistisk 
    grunnlag, kan være medlemmer i Studentersamfundet i Trondhjem. Dette gjelder også for åpent erkjente 
    rasister, nazister og sympatisører av disse organisasjonene.
    \item Samfundsretten avgjør saker som kommer inn under §2. 4.

  \end{enumerate}

\end{lovparagraf}


\begin{lovparagraf}{Innbudte medlemmer}
  
Ved styrevedtak kan enhver som har gjort seg fortjent til det, innbys som medlem av Studentersamfundet for to semester eller for livstid. Vedtaket skal legges fram for Studentersamfundet til godkjenning.
  
\end{lovparagraf}


\begin{lovparagraf}{Æresmedlemmer}

Personer som ved sin innsats i politikk, kultur eller i samfunnslivet forøvrig har gjort seg særlig fortjent til dette, kan
innbys som æresmedlemmer av Studentersamfundet etter innstilling fra Styret. Vedtaket legges fram for
Studentersamfundet til godkjenning.

\end{lovparagraf}

\end{lovkapittel}



\begin{lovkapittel}{Styrende Organer}

  \begin{lovparagraf}{Studentersamfundets høyeste myndighet}
    
Studentersamfundet samlet til Samfundsmøte er den høyeste myndighet i Studentersamfundet. Studentersamfundets
lover fastsettes av Studentersamfundet. Valg til tillitsverv, godkjenning av budsjett (se §7.10 Budsjett og regnskap) og
behandling av forslag (se § 15 Behandling av forslag) skal gjøres av Studentersamfundet. Saker som er av uvanlig art
eller av stor betydning skal behandles av Studentersamfundet. Dette omfatter blant annet vedtak om kjøp og salg av
fast eiendom og opptak av langsiktige lån, herunder pantsetting av Studentersamfundets faste eiendom.
Se også § 8 Generalforsamling.

  \end{lovparagraf}
  
  \begin{lovparagraf}{Styret}
  
Styret har ansvaret for avholdelse av samfundsmøter, festlige sammenkomster samt å representere
Studentersamfundet. Styret utarbeider Studentersamfundets politiske profil. Styret sørger for at det alltid blir ført
referat fra møtene og tilstelningene i Studentersamfundet (se §24 Arkivar).

Studentersamfundet velger en leder. Lederen velger selv sitt styre. Styret er sammensatt av minimum 6 medlemmer.
Styremedlemmene skal godkjennes av Studentersamfundet. Valg av leder for Studentersamfundet og valg til andre
tillitsverv kan ikke avholdes på samme møte.

Styrets leder velger selv en nestleder blant sine medlemmer, og fordeler Styrets øvrige oppgaver etter innbyrdes
overenskomst. Nestleder er leders stedfortreder i leders fravær og med leders skriftlige samtykke. Nytt styre tar over
den 17. mai og sitter i ett år eller inntil nytt lovlig styre er valgt.

Styrevedtak fattes i styremøter, som skal avholdes minst en gang i uken (feriene unntatt). Styret er bare
beslutningsdyktig når minst halvparten av medlemmene, deriblant lederen eller nestlederen, er til stede. Styret fører
protokoll over forhandlingene. Alle de medlemmene som er til stede, skal skrive under i protokollen.
Styrearrangementer utenom det vedtatte budsjett må godkjennes av Finansstyret.

  \end{lovparagraf}

\begin{lovparagraf}{Finansstyret og daglig ledelse}

  \begin{enumerate}
  
    \item Finansstyret består av 7-9 personer etter nærmere beslutning i Studentersamfundet. Finansstyret er
    sammensatt av 4-6 medlemmer valgt av generalforsamlingen (se §8 Generalforsamling), leder for
    Studentersamfundet, en representant for gjengene og en representant for de ansatte.
    
    UKEsjefen (se § 25 UKA), daglig leder, representant fra Rådet (se § 9 Rådet) og nyvalgte medlemmer av
    Finansstyret har adgang til møtene som observatører.
    
    \item Valg av medlemmer
    
    Etter innstilling fra Rådet (se §9 Rådet) velger generalforsamlingen leder, nestleder, og 2-4 medlemmer til
    Finansstyret. De velges slik at leder og ett medlem er på valg ett år, og nestleder og ett medlem er på valg
    påfølgende år.
    
    Første ledd gjelder ikke representanter for de ansatte og gjengene ved Studentersamfundet (se §7.3)
    
    \item Det velges en representant av, for og blant de ansatte, med vararepresentant. Tjenestetiden er to år.
    
    Det velges en representant av og for gjengene i Studentersamfundet, med en tjenestetid på ett år.
    
    \item Medlemmene har en tjenestetid på to år. Tjenestetiden regnes fra generalforsamling til generalforsamling når
    noe annet ikke er bestemt. Den opphører ved avslutningen av den ordinære generalforsamling i det året
    tjenestetiden utløper.
    
    Selv om tjenestetiden er utløpt, skal finansstyremedlemmet bli stående i vervet inntil nytt medlem er valgt.
    
    Første og annet ledd gjelder ikke finansstyremedlem som er valgt etter §7.3.
    
    \item Tilbaketreden og avsetting før tjenestetiden opphører
    
    Et finansstyremedlem har rett til å tre tilbake før tjenestetiden er ute dersom særlig grunn foreligger. Finansstyret skal gis rimelig forhåndsvarsel.
    
    Et finansstyremedlem kan avsettes av det organ som har valgt finansstyremedlemmet.
    
    \item Suppleringsvalg
    
    Opphører vervet for et finansstyremedlem før tjenestetiden er ute, skal det velges nytt finansstyremedlem for
    resten av tjenestetiden etter de regler som er angitt i §7.2 og §7.3.
    
    \item Forvaltningen av Studentersamfundet
    
    Finansstyret har ansvaret for forvaltningen av Studentersamfundets eiendommer, kapital og driftsmidler med
    et råd (se §9 Rådet) som kontrollerende instans.
    
    Finansstyret har fullmakt til å handle på vegne av Studentersamfundet i alle saker som berører den
    økonomiske driften av Studentersamfundet og Samfundsbygningen. Finansstyret kan på egen hånd ta opp
    kortsiktige lån på vegne av Studentersamfundet.
    
    Dokumenter som forplikter Studentersamfundet økonomisk i henhold til § 5 Studentersamfundets høyeste
    myndighet, må underskrives av både lederen i Finansstyret og leder for Studentersamfundet for å være gyldige.
    
    Finansstyret skal sørge for forsvarlig organisering av virksomheten. Finansstyret skal fastsette retningslinjer
    for Virksomheten, herunder utarbeide og ajourføre de nødvendige instrukser (se §26 Instrukser og reglementer).
      
    Finansstyret skal holde seg orientert om Studentersamfundets økonomiske stilling og plikter å påse at dets
    virksomhet, regnskap og formuesforvaltning er gjenstand for betryggende kontroll.
    
    Finansstyret iverksetter de undersøkelser det finner nødvendig for å kunne utføre sine oppgaver. Finansstyret
    skal iverksette slike undersøkelser dersom dette kreves av ett eller flere av finansstyremedlemmene.
    
    \item Daglig ledelse
    
    Studentersamfundet skal ha en daglig leder.
    
    Finansstyret tilsetter daglig leder og fastsetter dennes instruks.
    
    Daglig leder står for den daglige ledelse av Studentersamfundets forretningsvirksomhet. I daglig leders
    fravær står Finansstyrets leder for den daglige ledelse.
    
    Daglig leders underskrift forplikter Studentersamfundet økonomisk, i henhold til instruks gitt av Finansstyret.
    
    Den daglige ledelse omfatter ikke saker som etter Studentersamfundets forhold er av uvanlig art eller stor
    betydning. (Se § 5 Studentersamfundets høyeste myndighet).
    
    Daglig leder kan ellers avgjøre en sak etter fullmakt fra Finansstyret i det enkelte tilfellet, eller når
    Finansstyrets beslutning ikke kan avventes uten vesentlig ulempe for Studentersamfundet. Finansstyret skal
    snarest underrettes om avgjørelsen.
    
    Daglig leder skal sørge for at Studentersamfundets regnskap er i samsvar med lov og forskrifter, og at
    formuesforvaltningen er ordnet på en betryggende måte.
      
    Daglig leder skal minst hver tredje måned, i møte eller skriftlig, gi Finansstyret underretning om
    
    Studentersamfundets virksomhet, stilling og resultatutvikling.
    
    Finansstyret kan til enhver tid kreve at daglig leder gir en nærmere redegjørelse om bestemte saker. Slik
    redegjørelse kan også kreves av det enkelte medlem i Finansstyret.
    
    \item Saksbehandling
    
    Finansstyret skal behandle saker i møte. Finansstyret er vedtaksdyktig når minst fem medlemmer, deriblant
    lederen eller nestlederen, er til stede. For å fatte et vedtak i Finansstyret kreves simpelt flertall ved
    avstemning. Ved stemmelikhet gjelder det som møtelederen har stemt for.
    
    Finansstyrebehandlingen ledes av Finansstyrets leder, eller av Finansstyrets nestleder i lederens fravær.
    
    Finansstyrets leder skal sørge for behandling av aktuelle saker som hører inn under Finansstyret.
    
    Finansstyremedlemmene og daglig leder kan kreve at Finansstyret behandler bestemte saker.
    
    Daglig leder forbereder saker som skal behandles av Finansstyret i samråd med Finansstyrets leder.
    
    En sak skal forberedes og fremlegges slik at Finansstyret har et tilfredsstillende behandlingsgrunnlag.
    
    Finansstyrebehandling varsles på hensiktsmessig måte og med nødvendig frist. Ved innkalling skal saksliste
    refereres.
    
    Daglig leder har rett og plikt til å delta i Finansstyrets behandling av saker og til å uttale seg, med mindre
    annet er bestemt av Finansstyret i den enkelte sak.
    
    I ekstraordinære tilfeller kan saker behandles uten møte dersom samtlige medlemmer av Finansstyret godtar
    dette. Finansstyret kan ikke fatte noe vedtak dersom ikke alle medlemmene har fått anledning til å ta del i
    saksbehandlingen.
    
    Årsregnskap og årsberetning skal behandles i møte.
    
    \item Budsjett og regnskap
    
    Finansstyret skal utarbeide forslag til budsjett for Studentersamfundet. Budsjettforslaget skal kunngjøres på
    et av de siste møtene i høstsemesteret og minst 14 dager før det skal behandles. Rådets innstilling skal gjøres
    kjent for Studentersamfundet før behandling.
    
    Studentersamfundets behandling av budsjettet skal finne sted innen 10.februar i budsjettåret. Ved alternative
    innkomne budsjettforslag (se § 15.5 Forslag av økonomisk art) skal Finansstyret gi en innstilling som skal
    kunngjøres etter vanlige regler for møteinnkalling (se § 11 Samfundsmøte).
    
    De samlede reviderte regnskapene for Studentersamfundet med revisors beretning skal, sammen med
    innstilling fra Rådet og en årsmelding fra Finansstyret, legges fram for Generalforsamlingen til godkjenning.
    (se §8 Generalforsamling).
    
    \item Inhabilitet
    
    Et medlem i Finansstyret må ikke delta i behandlingen eller avgjørelsen av spørsmål som har slik særlig
    betydning for egen del eller for noen nærstående at medlemmet må anses for å ha fremtredende personlig
    eller økonomisk særinteresse i saken. Det samme gjelder for daglig leder.
    
    \item Misbruk av posisjon i Studentersamfundet
    
    Finansstyret må ikke foreta noe som er egnet til å gi andre virksomheter urimelige fordeler på
    Studentersamfundets bekostning.
    
    Finansstyret og daglig leder må ikke etterkomme noen beslutning i Generalforsamlingen eller
    Samfundsmøte, dersom beslutningen strider mot lov eller mot Studentersamfundets lover eller statutter.
    
    \item Finansstyreprotokoll
    
    Møteinnkallelse med saksliste og saksunderlag skal kunngjøres offentlig samtidig med innkalling til
    Finansstyremøte.
    
    Det skal føres protokoll over Finansstyrets behandling.
    
    Den skal minst angi tid og sted, deltakerne, behandlingsmåte, beslutninger og vedtak. Det skal fremgå i
    protokollen at saksbehandlingen oppfyller kravene i §7.9 Saksbehandling.
    
    Protokollen skal offentliggjøres så snart som mulig, og senest en uke etter møtet er avholdt.
    
    Dersom Finansstyrets beslutning ikke er enstemmig, skal det angis hvem som har stemt for og imot.
    
    Finansstyremedlem og daglig leder som ikke er enig i en beslutning, kan kreve sin oppfatning innført i
    protokollen.
    
    Protokollen skal underskrives av alle de medlemmer som har deltatt i finansstyrebehandlingen.

   
    \end{enumerate}
    
   \end{lovparagraf}
   
   \begin{lovparagraf}{Generalforsamling}
   
Ordinær generalforsamling blir avholdt hvert år innen utgangen av april. Ekstraordinær generalforsamling holdes når
det besluttes av Styret eller skriftlig forlanges av Finansstyret, Rådet, eller Studentersamfundets revisor.

Generalforsamlingen innkalles av Styret med minst 14 dagers varsel. Årsmelding og regnskap fra Finansstyret skal
kunngjøres samtidig med innkalling til ordinær generalforsamling. Innkomne saker på dagsorden må være Styret i
hende senest fire dager før generalforsamlingen. Leder for Studentersamfundet åpner generalforsamlingen.

Dagsorden for ordinær generalforsamling skal alltid inneholde følgende punkter:

\begin{itemize}
\item Valg av ordstyrer
\item Godkjenning av dagsorden
\item Årsmelding
\item Regnskap
\item Innkomne saker
\item Valg av medlemmer til Finansstyret (Se §7.2)
\end{itemize}

Forøvrig gjelder kapittel 3 om Studentersamfundets møter.

    \end{lovparagraf}
    
    \begin{lovparagraf}{Rådet}
    
Rådet ser til at den økonomiske driften av Studentersamfundet skjer i samsvar med lover, instrukser og vedtak i
Studentersamfundet. Rådet skal gå gjennom budsjett for Studentersamfundet og avgi innstilling om det til
Studentersamfundet. Rådet skal gå gjennom de samlede reviderte regnskapene for Studentersamfundet, på grunnlag
av merknader fra revisoren, og avgi innstilling om dem til Generalforsamlingen. Rådet skal gi innstilling til valg av
Finansstyrets Samfundsvalgte medlemmer og revisor (se §10 Revisjon). Innstillingene skal refereres på det siste
ordinære møtet i Studentersamfundet før valget. Før endelig avstemning over et lovforslag skal Rådet behandle det og
se til at forslaget formelt sett er i orden. I disiplinærsaker (se §33 Disiplinærmakt) fungerer Rådet som ankeinstans.

Rådet består av seks medlemmer. Tjenestetiden er to år, og valg avholdes slik at tre medlemmer er på valg hvert år.
Tjenestetiden følger kalenderåret. Styret innstiller til valg av nye medlemmer til Rådet. Innstillingen skal refereres på
det siste ordinære møtet i Studentersamfundet før valget. Rådet velger selv leder og nestleder blant sine medlemmer.
Nestleder er leders stedfortreder i leders fravær og med leders skriftlige samtykke.

Lederen innkaller Rådet når dette er nødvendig for å kunne være på linje med Rådets plikter, eller når et medlem
krever det. Sakene avgjøres med alminnelig flertall. Ved stemmelikhet er lederens stemme avgjørende. Rådet er
vedtaksdyktig når minst fire medlemmer, deriblant lederen eller nestlederen, er tilstede. Rådet fører protokoll over
møtene sine. Protokollen blir underskrevet av de tilstedeværende medlemmene og er tilgjengelig for alle medlemmer
av Studentersamfundet.

Rådet ajourfører Studentersamfundets lover slik at gjeldende lovtekst til enhver tid er tilgjengelig. Ved tvil om
tolkningen av lovteksten skal Rådets tolkning legges til grunn.

  \end{lovparagraf}
  
  \begin{lovparagraf}{Revisjon}
  
Etter innstilling fra Rådet velger Studentersamfundet en offentlig autorisert revisor som gjennomgår årsregnskapet for
Studentersamfundet. Revisors lønn fastsettes av Finansstyret.

  \end{lovparagraf}
  
\end{lovkapittel}





\begin{lovkapittel}{Studentersamfundets møter}

  \begin{lovparagraf}{Samfundsmøte}
  
Samfundsmøte holdes ordinært lørdag kveld i Universitetes semestre. Dagsorden for hvert møte, med den nøyaktige
ordlyden på resolusjonsforslag, mistillitsforslag, andre forslag og interpellasjoner, skal være bekjentgjort på
Studentersamfundets oppslagstavler 48 timer før møtet. Møtet skal også kunngjøres for byens øvrige befolkning.
   
   \end{lovparagraf}
   
   \begin{lovparagraf}{Adgang til møtene}
   
Følgende har adgang til Samfundsmøte:

\begin{itemize}
  \item Alle med gyldig medlemskort
  \item Fast vitenskapelige ansatte ved norske universiteter og høgskoler
  \item Medlemmer av andre norske studentsamfunn og medlemmer av norske studentlag i utlandet
\end{itemize}

Styret kan forøvrig innby til møtene hver og en som er ønsket der, enten for hele semesteret eller for hvert enkelt
møte. Ellers kan andre ikkemedlemmer få adgang til møtene så langt Styret finner dette tjenlig. Disse må for hvert
møte kjøpe et styrekort, til den prisen som Styret og Finansstyret fastsetter. Styret kan etter søknad gi slik adgang for
alle møtene i semesteret, med kontingent som Styret og Finansstyret fastsetter.

  \end{lovparagraf}
  
  \begin{lovparagraf}{Dagsorden}

Når møtet er satt, skal de viktigste styrevedtakene for tiden etter det forrige Samfundsmøtet refereres. Referatet skal
inneholde foreløpig dagsorden for møtet. Forslag til saker på dagsorden skal være Styret i hende senest kl.12.00 to
dager før møtet. Spesielle frister gjelder for lovforslag, resolusjonsforslag, interne meningsytringer, mistillitsforslag
og budsjettforslag (se §15 Behandling av forslag og §7 Finansstyret og daglig ledelse). Styret har plikt til snarest
mulig å slå opp en hvilken som helst sak som kommer fra navngitte medlemmer av Studentersamfundet dersom
forslaget er kommet inn før utgangen av fristen. Styret har videre plikt til å føre opp på dagsorden saker som er
kommet inn før utgangen av fristen.

Inntil dagsorden er godkjent, kan Styret og ethvert medlem få en sak som haster, ført opp på dagsorden dersom
Studentersamfundet vedtar dette med 2/3 flertall. Dette gjelder ikke lovforslag, mistillitsforslag og budsjettforslag (se
§15 Behandling av forslag og §7 Finansstyret og daglig ledelse). Studentersamfundet godkjenner dagsorden med
alminnelig flertall før møtet går over til neste sak. Deretter kan ingen sak føres opp som eget punkt, men må i tilfelle
tas opp under eventuelt. Det kan da heller ikke holdes noen avstemning. Saker som står på den godkjente dagsorden,
kan bare strykes med 2/3 flertall. Samme flertall kreves til senere endring av rekkefølgen på sakene.

  \end{lovparagraf}
  
  \begin{lovparagraf}{Møteledelse og talerett}

Et av medlemmene i Styret leder møtet. I spesielle tilfeller kan Studentersamfundet likevel vedta å velge en ordstyrer
utenom Styret. I saker der debatt er ført særskilt opp på dagsorden eller der Studentersamfundet skal gjøre et vedtak,
har enhver med lovlig adgang til møtet rett til å ta ordet. Dersom flere krever ordet samtidig, avgjør ordstyreren
rekkefølgen. Ordstyreren må ha tillatelse fra Studentersamfundet til å begrense taletiden eller stanse inntegning av nye
talere. Studentersamfundet kan etter forslag fra ordstyreren eller fra andre medlemmer med 2/3 flertall vedta å stanse
debatten når det finner at saken som foreligger er godt nok behandlet. På samme vilkår kan taletiden begrenses med
virkning for allerede inntegnede talere.

  \end{lovparagraf}
  
  \begin{lovparagraf}{Behandling av forslag}

Forslag kan bare settes under votering når de er i skriftlig form og er oppført på den vedtatte dagsorden (se §13
Dagsorden).

Endringsforslag kan fremmes samtidig med behandling av forslagene. Ved lovforslag, budsjettforslag og
resolusjonsforslag kan bare mindre vesentlige eller formelle endringer gjøres, og da bare dersom disse blir godtatt av
den eller de som har fremmet forslaget. Utsettelse av forslag anses som behandling av forslag.

Spesielt gjelder for:
\begin{enumerate}
\item Lovforslag skal slås opp på Studentersamfundets tavler minst en uke før det kommer opp til behandling og
kunngjøres på Samfundsmøte i nøyaktig ordlyd og med opplysning om navnet på forslagsstilleren. Før
endelig avstemning over et lovforslag skal Rådet behandle det (se §9). Rådets uttalelse skal gjøres kjent før
avstemning.

Lovforslaget kan ikke tas opp til ny avstemning i Studentersamfundet før det er gått minst åtte uker siden den
forrige avstemningen, dersom et flertall i salen stemte mot forslaget. §36 Avvikling kan bare endres etter
vedtak på to påfølgende generalforsamlinger.

\item Resolusjoner er alle selvstendige meningsytringer som blir vedtatt av Studentersamfundet, og som direkte tar
sikte på utenverdenen. Resolusjonsforslag skal være Styret i hende senest fire dager før møtet, og forslag til
endret ordlyd senest to dager før møtet.

\item Interne meningsytringer er meningsytringer som gjelder indre forhold i Studentersamfundet, rettet til Styret,
et annet organ eller en enkelt tillitsrepresentant. Interne meningsytringer skal behandles som
resolusjonsforslag (pkt. 2).

\item Mistillitsforslag

Forslag som inneholder mistillit til Styret, et annet organ eller en enkelt tillitsrepresentant, skal være Styret i
hende senest fire dager før møtet. Enhver kan likevel når som helst selv sette stillingen sin inn på utfallet av
en saksbehandling. Dersom Studentersamfundets leder går av etter mistillit, går også resten av
styremedlemmene av. Valg av leder for Studentersamfundet blir da ledet av lederen i Rådet. Dersom noen
har gått av som følge av samfundsvedtak, skal nyvalg holdes snarest mulig på Samfundsmøte.

\item Forslag av økonomisk art
Studentersamfundet kan ikke gjøre noe vedtak av økonomisk art uten at saken har vært behandlet i
Finansstyret.

Forslag om endring av budsjett fra andre enn Finansstyret må være Styret i hende senest en uke før en ønsker
at det skal behandles av Studentersamfundet. (Se § 7.10 Budsjett og regnskap).

\end{enumerate}

  \end{lovparagraf}
  
  \begin{lovparagraf}{Interpellasjoner}
  
Interpellasjoner er spørsmål om Studentersamfundets indre saker, rettet til Styret, et annet organ eller en enkelt
tillitsrepresentant. Den som interpellasjonen er rettet til, kan før han eller hun svarer kreve at denne blir grunngitt på
møtet. En interpellasjon må stå som eget punkt på dagsorden (se § 13 Dagsorden).
  
  \end{lovparagraf}

\end{lovkapittel}

\begin{lovkapittel}{Valg og avstemninger}
  
  \begin{lovparagraf}{Stemmerett og valgbarhet}

Medlemmer med gyldig medlemskort er stemmeberettigede og valgbare i Studentersamfundet. Ethvert medlem har
plikt til å ta imot valg som tillitsrepresentant. Alle kan si fra seg gjenvalg for like lang tid som de har tjenestegjort.
Ingen har plikt til å sitte med mer enn ett tillitsverv av gangen.

  \end{lovparagraf}
  
  \begin{lovparagraf}{Avstemning}
  
Avstemning skjer som regel ved håndsopprekning med medlemskort. Skriftlig avstemning (urnevalg) skal holdes
dersom minst ti medlemmer krever det. Ved valg der det er kommet forslag om flere kandidater enn det som skal
velges, skal avstemningen være skriftlig. Skriftlig avstemning skal være hemmelig. Ordstyreren skal si fra når
avstemningen begynner og når den slutter. Utenom dette tidsrommet kan ikke avstemning skje. Innenfor dette
tidsrommet kan ingen tale, dersom det ikke gjelder avstemningsordenen. Det kan ikke stemmes ved forfall, hverken
ved fullmektig eller med deponering av stemmeseddel.
  
  \end{lovparagraf}
  
  \begin{lovparagraf}{Valg}

For tillitsrepresentanter som skal velges av Studentersamfundet, skal ordinært valg avholdes før funksjonstiden til de
sittende tillitsrepresentantene utløper. Valg samt eventuell innstilling skal kunngjøres på Samfundsmøte minst en uke
før valget. Likevel kan valg av leder for Studentersamfundet (se §6 Styret) uhindret av dette finne sted på det siste
ordinære møtet i vårsemesteret.

En kandidat er bare valgt dersom vedkommende har fått minst halvparten av de avgitte stemmene. I motsatt fall skal
det være ny avstemning mellom de to kandidatene som har fått flest stemmer. Blir det likt stemmetall ved valg skal
det holdes nytt valg på neste ordinære eller, om nødvendig, på et ekstraordinært møte.
  
  \end{lovparagraf}
  
  \begin{lovparagraf}{Vedtak}
  
For at et vedtak skal kunne gjelde må minst 100 stemme-berettigede medlemmer ha stemt for. Forslag vedtas med
alminnelig flertall dersom ikke annet er fastsatt i lovene. Ved likt stemmetall er stemmen til ordstyreren avgjørende.
Vedtak om endring av lovene, endring av budsjett, kjøp eller salg av fast eiendom eller opptak av lån, krever 2/3
flertall.
  
  \end{lovparagraf}
  
  \begin{lovparagraf}{Referat}

Alle valg og vedtak i Studentersamfundet skal straks føres inn i Studentersamfundets vedtaksprotokoll. Utskrift av
vedtaksprotokollen skal bekjentgjøres på Studentersamfundets oppslagstavler senest 48 timer etter samfundsmøtet og
skal stå der en uke. Vedtatte resolusjoner skal sendes ut til: angjeldende myndigheter, lokale media, Norsk
telegrambyrå og utvalgte riksdekkende media senest dagen etter samfundsmøtet.
  
  \end{lovparagraf}
    
\end{lovkapittel}

\begin{lovkapittel}{Indre liv}

  \begin{lovparagraf}{Gjengene}
  
Det er opprettet gjenger til å løse spesielle oppgaver innen Studentersamfundet.

Gjengen velger selv sine nye gjengmedlemmer blant Studentersamfundets medlemmer. Alle gjengmedlemmer er å
betrakte som tillitsrepresentanter for Studentersamfundet, og er ansvarlige overfor Studentersamfundet i henhold til §
15.4 Mistillitsforslag og § 16 Interpellasjoner.

Finansstyret kan opprette og legge ned gjenger ut fra Studentersamfundets behov til enhver tid.

Gjengenes sammensetning, plikter og fullmaktsområder blir fastsatt i en instruks av Finansstyret i samråd med Styret
og gjengene (se §27 Instrukser og reglementer). Alle gjenger ledes av en gjengsjef som er ansvarlig overfor
Finansstyret. Gjengsjef skal velges av medlemmene i vedkommende gjeng, og skal presenteres for
Studentersamfundet. Tjenestetiden for en gjengsjef er normalt ett år.

Der det er konflikt mellom denne paragraf og statuttene til Studentradio'n, Under Dusken og Student-TV, er statuttene
overordnet.

  \end{lovparagraf}
  
  \begin{lovparagraf}{Mediastud AS}
  
Studentersamfundet eier andeler i selskapet Mediastud AS. Mediastud AS har som formål å utgi Studentersamfundets
trykte blad "Under Dusken" og å drive Studentradio’n og Student-TV, samt alt som står i forbindelse med dette,
herunder å delta i andre former for medievirksomhet.

For Under Dusken, Studentradio’n og Student-TV gjelder det egne statutter. Valg av medlemmer til styret i Mediastud
AS skjer i henhold til selskapets statutter.

Protokoll for generalforsamling i Mediastud AS refereres for Studentersamfundet. Protokoll for styremøter i
Mediastud AS refereres for Finansstyret.

  \end{lovparagraf}
  
  \begin{lovparagraf}{Arkivar}
  
Finansstyret ansetter en arkivar som ordner med og har ansvaret for arkivet i Studentersamfundet. Tillitsrepresentanter
for Studentersamfundet skal sende utskrevne protokoller og all korrespondanse fra sin tjenestetid til arkivaren.
Arkivaren skal tilse at dette blir gjort.
  
  \end{lovparagraf}
  
  \begin{lovparagraf}{Studentersamfundets håndskrevne avis "Propheten"}
  
Til redaksjonen av Studentersamfundets håndskrevne avis "Propheten" kan det velges to redaktører som er villige.
Redaktørene fungerer ett semester.
  
  \end{lovparagraf}
  
  \begin{lovparagraf}{UKA}
  
UKA arrangeres hvert annet år. Ukesjefen velges av Studentersamfundet, og velger selv sine medarbeidere blant
Studentersamfundets medlemmer. UKA utarbeider sitt eget budsjett som godkjennes av Finansstyret.

  \end{lovparagraf}
  
  \begin{lovparagraf}{Instrukser og reglementer}
  
Alle instrukser og reglementer til rettledning for tillitsrepresentanter i Studentersamfundet blir utarbeidet av
Finansstyret i samråd med tilltsrepresentanten/-e. Finansstyret skal sørge for at gjenpart av alle instrukser og
reglementer er tilgjengelig på Studentersamfundet, samt at instrukser og reglementer mangfoldiggjøres til de
tillitsrepresentanter som de ulike instrukser og reglementer er relevante for.
  
  \end{lovparagraf}
  
  \begin{lovparagraf}{Medlemskap for tillitsrepresentanter}

Alle tillitsrepresentanter skal i sin tjenestetid ha gyldig medlemskap i Studentersamfundet.

  \end{lovparagraf}

\end{lovkapittel}


\begin{lovkapittel}{Samfundsretten}

  \begin{lovparagraf}{Nedsettelse av Samfundsretten}
  
Samfundsrett blir etter oppfordring nedsatt for å avgjøre tvister mellom samfundsmedlemmer. Retten er sammensatt
av 7 medlemmer og 2 varamedlemmer.

  \end{lovparagraf}
  
  \begin{lovparagraf}{Valg av rettens medlemmer}

Når ønske om nedsettelse av samfundsrett er innkommet til Styret, velger Studentersamfundet 13 rettsmedlemmer.

Hver av partene velger så bort to av dem. De to som har fått færrest stemmer av de ni gjenværende fungerer som
varamedlemmer.

Ingen som Studentersamfundet betrakter som inhabil kan være medlemmer av retten. Medlemmer av det sittende styre
kan ikke innvelges.

  \end{lovparagraf}
  
  \begin{lovparagraf}{Forretningsorden}
  
Retten velger selv sin leder og bestemmer selv fremgangsmåten for undersøkelse og avgjørelse i saken. Retten kan
avvise saken. Hver av partene kan velge en prosessfullmektig. Varamedlemmene deltar i forhandlingene for retten,
men er ikke med i dennes rådslagninger.
  
  \end{lovparagraf}
  
  \begin{lovparagraf}{Kjennelse}
  
Retten skal komme med skriftlig kjennelse som må være ledsaget av premisser. For at dom skal kunne avsies, må alle
medlemmene ha vært til stede ved alle forhandlingene for retten. Rettens avgjørelse er endelig.
  
  \end{lovparagraf}
  
  \begin{lovparagraf}{Straff}
  
I saker der disiplinærstraff blir regnet som nødvendig, benytter Samfundsretten straff som nevnt under §34
(Disiplinærmakt).

  \end{lovparagraf}
  
\end{lovkapittel}



\begin{lovkapittel}{Alminnelige bestemmelser}

  \begin{lovparagraf}{Disiplinærmakt}

Styret kan idømme samfundsmedlemmer straff for forseelser av enhver art. Straffene er :
\begin{enumerate}
\item Erstatning,
\item Eksklusjon fra Studentersamfundet for kortere eller lengre tid,
\item Andre tiltak dersom det blir funnet mer tjenlig.
\end{enumerate}

Erstatningen inndrives av Finansstyret etter innberetning fra Styret. Den straffede eller leder for kontrollørene (jfr.
kontrollørinstruksen) kan anke straffeutmålingen inn for Rådet, som kan avgi dom i saken. Rådets avgjørelse er
endelig dersom Rådet ikke finner det nødvendig å sende saken til Samfundsrett.

  \end{lovparagraf}
  
  \begin{lovparagraf}{Festdager}

Studentersamfundet feirer som akademiske festdager:
\begin{enumerate}
\item Immatrikuleringsdagen
\item Studentersamfundets stiftelsesdag 1.oktober.
\end{enumerate}

  \end{lovparagraf}
  
  \begin{lovparagraf}{Avvikling}

I tilfelle avvikling av Studentersamfundet i Trondhjem skal alle eiendeler og andre aktiva tilfalle et fond med formål å
opprette et nytt studentsamfunn i Trondheim. Fondet som opprettes ved en eventuell avvikling skal forvaltes av
høyeste styringsorgan ved Norges teknisk-naturvitenskapelige universitet.

Denne paragrafen kan bare endres etter vedtak på to påfølgende generalforsamlinger.  

  \end{lovparagraf}

\end{lovkapittel}