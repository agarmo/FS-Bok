\chapter*{Vedvarende vedtak i Finansstyret}
\addcontentsline{toc}{chapter}{Vedvarende vedtak i Finansstyret}

\begin{instruksledd}{SAK 24/04 Tilrettelegging for Handikappede}
    Samskipnaden i Trondheim (SiT) har et ``Fond for spesielle velferdstiltak''. 21.04.04
    vedtok Velferdstinget følgende:
    Velferdstinget oppfordrer Studentersamfundet til å lage en plan for tilrettelegging
    for funksjonshemmede.
    Velferdstinget vil jobbe for at SiT skal være med på finansieringen av slik
    tilrettelegging.
    
    Til Samfundets bursdag fikk Samfundet overrakt en gave på 1,5 MNOK fra
    Studentsamskipnaden i Trondheim,
    øremerket tilrettelegging for handikappede.
    Det foreslås at Finansstyret nedsetter et utvalg som kan utrede hvilke
    tilretteleggingstiltak for handikappede som kan
    gjøres på Samfundet.


    \textbf{Vedtak:}

    Finansstyret nedsetter et utvalg som skal utrede hvilke tilretteleggingstiltak for
    handikappede som kan gjøres på
    Samfundet. Utvalget ledes av daglig leder Bjørn Granum, og skal i tillegg bestå av ett
    medlem fra Samfundets
    byggekomite (SBK), Mari Nordin – Velferdstinget, Tormod Jensen – Handicapforbundet og
    Einar Hallgren – Styret.
    
    Utvalget har som mandat å se på mulige tilretteleggingstiltak på Samfundet for
    handikappde, utrede kostnader og
    foreslå prioritering av tiltak. Innstilling forelegges Finansstyret innen mai 2005.
    Utvalget må se tiltakene i sammenheng med utbyggingen av Fengselstomta og øvrige
    tiltak i Samfundsbygningen.

\end{instruksledd}

\begin{instruksledd}{SAK 26/04 Avvikling av Jubileumsfondet av 1935}
    Styret i stiftelsen Jubileumsfondet av 1935 har vedtatt å avvikle stiftelsen og
    utbetale midlene til Studentersamfundet i
    Trondhjem. Det foreslås å avvikle Samfundets låneforhold til stiftelsen, herunder
    sletting av panteobligasjoner, for
    deretter å overføre alle midlene fra stiftelsen til vårt eget fond, Jubileumsfondet av
    1985.

    \textbf{Vedtak:}

    Finansstyret går inn for å avvikle låneforholdet til stiftelsen ”Jubileumsfondet av
    1935 for Studentersamfundet i
    Trondhjem”.
    
    Finansstyret går inn for at midler som utbetales Studentersamfundet i forbindelse med
    avvikling av denne stiftelsen
    tilføres Studentersamfundets fond ”Jubileumsfondet av 1985”.

\end{instruksledd}

\begin{instruksledd}{SAK 10/05 Jubileumsbok}
    Høsten 2004 ble Monica Rolfsen bedt om ta initiativ til å etablere en prosjektgruppe
    som kan sette i gang arbeidet
    med jubileumsboken. Det er etablert en arbeidsgruppe som blant annet har gjennomgått
    arkivmateriale og
    utgivelsesmodeller. Det foreslås å arbeide videre med utgangspunkt i en utgivelse med
    tre bind: En historisk
    gjennmgang, et festskrift med overordnede utviklingstrekk gjennom 100 år og en
    faktadel/oppslagsverk. I tillegg
    kommer DVDer med filmer og bilder. Det må regnes med kostnader til avlønning av
    forfatter og digitalisering av
    materiale.
    
    
    \textbf{Vedtak:}

    Finansstyret nedsetter Jubileumsbokprosjektet 2010, en prosjektgruppe som vil
    forberede utgivelse av en bok i
    forbindelse med Studentersamfundets 100 års jubileum i 2010.
    
    Til prosjektet oppnevnes Monica Rolfsen som leder, samt Håvard Hamnaberg, Ingar
    Pareliussen, Idar Lind, Magne
    Mæhre og Bernt Gran som medlemmer.
    
    Jubileumsbokprosjektet rapporterer til Finansstyret, som oppnevner leder av
    prosjektet. Prosjektet vil supplere seg
    selv og kan rekruttere ytterligere medlemmer ved behov.
    
    Jubileumsbokprosjektets økonomiske aktivitet inngår i Studentersamfundets
    driftsregnskap. Finansstyret tar sikte på å
    bevilge midler fra Jubileumsfondet av 1985 til større utgifter i forkant av
    utgivelsen.


\end{instruksledd}

\begin{instruksledd}{SAK 13/05 Studentenes fredspris – bevilgning til fredsprisfondet}
    Studentersamfundet ga i 1999 25 tusen kroner til Studentenes fredspris og disse
    midlene ble delt ut under festivalen i
    1999. Fondet for Studentenes fredspris ble etablert i 2001 og forvaltes av Stiftelsens
    ISFiTs styre. Stiftelsen ISFiT og
    to av stifterinstitusjonene har bidratt med 150 000 kroner til fondet. Styret foreslår
    at Studentersamfundet, som en av
    stifterene av ISFiT, bidrar med tilsvarende beløp og vil fremme forslaget for
    Studentersamfundet til behandling i møte
    27. april 2005. Finansstyret skal etter lovene behandle forslag av økonomisk art før
    de behandles av
    Studentersamfundet.
    
    \textbf{Vedtak:}

    Finansstyret gir sin tilslutning til Styrets forslag om å bevilge 150.000 kroner til
    fondet for Studentenes fredspris.

\end{instruksledd}

\begin{instruksledd}{SAK 22/05 Bevilgning til fondet for Studentenes fredspris}
    Fondet for Studentenes fredspris har som mål å finansiere prispengene og
    administrasjonsutgiftene til Studentenes
    fredspris som utgis annen hvert år under ISFiT.

    Styret fremmet sak for Finansstyret i møte 5. april om at Styret ønsket at
    Studentersamfundet skulle bidra med 150
    000 kroner til fondet. Fra før av har NTNU og Samskipnaden bidratt til fondet med
    samme beløp. I sak 13/05 ga
    finansstyret sin tilslutning til at saken om studentenes fredspris fremmes for
    Storsalen 27. april.
    
    27. april vedtok Studentersamfundet enstemmig følgende: ”Studentersamfundet i
    Trondhjem samlet til Samfundsmøte
    den 27. april 2005 vedtar å støtte Fondet for studentenes fredspris, med kroner 150
    000. Studentersamfundet
    oppfordrer stifterne HiST og Trondheim kommune til å gjøre det samme.”
    
    \textbf{Vedtak:}

    Finansstyret tar vedtaket i Samfundsmøtet til etterretning og bevilger 150.000 kroner
    til Fondet for Studentenes
    fredspris. Bevilgningen belastes UKEfond.

\end{instruksledd}

\begin{instruksledd}{SAK 36/05 Etablering av prosjektgruppe for Alumniløsning}
    Samfundet og UKA har i lengre tid sett behovet for en alumniløsning. SU har sett på
    behov, muligheter og krav til en
    slik løsning.
    
    \textbf{Vedtak:}

    Finansstyret gir SU mandat til å opprette og rekruttere til en prosjektgruppe. Gruppen
    kan bestå av inntil 5
    medlemmer og rapporterer direkte til Finansstyret.

\end{instruksledd}

\begin{instruksledd}{SAK 01/06 Resolusjonsforslag ``Samfundet for Fair Trade''}
    Samfundet har mottatt en oppfordring fra en av sine medlemmer om å benytte ``Fair
    Trade''-varer fra Max Havelaar.
    
    \textbf{Vedtak:}

    Finansstyret ser intet til hinder for å benytte Max Haavelaar-produkter gitt at dette
    ikke gir vesentlige økonomiske
    konsekvenser.

\end{instruksledd}


