
\begin{instruks}{Brannvern ved Studentersamfundet i Trondhjem}{1. januar 2008}{3. april 2008}
    
Ved Studentersamfundet i Trondheim er Finansstyrets leder, øverste brannansvarlig. Dette ansvaret er ved
Studentersamfundet delegert til daglig leder og sikringssjefen. Sikringssjefen har sammen med Husmann ansvaret for
de organisatoriske brannverntiltakene, og de har oppnevnt en rekke områdeansvarlige ved de ulike offentlige lokalene.
Disse rapporterer til Husmann eller sikringssjefen ved avvik innenfor brannsikkerheten.
Hvis brannalarmen utløses er det til en hver tid definert hvem som har kommando over situasjonen, og som de
områdeansvarlige skal forholde seg til. (vedlegg A) 

Som en del av sikkerhetsarbeidet ved Samfundet skal alle gjengsjefer og funksjonærer ha vært gjennom et
kontrollørkurs med brannopplæring, holdt av Husmann. Videre skal gjengsjefer informere sine kollegaer om dette skrivet 
i tillegg til annen nødvendig informasjon om det området de har ansvaret for. 

Det er ikke for å frata Studentersamfundet noe ansvar ved eventuelle kritiske situasjoner, men for å være sikker på at
alle gjengmedlemmer har forstått rutinene og ansvaret for egen sikkerhet.

\begin{instruksledd}{Generell branninstruks for alle Gjengmedlemmer}
   
    Den generelle branninstruksen kan sammenfattes i følgende begreper:
    
    \begin{center}
        \textbf{REDDE -- VARSLE -- SLUKKE -- BEGRENSE -- VEILEDE}
    \end{center}
    
    I en brannsituasjon er faktorenes orden i denne branninstruksen likegyldig, dvs. at
    tiltakene ikke skal iverksettes i en
    bestemt rekkefølge. Det viktigste er at man holder hodet kaldt, unngår panikk, og
    handler ut fra den konkrete
    situasjonen der og da. Dvs. at man iverksetter de tiltak som man selv føler er
    riktige, og mest hensiktsmessige.

    \begin{description}
        \item \textbf{Redde} Forsøke å redde hjelpetrengende (f.eks bevegelseshemmede) som man
            eventuelt oppdager. Redningsaksjoner skal kun
            gjennomføres i den utstrekning de ikke medfører fare for eget eller andres liv
            og helse.
        \item \textbf{Varsle} Oppdages et branntilløp er det svært viktig at brannvesenet varsles
            umiddelbart (manuell brannmelder eller telefon
            110), forutsatt at man ikke straks får kontroll over branntilløpet.
            Vakthavende vil alltid være i kontakt med
            brannvesenet, men ring du også for sikkerhets skyld. Branntilløp som man selv
            får kontroll over og slukket, skal
            meldes til din lokaleansvarlig, som så varsler Husmann og/eller sikringssjef.
            Hvis brannalarmen går blir brannvesenevarslet automatisk. Det kan likevel være
            fornuftig å ringe for å bekrefte alarmen overfor brannvesenet dersom du vet
            at det er en reell brann.
        \item \textbf{Slukke} I den utstrekning det er mulig med de hjelpemidler man har for hånden skal man
            forsøke å slukke branntilløp som
            oppdages. Men igjen er det viktig å understreke at dette bare skal
            gjennomføres i den grad det ikke setter eget eller
            andres liv og helse i fare. Det er viktig å være oppmerksom på at en brann
            utvikler seg svært raskt, og at et forsøk på å
            slukke en brann må skje raskt. Lær deg derfor hvor slukkeutstyret ved din
            arbeidsplass er plassert og hvordan dette
            brukes.
        \item \textbf{Begrense} Ved branntilløp og brannalarm har alle plikt til å foreta
            forebyggende tiltak med tanke på å begrense skadevirkningene
            av brannen/branntilløpet. Slike tiltak kan være å forsikre seg om at dører og
            vinduer er lukket i de rommene man
            forlater (men ikke lås av hensyn til brannvesenet som kan behøve å komme inn),
            at gassflasker ol. som benyttes
            stenges igjen før de forlates, at elektrisk apparatur (f.eks PC) slås av og at
            man sjekker at branndører og andre tekniske
            innretninger med brannhemmende formål fungerer slik de skal når de passeres,
            eventuelt lukke disse manuelt dersom
            teknikken ikke virker. Dersom du er i tvil om hva dette gjelder er det bare å
            spørre Husmann eller sikringssjef.
        \item \textbf{Veilede} Samtidig som lokalitetene evakueres har man plikt til å veilede og
                    informere andre som trenger det om hva som er i
                    ferd med å skje, og hva den reelle situasjon er slik man selv har
                    oppfattet det. Det er viktig at vakthavende og
                    brannvesenet får beskjed dersom du vet noe om det som har hendt.
    \end{description}
            
    \textbf{Det er den enkeltes \emph{plikt} å forlate bygget når brannalarmen går.} Velg korteste vei
       ut av bygget. Hold deg godt klar av inngangene slik at alle kan komme seg ut og brannvesenet kan komme seg inn.
       \textbf{Det er ikke tillatt å oppholde seg rett ved inngangspartiene til bygget.} Om mulig går alle til det området utenfor
       Samfundet som er avtalt for ens arbeidsplass eller hybel.

       \textbf{Det vil varsles over høytalerne om når alle kan gå inn igjen. Dette signaliserer at
       brannvesenet har klarert stedet.}

   
    \end{instruksledd}



    \begin{instruksledd}{Brukerinstruks for alle gjengmedlemmer}

        
        Den generelle branninstruksen gjelder når en brann oppdages og/eller når
        brannalarmen går. Brukerinstruksen er en
        forebyggende instruks med formål å ivareta bygningsmassens branntekniske standard,
        slik at denne til enhver tid er
        optimal etter forutsetningene. Brukerinstruksen er i samsvar med Forskrift om
        brannforebyggende tiltak og brannsyn,
        og gjengis delvis her:
 

        \begin{center}\textbf{En bruker av et bygg har ansvar for}\end{center}


            \begin{enumerate}
                \item Ikke ødelegge brannverntiltak som er iverksatt. Dette gjelder rømningsveier, nødlys, brannalarmanlegg,
                    slukkeutstyr, brannvegger, branndører mv.
                \item Rapportere til sin sjef hvis han/hun oppdager feil eller mangler.
                \item  Ved alarm skal den generelle branninstruksen følges (Redde, varsle,
                    slukke, begrense, veilede).
            \end{enumerate}
    \end{instruksledd}


    \begin{center}\textbf{Nærmere kommentarer til enkelte av brannverntiltakene} \end{center}
    \begin{description}
        \item \textbf{Rømningsveier} Alle Samfundets lokaler er ved hjelp av brannvegger, branndører og andre tekniske tiltak
                forsøkt delt inn i brannceller, slik at skaden i størst mulig grad skal begrenses ved en eventuell brann.
                Rømningsveier skal alltid være egne brannceller, og det betyr at brann og røykgasser ikke skal kunne trenge inn i
                rømningsveien før etter en viss tid (normalt en time, men vesentlig kortere på Samfundet). Derfor er for eksempel dører til
                trapperom selvlukkende (enkelte står til daglig åpne, men lukkes når brannalarmen går) og brannklassifiserte.
                Dersom slike dører ødelegges, settes åpne eller sperres slik at de ikke lukker ordentlig, mister de sin funksjon.
                Branndører som ikke virker kan være svært farlig i en evakueringssituasjon, tiltak som gjør at disse mister sin funksjon er
                derfor brudd på brukerinstruksen. Fra hver branncelle i en bygning skal det være minst to veier ut. I rømningsveiene skal
                det ikke oppbevares brennbart materiale som papirlapper og lignende. Det er også viktig at de ikke sperres av møbler,
                esker etc. Ved brannalarm velges den nærmeste og lettest tilgjengelige utgangen, slik at man kan
                komme seg ut og i sikkerhet raskest mulig. Disse rømningsveiene er gjengitt i branninstruksen for den enkelte hybel.
                Hver enkelt har selv ansvaret for å kartlegge rømningsveiene fra området man befinner seg i.

            \item \textbf{Branntekniske hjelpemidler} Dette omfatter i hovedsak brannvegger, branndører, nødlys, brannalarmanlegg, røykventilasjon,
                automatiske slukkeanlegg og slukkeutstyr (brannslanger og brannslukkingsapparater).
                Sikringssjefen ved Samfundet tar seg av vedlikehold av de branntekniske hjelpemidlene. Hver enkelt må derfor gi beskjed
                dersom det oppdages noe som ikke er som det skal. Dette kan være: Markeringslys (grønne lysende skilt med ``løpende menn'') som har slokket.
                Etterlysende ledelys som er ødelagt. Rom der brannalarmen ikke hørtes / hørtes dårlig. Slukkeapparat som er fjernet / flyttet / brukt opp.
                Brannslangeskap og slukkeapparat som er blokkert. Dører til og i rømningsvei som ikke lukker eller åpner skikkelig.
        \end{description}

   
\end{instruks}

