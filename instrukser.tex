
\addcontentsline{toc}{chapter}{Generelle Instrukser}

\begin{instruks}{Brannvern ved Studentersamfundet i Trondhjem}{1. januar 2008}{3. april 2008}
    Ved Studentersamfundet i Trondheim er Finansstyrets leder, øverste brannansvarlig. Dette ansvaret er ved
    Studentersamfundet delegert til daglig leder og sikringssjefen. Sikringssjefen har sammen med Husmann ansvaret for
    de organisatoriske brannverntiltakene, og de har oppnevnt en rekke områdeansvarlige ved de ulike offentlige lokalene.
    Disse rapporterer til Husmann eller sikringssjefen ved avvik innenfor brannsikkerheten.
    Hvis brannalarmen utløses er det til en hver tid definert hvem som har kommando over situasjonen, og som de
    områdeansvarlige skal forholde seg til. (vedlegg A) 

    Som en del av sikkerhetsarbeidet ved Samfundet skal alle gjengsjefer og funksjonærer ha vært gjennom et
    kontrollørkurs med brannopplæring, holdt av Husmann. Videre skal gjengsjefer informere sine kollegaer om dette skrivet 
    i tillegg til annen nødvendig informasjon om det området de har ansvaret for. 

    Det er ikke for å frata Studentersamfundet noe ansvar ved eventuelle kritiske situasjoner, men for å være sikker på at
    alle gjengmedlemmer har forstått rutinene og ansvaret for egen sikkerhet.

    \begin{instruksledd}{Generell branninstruks for alle Gjengmedlemmer}

        Den generelle branninstruksen kan sammenfattes i følgende begreper:

        \begin{center}
            \textbf{REDDE -- VARSLE -- SLUKKE -- BEGRENSE -- VEILEDE}
        \end{center}

        I en brannsituasjon er faktorenes orden i denne branninstruksen likegyldig, dvs. at
        tiltakene ikke skal iverksettes i en
        bestemt rekkefølge. Det viktigste er at man holder hodet kaldt, unngår panikk, og
        handler ut fra den konkrete
        situasjonen der og da. Dvs. at man iverksetter de tiltak som man selv føler er
        riktige, og mest hensiktsmessige.

        \begin{description}
            \item \textbf{Redde} Forsøke å redde hjelpetrengende (f.eks bevegelseshemmede) som man
                eventuelt oppdager. Redningsaksjoner skal kun
                gjennomføres i den utstrekning de ikke medfører fare for eget eller andres liv
                og helse.
            \item \textbf{Varsle} Oppdages et branntilløp er det svært viktig at brannvesenet varsles
                umiddelbart (manuell brannmelder eller telefon
                110), forutsatt at man ikke straks får kontroll over branntilløpet.
                Vakthavende vil alltid være i kontakt med
                brannvesenet, men ring du også for sikkerhets skyld. Branntilløp som man selv
                får kontroll over og slukket, skal
                meldes til din lokaleansvarlig, som så varsler Husmann og/eller sikringssjef.
                Hvis brannalarmen går blir brannvesenevarslet automatisk. Det kan likevel være
                fornuftig å ringe for å bekrefte alarmen overfor brannvesenet dersom du vet
                at det er en reell brann.
            \item \textbf{Slukke} I den utstrekning det er mulig med de hjelpemidler man har for hånden skal man
                forsøke å slukke branntilløp som
                oppdages. Men igjen er det viktig å understreke at dette bare skal
                gjennomføres i den grad det ikke setter eget eller
                andres liv og helse i fare. Det er viktig å være oppmerksom på at en brann
                utvikler seg svært raskt, og at et forsøk på å
                slukke en brann må skje raskt. Lær deg derfor hvor slukkeutstyret ved din
                arbeidsplass er plassert og hvordan dette
                brukes.
            \item \textbf{Begrense} Ved branntilløp og brannalarm har alle plikt til å foreta
                forebyggende tiltak med tanke på å begrense skadevirkningene
                av brannen/branntilløpet. Slike tiltak kan være å forsikre seg om at dører og
                vinduer er lukket i de rommene man
                forlater (men ikke lås av hensyn til brannvesenet som kan behøve å komme inn),
                at gassflasker ol. som benyttes
                stenges igjen før de forlates, at elektrisk apparatur (f.eks PC) slås av og at
                man sjekker at branndører og andre tekniske
                innretninger med brannhemmende formål fungerer slik de skal når de passeres,
                eventuelt lukke disse manuelt dersom
                teknikken ikke virker. Dersom du er i tvil om hva dette gjelder er det bare å
                spørre Husmann eller sikringssjef.
            \item \textbf{Veilede} Samtidig som lokalitetene evakueres har man plikt til å veilede og
                informere andre som trenger det om hva som er i
                ferd med å skje, og hva den reelle situasjon er slik man selv har
                oppfattet det. Det er viktig at vakthavende og
                brannvesenet får beskjed dersom du vet noe om det som har hendt.
        \end{description}

        \textbf{Det er den enkeltes \emph{plikt} å forlate bygget når brannalarmen går.} Velg korteste vei
        ut av bygget. Hold deg godt klar av inngangene slik at alle kan komme seg ut og brannvesenet kan komme seg inn.
        \textbf{Det er ikke tillatt å oppholde seg rett ved inngangspartiene til bygget.} Om mulig går alle til det området utenfor
        Samfundet som er avtalt for ens arbeidsplass eller hybel.

        \textbf{Det vil varsles over høytalerne om når alle kan gå inn igjen. Dette signaliserer at
        brannvesenet har klarert stedet.}


    \end{instruksledd}



    \begin{instruksledd}{Brukerinstruks for alle gjengmedlemmer}


        Den generelle branninstruksen gjelder når en brann oppdages og/eller når
        brannalarmen går. Brukerinstruksen er en
        forebyggende instruks med formål å ivareta bygningsmassens branntekniske standard,
        slik at denne til enhver tid er
        optimal etter forutsetningene. Brukerinstruksen er i samsvar med Forskrift om
        brannforebyggende tiltak og brannsyn,
        og gjengis delvis her:


        \begin{center}
            \textbf{En bruker av et bygg har ansvar for}
        \end{center}


        \begin{enumerate}
            \item Ikke ødelegge brannverntiltak som er iverksatt. Dette gjelder rømningsveier, nødlys, brannalarmanlegg,
                slukkeutstyr, brannvegger, branndører mv.
            \item Rapportere til sin sjef hvis han/hun oppdager feil eller mangler.
            \item  Ved alarm skal den generelle branninstruksen følges (Redde, varsle,
                slukke, begrense, veilede).
        \end{enumerate}


        \begin{center}
            \textbf{Nærmere kommentarer til enkelte av brannverntiltakene} 
        \end{center}
        \begin{description}
            \item \textbf{Rømningsveier} Alle Samfundets lokaler er ved hjelp av brannvegger, branndører og andre tekniske tiltak
                forsøkt delt inn i brannceller, slik at skaden i størst mulig grad skal begrenses ved en eventuell brann.
                Rømningsveier skal alltid være egne brannceller, og det betyr at brann og røykgasser ikke skal kunne trenge inn i
                rømningsveien før etter en viss tid (normalt en time, men vesentlig kortere på Samfundet). Derfor er for eksempel dører til
                trapperom selvlukkende (enkelte står til daglig åpne, men lukkes når brannalarmen går) og brannklassifiserte.
                Dersom slike dører ødelegges, settes åpne eller sperres slik at de ikke lukker ordentlig, mister de sin funksjon.
                Branndører som ikke virker kan være svært farlig i en evakueringssituasjon, tiltak som gjør at disse mister sin funksjon er
                derfor brudd på brukerinstruksen. Fra hver branncelle i en bygning skal det være minst to veier ut. I rømningsveiene skal
                det ikke oppbevares brennbart materiale som papirlapper og lignende. Det er også viktig at de ikke sperres av møbler,
                esker etc. Ved brannalarm velges den nærmeste og lettest tilgjengelige utgangen, slik at man kan
                komme seg ut og i sikkerhet raskest mulig. Disse rømningsveiene er gjengitt i branninstruksen for den enkelte hybel.
                Hver enkelt har selv ansvaret for å kartlegge rømningsveiene fra området man befinner seg i.

            \item \textbf{Branntekniske hjelpemidler} Dette omfatter i hovedsak brannvegger, branndører, nødlys, 
                brannalarmanlegg, røykventilasjon,
                automatiske slukkeanlegg og slukkeutstyr (brannslanger og brannslukkingsapparater).
                Sikringssjefen ved Samfundet tar seg av vedlikehold av de branntekniske hjelpemidlene. Hver enkelt må derfor gi beskjed
                dersom det oppdages noe som ikke er som det skal. Dette kan være: Markeringslys 
                (grønne lysende skilt med ``løpende menn'') som har slokket.
                Etterlysende ledelys som er ødelagt. Rom der brannalarmen ikke hørtes / hørtes dårlig. Slukkeapparat 
                som er fjernet / flyttet / brukt opp.
                Brannslangeskap og slukkeapparat som er blokkert. Dører til og i rømningsvei som ikke lukker eller åpner skikkelig.
        \end{description}

    \end{instruksledd}

\end{instruks}


\begin{instruks}{Vedlegg A: Oversikt over ansvarsfordeling på brannsentralen}{}{}
    
    
    \textbf{Vanlige ukedager (mandag til fredag i semesteret)}

    \begin{tabular}{ll}
        00.00 til 09.00 &    Alle med opplæring \\
        09.00 til 15.00 &    Faste ansatte\\
        15.00 til 24.00 &    Kaf\'eansvarlige 
    \end{tabular}

    
    \textbf{Fredag (i semesteret)}
    
    \begin{tabular}{ll}
        00.00 til 09.00 &Alle med opplæring \\
        09.00 til 15.00 &Faste ansatte\\
        15.00 til 18.00 &Kaf\'eansvarlige\\
        18.00 til 24.00 &Klubbstyret
    \end{tabular}

    
    \textbf{Lørdag (i semesteret)}
    
    
    \begin{tabular}{ll}
        00.00 til 09.00 &Alle med opplæring \\
        09.00 til 15.00 &Faste ansatte\\
        15.00 til 18.00 &Kaf\'eansvarlige\\
        18.00 til 24.00 & Lørdagskomiteen
    \end{tabular}

    
    \textbf{Søndag (i semesteret)}
    
    
    \begin{tabular}{ll}
        00.00 til 04.00 &Lørdagskomiteen\\
        04.00 til 15.00 & Alle med opplæring\\
        15.00 til 24.00 & Kaf\'eansvarlige
    \end{tabular}

    
    \textbf{Ikke KLST/LK arrangement på hverdager (i semesteret).}
    
    
    \begin{tabular}{ll}
        00.00 til stengetid &    Vaktene\\
        Stengetid til 09.00 &    Alle med opplæring\\
        09.00 til 15.00 &     Faste ansatte \\
        15.00 til 24.00 &    Kaf\'eansvarlige 
    \end{tabular}

    
    \textbf{KLST/LK arrangementer på hverdager (i semesteret).}
    
    \begin{tabular}{ll}
        00.00 til Stengetid &   KLST eller LK \\
        Stengetid til 09.00 &   Alle med opplæring \\
        09.00 til 15.00 &     Faste ansatte \\
        15.00 til 24.00 &      Kaf\'eansvarlige 
    \end{tabular}

    
    \textbf{Arrangement utenfor semester.}
    
    \begin{tabular}{ll}
        Åpningstid til stengetid &     Vaktene \\
        Stengetid til åpningstid &     Alle med opplæring
    \end{tabular}
    
    
    \vspace{10mm}
    Dette er hvordan ansvaret for betjeningen av brannsentralen er fordelt. Dette for at det i
    åpningstiden alltid skal være
    noen som er ansvarlig. Selv om det er faste ansvarlige skal alltid folk med kompetanse på
    sentralen møte opp ved alarm.
\end{instruks}



\begin{instruks}{Instruks for Funksjonærer}{1 .januar 2008}{3. April 2008}

    \begin{enumerate}
        \item Funksjonærstatus kan tildeles av Finansstyret og av grupper som Finansstyret
            har delegert myndighet
            til. Funksjonærsøknader sendes til Gjengsekretariatet.
        \item Personer som har vært aktive funskjonærer i en gjeng i to år kan tildeles
            funksjonærpangsjoniststatus.
            Betingelsene er at vedkommende er en sosial og/eller faglig ressurs for
            gjengen, er student (evt. avtjener
            militær eller sivil verneplikt) og bor i Trondheim. Gjengen sender søknad for
            alle aktuelle kandidater til
            Gjengsekretariatet.
        \item Personer som har vært aktive funksjonærer i en gjeng i ett år kan i spesielle
            tilfeller bli tildelt
            funksjonærpangsjoniststatus etter særskilt søknad til Gjengsekretariatet.
        \item Funksjonærer har gratis adgang til Huset med følge på arrangementer i regi av
            Studentersamfundets
            gjenger, men arrangør kan i visse tilfeller reservere seg helt eller delvis
            mot ordningen. Hvis arrangør
            reserverer seg mot ordningen, skal dette informeres om på forhånd, helst en
            uke i forveien. Det åpnes
            for at det kan legges ut et begrenset antall billetter på Kontrollkontoret til
            Husets funksjonærer i slike
            tilfeller.
    \end{enumerate}

\end{instruks}


\begin{instruks}{Husorden}{29. oktober 2009}{3. desember 2009}
    \begin{enumerate}
        \item Husorden gjelder for alle gjengmedlemmer og funksjonærer ved
            Studentersamfundet i Trondhjem.
            Gjenger og enkeltpersoner som har gjester plikter å påse at disse også
            overholder Husorden. Styret regnes
            som en gjeng når det gjelder husorden.
        \item Husorden gjelder hele året uavhengig av semestrene. I semestrene har Huset
            åpningstider fastsatt av
            Finansstyret. Vaktmester er ansvarlig for åpning og stenging av Huset. Ved
            stengetid skal alle aktiviteter i
            offentlige lokaler opphøre. Unntak herfra er arrangementer hvor annen
            stengetid er avtalt med Daglig
            leder. Vaktmester skal sørge for utlåsing ved slike arrangementer. Alle som
            ikke er gjengmedlemmer eller
            innbudte gjester på gjengenes hybler, skal være ute av Huset ved stengetid.
        \item Styret og Lørdagskomiteen deler disposisjonsrett til offentlige lokaler på
            lørdager. Klubbstyret disponerer
            de offentlige lokaler på fredager. Forøvrig disponerer Daglig leder gjennom
            rombookingsinstruksen
            Husets offentlige lokaler. Gjengene disponerer arealer fastsatt av
            Finansstyret. Utenom semestrene
            disponerer Daglig leder de offentlige lokalene. Husets gjenger får, når det er
            behov for det, reservert
            lokaler vederlagsfritt. Man må imidlertid være oppmerksom på at Daglig leder
            kan prioritere økonomisk
            gunstig utleie av de offentlige lokalene. I slike tilfeller skal Daglig leder
            uten opphold underrette den
            berørte gjeng. Se også Rombookingsinstruks for prioritering av bruk av
            offentlige lokaler.
        \item Kontrollkontoret er ansvarlig for renhold av offentlige arealer, herunder også
            scene og garderober ved
            arrangement der disse er i bruk.
        \item Funksjonærer med ett følge har gratis adgang til alle arrangementer i regi av
            Studentersamfundet. Se for
            øvrig ”Instuks for funksjonærer”.
        \item Husorden gjelder ikke under Studentersamfundets UKEr. UKEstyret lager sine
            egne regler/instrukser.
            Disse skal godkjennes av Finansstyret.
        \item Gjenger som ønsker å gjøre forandringer i egne lokaler plikter å ta kontakt
            Gjengsekretariatet. Se forøvrig
            ”Instruks for arealdisponering på Samfundet”.
        \item Det skal ikke forekomme unødig ferdsel på taket.
        \item Funksjonærer i KLST, Regi, DG, FG, KSG, MG, KU, LK, VK, ITK, og FK er
            kontrollører overfor Studentersamfundets medlemmer og gjester. Det vises til egen
            kontrollørinstruks.
        \item For bruk av hybler vises til egen instruks.
        \item Sikringssjef leverer ut nøkler etter fordeling fastsatt av Finansstyret. Bruk
            av nøkler skal skje i henhold til
            nøkkelinstruks.
        \item Gjengene ved Studentersamfundet er regnskapspliktige.
        \item Gjengene plikter å følge de påbud Sikringssjefen fastsetter innen brann og
            sikringsarbeid. Se for øvrig
            ”Instruks for Sikringskomiteen”.
        \item Endringer av Husorden skal skje i samsvar med Studentersamfundets lover og i
            samråd med de berørte
            parter. Alle endringer skal tas opp i GSK til uttalelse og vedtak fattes i FS.
            Dersom det oppstår tvister
            angående tolking av Husorden, skal Rådet avgjøre tvisten.
        \item Husmann plikter å formidle Husorden og alle gjeldende instrukser til alle
            gjenger og påse at disse
            overholdes. Den enkelte gjengsjef er ansvarlig for at alle gjengmedlemmer
            kjenner og kontrollerer
            Husorden. Innunder Husorden kommer Kontrollørinstruks med Hybelinstruks,
            Nøkkelinstruks, Instruks
            om brannvern, Gjenginstruks og Lenkevaktinstruks.
    \end{enumerate}
\end{instruks}


\begin{instruks}{Hybelinstruks}{29. oktober 2009}{3. desember 2009}
    \begin{instruksledd}{Generelle regler}
        \begin{enumerate}
            \item Hyblene er et privat sted hvor ro og orden er gjengens egen sak. Det
                forutsettes her at en ikke er til
                sjenanse for andre. Hvis så er tilfelle, har Husmann eller gjengsjef i LK
                rett til å tømme hybelen.
            \item Etter kl 21 på dager der store deler av Huset er åpent, er det kun
                gjengmedlemmer med ett følge som har
                adgang til hyblene. Det enkelte gjengmedlem er i alle tilfeller ansvarlig
                for at følget det har med seg kan
                og overholder hybelinstruksen. På fredager og lørdager må følget stemples
                av lenkevaktene før midnatt for
                å slippe inn på sideloftene. Arrangør kan gi innslipp til sideloftene i
                spesielle tilfeller.
            \item Gjengmedlemmer og deres følger (med stempel) må være inne på sideloftene
                før kl 02:00.
            \item Bakdøra stenger kl 02:00, og innlåsing skal ikke 
forekomme etter dette
                tidspunktet.
            \item Ved utlåsing utenom Husets åpningstid er det enkelte gjengmedlem ansvarlig
                for at ingen uvedkommende slipper inn.
            \item Det er ikke lov å ta med alkohol ut fra hyblene. Alkohol må ikke under
                noen omstendighet tas med ut av Huset.
            \item Det er ikke lov å omsette med penger på hyblene.
            \item Det er ikke tillatt å sove på hyblene.
            \item Restavfall og papp kastes i dertil egnede containere i Kronprinsesse
                Märthas all\'e, papir kastes i egne
                beholdere. Det skal ikke lagres materialer, skrot eller annet løsøre i
                korridorene utenfor hyblene.
        \end{enumerate}
    \end{instruksledd}
\end{instruks}


\begin{instruks}{Instruks for nøkler og nøkkelkort}{1. januar 2008}{3. april 2008}
    \begin{instruksledd}{Ansvar}
        \begin{enumerate}
            \item Finansstyret har det overordnede ansvar for adgangskontroll på Huset.
                Systemer for adgangskontroll omfatter
                nøkkelsystem og kortlås-systemet.
            \item Finansstyret eller den det gir fullmakt avgjør hvilke nøkler og
                adgangsrettigheter brukergruppene skal ha.
        \end{enumerate}
    \end{instruksledd}

    \begin{instruksledd}{Forvaltning av nøkler}
        \begin{enumerate}
            \item Sikringssjef forvalter alle nøkler og nøkkelsylindere på Huset. Det skal til enhver
                tid føres ajourført
                nøkkeloversikt. Sikringssjef kvitterer ut nøkler til gjengsjef eller tilsvarende.
            \item Gjengsjef eller tilsvarende er ansvarlig for at gjengens nøkler blir behandlet
                forsvarlig. Vedkommende plikter
                derfor å føre en detaljert nøkkeloversikt over gjengens nøkler. Her skal det til
                enhver tid fremgå hvem som
                besitter hvilken nøkkel. Dette føres på egne standardiserte skjema. Sammendrag leveres
                på oppfordring til
                Sikringssjef.
            \item Alle brukergrupper plikter å ha et nøkkelskap med systemsylinder.
            \item  Nøkler det ikke lenger er behov for returneres Sikringssjef for oppbevaring i
                nøkkelskap.
        \end{enumerate}
    \end{instruksledd}

    \begin{instruksledd}{Nøkkelkortsystemet}
        \begin{enumerate}
            \item Aktivering av nøkkelkort skjer via medlemsdatabasen.
            \item Aktivering forutsetter betalt medlemskap av Studentersamfundet.
            \item Gjengsjefer må registrere gjengmedlemskap for at nøkkelkort skal gis riktige
                adgangsrettigheter.
            \item Ved arrangement kontaktes om nødvendig Kontrollkontoret for utkvittering av
                nøkkelkort eller for å få låst
                opp lokaler.
        \end{enumerate}
    \end{instruksledd}

    \begin{instruksledd}{Regler for bruk av nøkkelsystem og kortlås-systemet}
        \begin{enumerate}
            \item Jobbehanker skal aldri ut av Huset.
            \item Nøkler og nøkkelkort skal behandles som verdigjenstander.
            \item Nøkler og nøkkelkort skal ikke lånes ut.
            \item Ødelagte nøkler leveres Sikringssjef.
            \item Tap av nøkler skal umiddelbart rapporteres til Sikringssjef og medfører normalt
                nøkkelkarantene.
            \item Tap av nøkkelkort skal umiddelbart rapporteres til Kontrollkontoret. Nytt
                nøkkelkort utstedes mot et gebyr.
            \item Nøkler og nøkkelkort som blir funnet skal leveres Kontrollkontoret.
            \item Man har ingen rett til å låse seg inn et sted selv om man besitter
                nøkkel eller har nøkkelkort med
                tilgangsrettighet. Skjer dette i utleide lokaler, på eksterne
                arrangementer eller arrangementer hvor det skal
                betales entr\'e , er det å betrakte som brudd på denne instruks. Har man
                behov for å låse seg inn på gjenghybler,
                skal dette såfremt mulig avtales i hvert enkelt tilfelle.
            \item Låste dører skal \emph{alltid} låses igjen. Vær oppmerksom på hvem du slipper inn
                en låst dør, spør gjerne etter id-kort.
        \end{enumerate}
    \end{instruksledd}

    \begin{instruksledd}{Nøkkelmisbruk}
        \begin{enumerate}
            \item Nøkkelmisbruk omfatter både nøkkelsystem og kortlåssystemet og
                innebefatter blant annet å:
                \begin{enumerate}
                    \item  låse seg inn på områder hvor det foregår uteleiearrangement
                        eller arrangement hvor det tas betalt for
                        adgang, utenfor jobbøyemed.
                    \item låse seg inn på offentlige arealer utenfor husets
                        åpningstider, med mindre det er i jobbøyemed eller er
                        klarert med ansvarlig.
                \end{enumerate}
        \end{enumerate}
    \end{instruksledd}

    \begin{instruksledd}{Formidling}
        \begin{enumerate}
            \item Gjengsjef plikter å gjøre seg selv og alle gjengmedlemmer kjent med innholdet i
                denne instruks.
        \end{enumerate}
    \end{instruksledd}

    \begin{instruksledd}{Brudd på instruks}
        \begin{enumerate}
            \item Brudd på denne instruks eller annen nøkkelmisbruk er alvorlig og kan medføre
                inndragelse av nøkler, nøkkelkort
                og/eller medlemskort eller bli behandlet i samsvar med § 34 av Lover og Statutter
                for Studentersamfundet i
                Trondhjem.
        \end{enumerate}
    \end{instruksledd}

\end{instruks}


\begin{instruks}{Instruks for kontrollører}{13. desember 2010}{24. februar 2011}

    \begin{enumerate}
        \item Samfundets kontrollører er alle funksjonærer i Diversegjengen,
            Forsterkerkomiteen, Fotogjengen, Markedsføringsgjengen, ITK, 
Klubbstyret, Kulturutvalget, Lørdagskomiteen,
            Videokomiteen,
            Kaf\'e- og serveringsgjengen og Regi. Kontrollørkortet er 
strengt knyttet til medlemsskap
            i en av de ovenfor nevnte gjenger,
            og kun til disse. Husmann fungerer som formann for kontrollørene.
        \item Kontrollørene har ansvaret for ro og orden i Huset. De plikter derfor å
            reagere på uakseptabel opptreden i
            Huset såfremt de ikke selv er overstadig beruset.
        \item Uakseptabel opptreden definers bl.a. som:
            \begin{enumerate}
                \item  Glassknusing
                \item  Slossing
                \item  All omgang med narkotika, sprit og annet medbragt
                \item  Generelt bråk i Storsalen
                \item  Røyking
                \item  Generelt bråk
            \end{enumerate}
        \item Grip aldri inn med voldsbruk. Vis alltid kontrollørkortet ditt først og virk
            rolig og behersket. Ser du andre
            kontrollører i vanskeligheter skal du alltid stille deg opp ved siden av og
            støtte vedkommende.
        \item Du kan kaste ut en person ut av Huset dersom du mener det er nødvendig. Det
            er imidlertid kun Klubbstyret,
            Lørdagskomiteen og de ansatte vaktene som har lov til å inndra medlemskort.
        \item Kontrollørene plikter å stille til vakttjeneste på 48 timers varsel, men de
            skal i første rekke bli innkalt
            minimum en uke på forhånd. Det er absolutt alkoholforbud på vakt.
        \item  Kontrollører skal tilstrebe å alltid ha kontrollørkort med seg når de
            ferdes på Huset. Ved misbruk av
            kontrollørkort og brudd på andre tillitsforhold, kan kontrollørkortet inndras
            av Husmann i samråd med
            gjengsjefene i de gjenger som har kontrollører. Saken oversendes Finansstyret
            for avgjørelse.
        \item Gjengsjefer plikter å informere nye kontrollører om innholdet i denne
            instruks.
        \item Husmann plikter å avholde kontrollørkurs minst en gang i semesteret.
    \end{enumerate}

\end{instruks}


\begin{instruks}{Instruks for lenkevakter}{16. desember 2010}{24. februar 2011}

    \begin{enumerate}
        \item Vaktene fordeles til gjengene av Husmann. Husmann er formann for
            lenkevaktene. Gjengsjefen er ansvarlig
            for å fordele vaktene internt i gjengen og at disse møter til avtalt tid.
        \item Lenkevaktene skal innkalles minimum en uke før arrangementet. Arrangør
            (Husmann eller gjengsjef i
            Lørdagskomit\'een) skal senest to dager før arrangementet få tilbakemelding med
            navngitte personer som
            stiller som lenkevakt. Dersom en gjeng ikke kan stille på vakt kan den bytte
            innbyrdes med en annen gjeng.
            Arrangør skal holdes informert om byttene.
        \item På fredager møter lenkevaktene normalt på Biblioteket kl 2030. På lørdager
            møter lenkevaktene normalt på
            LKs hybel kl 2045. Dette gjelder så lenge ingen annen beskjed er gitt.
        \item Det er totalt alkoholforbud for alle lenkevakter så lenge de er på vakt.
        \item Lenkevaktene skal ha med seg id-kort og evt. kontrollørkort på vakt.
        \item Personer som står bak barrikader (scenevakt) skal stå med ryggen til scenen
            slik at publikum alltid er under oppsikt.
        \item Ved eventuell brann skal lenkevaktene hjelpe til med rømming av Huset.
            Lenkevaktene må derfor sette seg nøye inn i Rømmingsinstruksen.
        \item Alle Husets gjenger kan i prinsippet bli innkalt som lenkevakter. Men
            Klubbstyret, Lørdagskomit\'een, Kulturutvalget, Videokomit\'een, Regi,
            Forsterkerkomit\'een, Styret og Markedsføringsgjengen har hovedsakelig fritak. 
            Styrene i ISFiT og UKA har fritak første semesteret de sitter.
    \end{enumerate}
\end{instruks}




