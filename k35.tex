\begin{instruks}{Instruks for Klostergata 35}{Februar 2008}{Februar 2008}

    \begin{instruksledd}{Formål}
        Denne instruksen inneholder retningslinjer for beboere, Klostergatas vaktmester,
        bestyrer og andre ansvarlige
        i Klostergata 35, samt daglig leder ved Studentersamfundet i Trondhjem
        (Samfundet). Instruksen forholder seg til
        husleiekontrakten og står ikke over Norges lover.
    \end{instruksledd}
    
    \begin{instruksledd}{Formidling}
        Bestyrer og gjengsjef plikter å holde aktuelle og potensielle beboere orientert om
        innholdet i denne
        instruksen.
    \end{instruksledd}

    \begin{instruksledd}{Klostergata 35}
        Klostergata 35 er en bygård som eies av Studentersamfundet i Trondhjem. Gården
        skal gi et godt botilbud til
        Samfundets ansatte, sivilarbeidere og frivillige.
        
        Gården består av åtte leiligheter som er oppdelt i hybler. Leilighetene har navn
        etter etasje og side (sett forfra):
        \begin{description}
            \item 1. venstre: 3 Hybler
            \item 1. høyre: 3 Hybler
            \item 2. venstre: 3 Hybler
            \item 2. høyre: 5 Hybler
            \item 3. venstre: 3 Hybler
            \item 3. høyre: 5 Hybler
            \item 4. venstre: 3 Hybler
            \item 4. høyre: 5 Hybler
        \end{description}
        Leiepriser fastsettes av Finansstyret.
    \end{instruksledd}

    \begin{instruksledd}{Borett}
        Personer som har borett i Klostergata 35, etter prioriteringsrekkefølge:
        \begin{enumerate}
            \item Ansatte
            \item Sivilarbeidere
            \item Funksjonærer i stilling med minimum to års bindingstid i en fast gjeng på Samfundet.
            \item Funksjonærer i stilling med minimum ett års bindingstid i en fast gjeng på Samfundet.
            \item Gjengmedlemmer med minimum ett års bindingstid i en fast gjeng på Samfundet.
            \item Andre aktive ved og medlemmer av Studentersamfundet i Trondhjem.
        \end{enumerate}
        
        Beboere som omfattes av punkt 1 og 2 har kun borett så lenge deres
        ansettelsesforhold varer. Når forholdet opphører
        bortfaller boretten med øyeblikkelig virkning.
        
        Beboere som omfattes av punktene 3, 4, 5 må være enten aktive i en gjeng/gruppe på
        Samfundet eller aktiv
        pangsjonist i en gjeng på Samfundet. I tillegg må de være medlemmer av Samfundet
        og studere ved en
        utdanningsinstitusjon med medlemsrett i Samfundet. Borett bortfaller umiddelbart
        dersom gyldig studiebevis ikke kan
        fremvises eller funksjonærstatusen/medlemskapet opphører. Til slutt og med lavest
        prioritet åpnes det i punkt 6 for
        borett for vanlige medlemmer av Samfundet.
    \end{instruksledd}

    \begin{instruksledd}{Prioritering av beboere}
        \begin{enumerate}
            \item Fast ansatte ved Samfundet bor vanligvis ikke i Klostergata 35. Ved spesielle
                omstendigheter kan de
                søke til daglig leder om å få leie neste hybel som blir ledig.
            \item Sivilarbeidere får som regel tilbud om hybel i Klostergata 35 fra første til
                siste arbeidsdag.
            \item  Funksjonærer i stillinger med minimum to års bindingstid får tilbud om å bli
                satt opp på en venteliste og
                er garantert å få tilbudt hybel når det blir dennes tur så fort det blir ledig.
            \item Funksjonærer i stillinger med minimum ett års bindingstid kan bli satt opp på
                hybelventelista dersom for
                få fra punkt 3 ønsker å stå på den.
            \item Gjengmedlemmer kan bli satt opp på hybelventelista dersom for få fra punkt 3 og 4
                ønsker å stå på den.
            \item Andre medlemmer av Studentersamfundet i Trondhjem kan bli satt opp på
                hybelventelista dersom for få fra punkt 3, 4 og 5 ønsker å stå på den.
        \end{enumerate}

        Potensielle leietakere med lavere prioritet kan bli tildelt rom dersom noen
        fortsatt står ledige etter tildeling til
        leietakere med øverst prioritet.
    \end{instruksledd}

    \begin{instruksledd}{Ventelista}
        Gjengsjef for hver enkelt gjeng er ansvarlig for å gi bestyrer beskjed om hvem som
        har rett til og ønske om å stå på
        lista over fremtidige beboere i Klostergata 35. Gjengsjef skal gi denne
        informasjonen til bestyrer rett etter hvert
        opptak. En gang per semester vil bestyrer utføre loddtrekning mellom
        funksjonærene. Det er bare mulighet for å bli
        ført opp på prioriteringslista i sitt første semester som funksjonær, og det er
        bare mulighet til å bli ført opp n gang.
        
        Gjengsjef er pliktig å melde fra om personer som har mistet sin borett. Slike
        opplysninger skal gis fortløpende til
        daglig leder av Samfundet og bestyrer, som igjen vil ta personen av
        prioriteringslista. Det åpnes for at potensielle
        beboere i Klostergata 35 kan være midlertidig bortreist fra Trondhjem, dog kun i
        studieøyemed, uten å bli fjernet fra
        lista, dersom de når de kommer tilbake til byen, fortsatt oppfyller alle kravene
        til boretten omtalt ovenfor.
        
        Basert på informasjonen fra gjengsjef og daglig leder, skal bestyrer vedlikeholde
        ventelista over funksjonærer med
        potensiell fremtidig borett i Klostergata 35. Bestyrer skal holde oversikt over
        hvem som har høyest prioritert borett til
        enhver tid. De som står høyere enn plass nummer ti på ventelista har krav på å få
        oppdateringer av minst n gang per
        semester dersom ikke oppdatert liste ligger tilgjengelig på internett. Den som
        står høyest opp på ventelista skal få
        leietilbud minst en måned før innflyttingsdato.
        
        Nærmere beskrivelse av venteliste, inn- og utflytting kommer under punkt 4.1 om
        bestyrer.
    \end{instruksledd}

    \begin{instruksledd}{Botid}
        Det inngås leiekontrakt for ett år ved innflytting. Denne kan fornyes to ganger
        utover dette, så lenge betingelsene for
        borett er oppfylt. Etter dette vil ikke kontrakten bli fornyet ytterligere.
        Maksimal botid er dermed tre år. Beboer plikter
        å gi daglig leder ved Samfundet og bestyrer skriftlig oppsigelse minst to måneder
        før utflytting.
    \end{instruksledd}

    \begin{instruksledd}{Regler for beboere}
        Alle som bor i Klostergata 35 forplikter seg til å kjenne og etterfølge reglene i
        denne instruksen.
    \end{instruksledd}

    \begin{instruksledd}{Felles epostliste}
        Alle beboere må gå til https://lists.samfundet.no/mailman/listinfo/klostergt og
        melde seg på epostlisten
        klostergt@samfundet.no. All informasjon som vedrører Klostergata 35 sendes til
        denne listen.
    \end{instruksledd}

    \begin{instruksledd}{Vedlikehold internt i leilighetene}
        All oppussing/renovering skal skje i samråd med bygårdens vaktmester. Det er mulig
        å søke om tilskudd ved å sende
        budsjett via vaktmester til daglig leder.
    \end{instruksledd}

    \begin{instruksledd}{Vedlikehold av fellesarealer}
        Alle beboere skal gjøre seg kjent med og følge vaskelistene, som settes opp av
        vaktmester. Beboerne bør også delta
        når vaktmester innkaller til dugnad på fellesarealer (trappegang på for- og
        bakside, loft- og kjellerrom).

        Vaktmester har oppsyn med og vurderer hvor vedlikehold er nødvendig. Han melder
        behov til daglig leder. Daglig
        leder og vaktmester tar i samråd beslutning om nødvendige tiltak. Samfundets
        Byggekomit (SBK) har ansvar for
        gjennomføring av større vedlikeholdsarbeider i, på og rundt bygningen.
    \end{instruksledd}

    \begin{instruksledd}{Oppbevaring}
        Fellesarealene skal ikke brukes til lagring. Begge trappegangene fungerer som
        nødutganger og skal til enhver tid være
        ryddige.

        Hver leilighet disponerer n bod i kjelleren og n på loftet til lagringsplass.
        Døren til disse skal merkes når de er i
        bruk. Hvert kolli som oppbevares her skal merkes med navn, leilighetsnummer, dato
        og telefonnummer.
        Bestyrer/vaktmester/andre beboere er i sin fulle rett til å kaste umerket gods.
    \end{instruksledd}

    \begin{instruksledd}{Vaskekjeller}
        I vaskekjelleren finnes det fire vaskemaskiner og to/tre tørketromler. Alle
        beboere er ansvarlige for at apparatene blir
        omsorgsfullt brukt.
    \end{instruksledd}

    \begin{instruksledd}{Parkeringsplassen}
        Klostergata 35 disponerer en parkeringsplass i bakgården. Denne fordeles blant
        interesserte ved loddtrekning. Beboere
        som kan dokumentere spesielle forhold som tilsier et særlig behov for
        parkeringsplass har fortrinnsrett. ”Særlig
        behov” defineres om nødvendig av Finansstyret.
    \end{instruksledd}

    \begin{instruksledd}{Bakgården}
        Arealet til høyre for innkjørselen samt området inntil husveggen tilhører
        Klostergata 35. På disse områdene befinner
        det seg en platting og en stor grill. Vaktmester har ansvar for at forefallende
        vedlikehold (gressklipping, oljing av
        platting, snømåking osv) blir gjort. Eksternt utlån av arealer avtales med
        vaktmester.

        Grillen tilhører Klostergata 35, lån foretas etter avtale med vaktmester.
    \end{instruksledd}

    \begin{instruksledd}{Nøkler}
        Alle skal ha to nøkler hver til sitt rom/leilighet. Disse går til postkasse,
        leilighet og den enkeltes hybel. Nøklene skal
        leveres inn ved utflytting.
    \end{instruksledd}

    \begin{instruksledd}{Ansvarsområder for daglig leder, bestyrer og vaktmester}
        Bestyrer og vaktmester bør velges blant beboere som har bodd minimum ett år i
        Klostergata 35. Arvtaker velges av
        avtroppende bestyrer/vaktmester ved loddtrekning blant interesserte. Vervet
        belønnes i form av redusert husleie.
        Reduksjonenes størrelse fastsettes av daglig leder.
    \end{instruksledd}

    \begin{instruksledd}{Bestyrer}
        Bestyrer har ansvar for å vedlikeholde ventelista for potensielle beboere. Dette
        innebærer:
        
        \begin{enumerate}
            \item Å føre opp nye potensielle beboere. Rekkefølgen bestemmes ved loddtrekning.
            \item Å holde medlemmene på ventelista oppdatert om deres status på ventelista
            \item Å gi tilbud om hybel til neste på lista ved oppsigelse (minst en måned før
                ut-/innflytting)
        \end{enumerate}

        I god tid før en hybel blir ledig bør bestyrer varsle beboerne muligheten for
        intern overflytting til det ledige rommet.
        Bestyrer skal sørge for at den nyinnflyttede får nøkler, kontrakt og kjennskap til
        denne instruksen.

        Bestyrer har også ansvar for å oppdatere beboerliste og holde seg orientert om at
        alle beboere har gyldige kontrakter
        og følger denne instruksen. Dersom det er mistanke eller åpenbart at beboere ikke
        oppfyller kravene i
        husleiekontrakten eller denne instruksen, skal bestyrer omgående gi beskjed til
        daglig leder.

    \end{instruksledd}

    \begin{instruksledd}{Vaktmester}
        Vaktmesters ansvarsoppgaver omfatter:
        \begin{enumerate}
            \item ppfølging av renhold og orden på fellesområder
            \item Sørge for mindre teknisk vedlikehold av gården (skifting av lys,
                utbedring av skader osv)
            \item Sørge for vedlikehold av hage og fellesområder
            \item Holde oversikt over områder som er pusset opp
            \item Informere SBK/daglig leder/vaktmester ved Samfundet om vedlikehold som
                faller inn under deres ansvar
                (som spesifisert under punkt 3.3)
            \item Koordinere bruk/utlån av fellesarealer som hage, loft og
                parkeringsplass
        \end{enumerate}
        Dersom det er mistanke eller åpenbart at beboere ikke oppfyller kravene i
        husleiekontrakten eller denne instruksen,
        skal vaktmester omgående gi beskjed til daglig leder.
    \end{instruksledd}

    \begin{instruksledd}{Daglig leder}
        Daglig leder ved Studentersamfundet i Trondhjem har ansvar for at utleierpliktene
        som står i kontrakten oppfylles.
        Under dette kommer for eksempel vann, varme, belysning, vedlikehold,
        prioriteringsliste/venteliste og
        kontraktfornyelse. Daglig leder bør derfor sørge for at det til enhver tid finnes
        fungerende bestyrer og fungerende
        vaktmester.

        Regulering av husleie og godtgjøring av bestyrer og vaktmester i Klostergata 35
        utføres av daglig leder.
        Det er også viktig at daglig leder har nær kontakt med vaktmester og Samfundets
        Byggekomit (SBK) for å holde
        oversikt over behov for fortløpende vedlikehold eller større
        restaureringsprosjekt. Økonomiske rammer for større
        vedlikeholdsoppgaver fastsettes av Finansstyret. SBK står for planlegging,
        gjennomføring og tilsyn av slike
        prosjekter.

        Daglig leder har øverste ansvar for at samtlige av beboerne i Klostergata 35
        oppfyller kravene i husleiekontrakten og
        følger retningslinjene i denne instruksen. Ved kontraktsbrudd skal daglig leder
        sørge for at leieforholdet bringes til
        ende umiddelbart.

    \end{instruksledd}
\end{instruks}



















