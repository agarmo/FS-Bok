\begin{instruks}{Avtale om bruk av Gløshaugen Idrettspark}{20. november 2011}{13. desember 2011}



Det er i dag inngått følgende avtale mellom UKA og Studentsamskipnaden i Trondheim. Avtalen regulerer samarbeid om etablering av Gløshaugen Idrettspark i Høgskoledalen, og UKA sin rett til å benytte denne ved UKA sine kulturarrangementer. Avtalens hovedformål er å sikre et best mulig tilbud innen idrett og kultur til Trondheims studenter, og støtte opp om frivillig arbeid rundt dette.


\begin{instruksledd}{Generelt} 
Det har blitt aktuelt å etablere et mindre og midlertidig idrettsanlegg i Høgskoledalen. Dette anlegget skal også legge til rette for fortsatt konsertvirksomhet i dalen under UKA.  Det oppgraderte anlegget går under navnet Gløshaugen Idrettspark.\\ 

UKA [985 163 693] (”UKA”) er en del av Studentersamfundet i Trondhjem [970 088 466]. Studentersamfundet i Trondhjem ved Finansstyret forplikter UKA utover UKEsjefens valgperiode.\\

Studentsamskipnaden i Trondheim [947 506 579] (”SiT”) har gjennom kommunen fått brukstillatelse til Høgskoledalen. Denne brukstillatelsen forutsetter at UKA skal få benytte Høgskoledalen vederlagsfritt for den perioden UKA varer, på samme måte som tidligere.\\  

SiT inngår bruksavtaler med aktuelle parter inngås i tråd med driftsfinansiering av Gløshaugen Idrettspark, og med likeverdige vilkår for alle studentorganisasjoner.\\

NTNUI (”NTNUI”) er studentidrettsforeningen til NTNU og har bruksavtale til anlegget utenfor UKA sin bruksperiode.\\

Avtalen gjelder fra avtalens inngåelse og så lenge Gløshaugen Idrettspark eksisterer.\\
\end{instruksledd}
 
\begin{instruksledd}{Avtalens formål}
Avtalen regulerer UKA sin bruksrett til Høgskoledalen og Gløshaugen Idrettspark i de semestrene det arrangeres UKE, og presiserer rettigheter og plikter knyttet til denne bruksretten.\\
\end{instruksledd}
 
\begin{instruksledd}{Uavhengig rådgiver}
Før UKA sin bruksperiode skal partene SiT, UKA og NTNUI i fellesskap oppnevne en uavhengig rådgiver. (”Rådgiver”)  Rådgiver skal bistår partene ved gjennomføring av avtalen, og påse at partene oppfyller sine plikter. Kostnadene til Rådgiver dekkes av SiT.\\  

Rådgiver gjennomgår anlegget før bruksovertakelse, påser at anlegget er forsvarlig sikret mot skade og slitasje, og at anlegget er ryddet og satt i stand ved tilbakelevering.\\
\end{instruksledd}
 
\begin{instruksledd}{SiTs rettigheter og forpliktelser}
SiT sine rettigheter til Høgskoledalen er begrenset av bruksavtalen med Trondheim kommune som grunneier. Bruksavtalen går foran avtaler mellom SiT og brukere av Høgskoledalen.\\

Etter bruksavtalen mellom SiT og Trondheim kommune har SiT rettighetene hva angår all bruk av Høgskoledalen. Bruksavtalen forutsetter at anlegget stilles til rådighet for UKA vederlagsfritt.\\

SiT forplikter seg til å reservere anlegget for UKEarrangementer. Det er UKA sitt ansvar å innhente de nødvendige tillatelser fra det offentlige til sin bruk av anlegget.\\

Om SiT må gjennomføre større vedlikeholds- og/eller opprustningsarbeider i Høgskoledalen, som tidsmessig vil overlappe med et UKEarrangement, må SiT varsle sittende UKEstyre minimum 18 måneder i forveien. Slike arbeider skal fortrinnsvis legges utenom perioder der UKA bruker dalen.\\

SiT stiller areal til rådighet for lagring av NTNUI sine installasjoner i bruksperioden, og for UKA sitt beskyttelsesutstyr utenfor bruksperioden.\\
\end{instruksledd}

\begin{instruksledd}{Bruksperiode for UKA}
Bruksperioden for UKA defineres av en oppriggingsperiode, en festivalperiode og en nedriggingsperiode. UKA har behov for 14 dager før festivalstart for opprigg av telt og tilsluttende infrastruktur. Festivalperioden vil være de programfestede arrangementsdatoene. Etter festivalen har UKA behov for ti dagers nedriggingsperiode.  Bruksperioden er begrenset til 49 dager, uavhengig av festivalens lengde.\\

For UKA-11 er disse periodene avtalt til:\\

Oppriggingsperiode: 	22. september – 5. oktober 2011\\
Festivalperiode: 	6. oktober - 30. oktober 2011\\
Nedriggingsperiode:	31. oktober – 9. november 2011\\

For påfølgende UKEr plikter UKEstyret seg å formidle de ulike tidsperioder til SiT og NTNUI så snart disse er bestemt, typisk våren i partallsår.
\end{instruksledd}

\begin{instruksledd}{Sikring mot skade}
UKA kjøper plater av kryssfiner som skal brukes for å dekke til de deler av kunstgressbanen som er innenfor teltet.  Øvrige deler av kunstgresset dekkes av fiberduk eller plater.  Leggemåte anvises av Rådgiver. Rådgiver skal inspisere tildekkingen når den er ferdig, og godkjenne dette som forsvarlig sikring av anlegget.\\  

UKA skal sikre de øvrige deler av anlegget ut over kunstgressmatten som kan bli skadet eller utsettes for stor slitasje. Omfanget av sikring anvises av Rådgiver. \\
 
Partene forventer uansett sikringsmåte, noe naturlig slitasje som følge av anleggstrafikk og konsertavvikling.\\
\end{instruksledd}
 
\begin{instruksledd}{Økonomi}
UKAs ansvar for reparasjon skader på kunstgressdekket er begrenset til kr. 50.000,- per bruksperiode.  Ansvarsbegrensningen forutsetter at UKA har lagt plater/fiberduk på kunstgressmatten, og på den måte som foreskrevet av Rådgiver.\\
\end{instruksledd}
  

\begin{instruksledd}{Klargjøring og overlevering}
SiT og NTNUI plikter å klargjøre anlegget for UKEarrangement før UKAs bruksperiode. Alle permanente installasjoner som mål, ballnett, basketballkurver og lignende tas ned og lagres utenfor Høgskoledalen.\\

Før UKAs overtagelse av anlegget for bruksperioden skal det gjennomføres en inspeksjon av Høgskoledalen med representanter fra SiT, NTNUI, UKA og Rådgiver. \\

Dersom inspeksjonen viser at anlegget ikke er klargjort som avtalt, bestemmer partene sammen hvilke tiltak som bør gjennomføres og en tidsplan for dette. Dersom det er uenighet om anlegget er klargjort som avtalt, avgjøres spørsmålet av Rådgiver. Kostnader til klargjøring bæres av NTNUI.\\

Inspeksjonen ved bruksovertakelse danner grunnlag for avviksrapport for tilsvarende inspeksjon ved tilbakelevering.  Ved inspeksjonene skal alle avvik dokumenteres fotografisk og skriftlig.\\
\end{instruksledd}

\begin{instruksledd}{Tilbakelevering}
Ved utløp av bruksperioden skal Høgskoledalen tilbakeleveres i samme stand som ved overtakelse.  Naturlig slitasje som følge av aktiviteten må påregnes.\\  

Ved tilbakelevering gjennomfører partene tilsvarende inspeksjon av området som ved overtakelse.  Partene vurderer her om området tilbakeleveres i forsvarlig stand.  Dersom det er nødvendig med tiltak, bestemmer partene sammen hvilke tiltak som bør gjennomføres og en tidsplan for dette.\\  

UKA er ansvarlig for det arbeid og de kostnader som måtte påløpe, med mindre annet blir avtalt.\\ 

Dersom det er uenighet mellom partene ved inspeksjonen eller om UKA har utført de tiltak som bestemt ved inspeksjonen, skal spørsmålet avgjøres av Rådgiver.\\
\end{instruksledd}

\end{instruks*}
