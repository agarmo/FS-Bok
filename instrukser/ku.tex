
\begin{instruks}{Instruks for Kulturutvalget}{ }{ }

    \begin{instruksledd}{Formål}
        \begin{enumerate}
            \item Kulturutvalget (KU) skal bidra til å gjøre Studentersamfundet i Trondhjem til et
                senter for kulturelle
                arrangementer og aktiviteter. KU skal gi et kulturelt tilbud rettet mot studenter, og da spesielt
                mot
                Samfundets medlemmer. KU er i samarbeid med Videokomiteen ansvarlig for eventuell kinodrift på
                Samfundet. KU skal arrangere Excenteraftener noen ganger i semesteret.
        \end{enumerate}
    \end{instruksledd}

    \begin{instruksledd}{Sammensetning}
        \begin{enumerate}
            \item KU består av inntil 9 funksjonærer. 
            \item En funksjonær i KU er aktiv i 2 år og har etter dette mulighet for å beholde
                funksjonærstatusen som
                pangsjonist.
            \item KU opprettholder ikke virksomheten under UKA.
        \end{enumerate}
    \end{instruksledd}

    \begin{instruksledd}{Ansvarsområde og plikter}
        \begin{enumerate}   
            \item  Medlemmer i KU plikter å kjenne innholdet av og overholde følgende instrukser på
                Huset:
                \begin{enumerate}
                    \item Generell gjenginstruks.
                \end{enumerate}
            \item KU disponerer og har ansvaret for følgende rom med tilhørende utstyr:
                \begin{enumerate}
                    \item KUs kontor med hybel
                \end{enumerate}
            \item KU disponerer og har ansvaret for Studentersamfundets kinomaskin.
        \end{enumerate}
    \end{instruksledd}

    \begin{instruksledd}{Formidling}
        \begin{enumerate}
            \item Gjengsjefen plikter å gjøre nye medlemmer av KU kjent med denne instruks.
        \end{enumerate}
    \end{instruksledd}

\end{instruks}


