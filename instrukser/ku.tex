
\begin{instruks}{Instruks for Kulturutvalget}{ }{ }

    \begin{instruksledd}{Form�l}
        \begin{enumerate}
            \item Kulturutvalget (KU) skal bidra til � gj�re Studentersamfundet i Trondhjem til et senter for kulturelle
arrangementer og aktiviteter. KU skal gi et kulturelt tilbud rettet mot studenter, og da spesielt mot
Samfundets medlemmer. KU er i samarbeid med Videokomiteen ansvarlig for eventuell kinodrift p�
Samfundet. KU skal arrangere Excenteraftener noen ganger i semesteret.
                    \end{enumerate}
    \end{instruksledd}

    \begin{instruksledd}{Sammensetning}
        \begin{enumerate}
            \item KU best�r av inntil 9 funksjon�rer. 
            \item En funksjon�r i KU er aktiv i 2 �r og har etter dette mulighet for � beholde funksjon�rstatusen som
pangsjonist.
            \item KU opprettholder ikke virksomheten under UKA.
        \end{enumerate}
    \end{instruksledd}
    
    \begin{instruksledd}{Ansvarsomr�de og plikter}
        \begin{enumerate}   
            \item  Medlemmer i KU plikter � kjenne innholdet av og overholde f�lgende instrukser p� Huset:
                \begin{enumerate}
                    \item Generell gjenginstruks.
                \end{enumerate}
            \item KU disponerer og har ansvaret for f�lgende rom med tilh�rende utstyr:
                \begin{enumerate}
                    \item KUs kontor med hybel
                \end{enumerate}
            \item KU disponerer og har ansvaret for Studentersamfundets kinomaskin.
        \end{enumerate}

    \begin{instruksledd}{Formidling}
        \begin{enumerate}
            \item Gjengsjefen plikter � gj�re nye medlemmer av KU kjent med denne instruks.
        \end{enumerate}
    \end{instruksledd}

\end{instruks}


