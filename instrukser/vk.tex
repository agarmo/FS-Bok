
\begin{instruks}{Instruks for Videokommiteen}{ }{ }
    
    \begin{instruksledd}{Formål}
        \begin{enumerate}
            \item Videokommiteen skal drive, vedlikeholde og sørge for 
eventuell
                utbygging av Studentersamfundet i
                Trondhjems videotekniske anlegg, samt TV-nettverk og dets tilhørende utstyr.
            \item Videokomiteen skal stå for kjøringer av 35 mm 
filmvisninger i
                Storsalen i regi av Studentersamfundet i
                Trondhjems filmklubb. De skal også stå for drift og vedlikehold av filmklubben
                kinotekniske utstyr.
            \item Videokomiteen skal assistere Student-TV i drift og 
vedlikehold av
                flerkamerariggen på Studentersamfundet.
        \end{enumerate}
    \end{instruksledd}

    \begin{instruksledd}{Sammensetning}
        \begin{enumerate}
            \item Videokomiten består av inntil 12 aktive funksjonærer. Nye medlemmer tas opp av
            gjengsjefen en gang i året.
            \item  Et medlem i Videokomiten er aktivt i 2,5 år, og har etter det mulighet for å
            beholde funksjonærstatusen som
            pangsjonist.
            \item Gjengsjefen velges av medlemmene i Videokomiten og har en funksjonstid på ett
            år. Bare den som er eller
            har vært aktivt medlem av Videokomiteen kan bli valgt til gjengsjef.
            \item Videokomiteen opprettholder sin virksomhet under ISFiT og UKA. Videokomiteen
            kan midlertidig utvides
            til det antall nødvendig for å utføre Videokomitens arbeid i forbindelse med UKA.
        \end{enumerate}
    \end{instruksledd}

     \begin{instruksledd}{Ansvarsområde og plikter}
        \begin{enumerate}
           \item Medlemmer i Videokomiteen plikter å kjenne innholdet av og overholde følgende
                instrukser på Huset:
                \begin{enumerate} 
                    \item Generell gjenginstruks
                \end{enumerate}
            \item Videokomiteen disponerer og har ansvaret for følgende rom med tilhørende utstyr:
                \begin{enumerate}
                    \item Lagerrom under 1. kvadrant i Storsalen.
                    \item Søndre del av Märthalosjen
                    \item Kontor i Søndre rotunde
                \end{enumerate}
            \item Videokomiteen har mulighet til å benytte ARKs hybel, verksted og lager på midlertidig
            basis, etter nærmere
            avtale med ARK. Videokomiteen har også ansvaret for vedlikehold av Kinorommet med
            tilhørende utstyr.
        \end{enumerate}
    \end{instruksledd}

    \begin{instruksledd}{Formidling}
        \begin{enumerate}
            \item Gjengsjefen plikter å gjøre nye medlemmer av Videokomiteen kjent med
                denne instruks.
        \end{enumerate}
    \end{instruksledd}


 \end{instruks}

