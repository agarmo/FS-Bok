\begin{instruks}{Prinsippavtale mellom Mediastud AS og Studentersamfundet i Trondhjem}{6. februar 2011}{24. februar 2011}


    Denne avtalen er ment å avklare Student-TV, Under Dusken og Radio Revolt
    (``mediegjengene'') sin rolle i forhold til Samfundet. I tillegg til denne avtalen, reguleres forholdet mellom
    eierne av Mediastud (Samfundet og Studentsamskipnaden i Trondheim) i en aksjonæravtale
    (``Aksjonæravtalen''), herunder hvilke premisser som skal ligge til grunn for forholdene mellom
    eierne og Mediastud. Mediegjengenes roller reguleres i tillegg av de respektive gjenginstrukser.


    \begin{instruksledd}{Overordnet prinsipp}
        Mediegjengene betraktes som gjenger på Samfundet. I motsetning til andre gjenger,
        styrer og finansierer Samfundet driften av mediegjengene gjennom Mediastud.  
    \end{instruksledd}

    \begin{instruksledd}{Mediegjengenes oppgaver}
        Mediegjengenes hovedoppgave og bidrag til Samfundet skjer gjennom å gi studentene i
        Trondheim et studentbasert medietilbud. Eventuelle andre oppgaver mediegjengene gjør for
        Samfundet, skal være i tråd med Aksjonæravtalen.  
    \end{instruksledd}        

    \begin{instruksledd}{Andre plikter og rettigheter}
        Mediegjengene har generelle plikter og rettigheter på lik linje med andre gjenger på
        Samfundet.  Dette skal fremgå av gjenginstruksene. Oppgaver som kommer av slike plikter, omfattes ikke
        av Aksjonæravtalen.  
    \end{instruksledd}

    \begin{instruksledd}{Dobbeltroller}
        Medarbeidere i mediegjengene har dobbeltroller ved at de er både uavhengige
        mediemedarbeidere
        og gjengmedlemmer med verv på Samfundet. I henhold til Vær Varsom-plakaten, er det
        mediegjengenes ansvar å til enhver tid sørge for en tydelig rolleavklaring.
    \end{instruksledd}


    \begin{instruksledd}{Stillinger på Samfundet}
        Samfundet skal gi gjengmedlemskap eller funksjonærstilling til medarbeidere i
        mediegjengene i henhold til hver enkelt gjengs instruks.
    \end{instruksledd}


    \begin{instruksledd}{Disponering av lokaler }
        Samfundet skal for å sikre mediegjengenes tilhørighet på Samfundet, gi et sosialt
        tilholdssted for
        medarbeidere i mediegjengene i henhold til hver enkelt gjengs instruks.
    \end{instruksledd}

    \begin{instruksledd}{Arkivering av produksjonen}
        Mediegjengene plikter å regelmessig levere alt publisert innhold til arkivering i
        Samfundets arkiv.  Arkiveringen bekostes av Samfundet.
    \end{instruksledd}
\pagebreak
    \begin{instruksledd}{Varemerket Under Dusken }
        Samfundet gir Mediastud en vederlagsfri, ubegrenset og eksklusiv bruksrett av logo
        og merkenavn
        ``Under Dusken'' (varemerke regnr 240240) så lenge Samfundet er eier av Mediastud.
        Samfundet
        plikter å sørge for fornyelse av logo- og varemerkeregisterering senest hvert
        tiende år etter 2010.
    \end{instruksledd}

    \begin{instruksledd}{ Endring av avtalen og gjenginstrukser } 
        Denne avtalen og den enkelte av mediegjengenes instruks kan kun endres i samråd
        mellom styret i
        Mediastud og Finansstyret ved Samfundet, og i tråd med Aksjonæravtalen.
    \end{instruksledd}

\end{instruks}


