\begin{instruks}{Generell gjenginstruks}{1. januar 2008}{3. april 2008}
    
    \begin{instruksledd}{Formål}
        \begin{enumerate}
            \item Alle gjenger ved Studentersamfundet i Trondhjem skal fremme og arbeide for et
                  tilbud til studentene i Trondheim og byens befolkning.
        \end{enumerate}
    \end{instruksledd}


    \begin{instruksledd}{Sammensetning}
        \begin{enumerate}
            \item Nye medlemmer tas opp en gang i semesterete. Opptak utlyses i rimelig
                tid før opptaksfristen. Bare medlemmer av Studentersamfundet kan tas opp i
                gjengene.
            \item Et medlem kan innvilges permisjon og kan også fritas fra sitt medlemskap
            \item Gjengsjefene har ansvar for gjengen innad.
        \end{enumerate}

    \end{instruksledd}


    \begin{instruksledd}{Ansvarsområde og plikter}
        \begin{enumerate}
            \item Medlemmene i Studentersamfundets gjenger plikter å kjenne innholdet av
                og overholde følgende av Studentersamfundet instrukser:
                \begin{enumerate}
                    \item Husorden
                    \item Nøkkelinstruks
                    \item Rømningsintruksen
                    \item Branninstruks
                    \item Kontrollørinstruks
                    \item Instuks for den gjeng man er medlem av
                    \item Studentersamfundets lover
                \end{enumerate}
            \item Gjengene kan etter avtale med daglig leder eller arrangerende gjenger
                legge beslag på lokaler i Huset når dette er nødvendig for gjengens
                arbeid. Arbeid skal utføres på en slik måte at det, så fremt det er mulig,
                ikke er til ulempe for annen utleie eller andre tilstelninger på Huset.
        \end{enumerate}

    \end{instruksledd}

\end{instruks}


