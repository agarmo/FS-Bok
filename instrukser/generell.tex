\begin{instruks}{Generell gjenginstruks}{1. januar 2008}{3. april 2008}
    
    \begin{instruksledd}{Formål}
        \begin{enumerate}
            \item Alle gjenger ved Studentersamfundet i Trondhjem skal fremme og arbeide for et
                  tilbud til studentene i Trondheim og byens befolkning.
        \end{enumerate}
    \end{instruksledd}


    \begin{instruksledd}{Sammensetning}
        \begin{enumerate}
            \item Nye medlemmer tas opp en gang i semesterete. Opptak utlyses i rimelig
                tid før opptaksfristen. Bare medlemmer av Studentersamfundet kan tas opp i
                gjengene.
            \item Et medlem kan innvilges permisjon og kan også fritas fra sitt medlemskap
            \item Gjengsjefene har ansvar for gjengen innad.
        \end{enumerate}

    \end{instruksledd}


    \begin{instruksledd}{Ansvarsområde og plikter}
        \begin{enumerate}
            \item Medlemmene i Studentersamfundets gjenger plikter å kjenne innholdet av
                og overholde følgende av Studentersamfundet instrukser:
                \begin{enumerate}
                    \item Husorden
                    \item Nøkkelinstruks
                    \item Rømningsintruksen
                    \item Branninstruks
                    \item Kontrollørinstruks
                    \item Instuks for den gjeng man er medlem av
                    \item Studentersamfundets lover
                \end{enumerate}
            \item Gjengene kan etter avtale med daglig leder eller arrangerende gjenger
                legge beslag på lokaler i Huset når dette er nødvendig for gjengens
                arbeid. Arbeid skal utføres på en slik måte at det, så fremt det er mulig,
                ikke er til ulempe for annen utleie eller andre tilstelninger på Huset.
        \end{enumerate}

    \end{instruksledd}

	\begin{instruksledd}{Økonomi og regnskap}
		\begin{enumerate}
			\item Økonomirapportering
			\begin{enumerate}  
				\item Gjengene skal oversende periodiske økonomirapporter til
				Gjengsekretariatet pr. 5. hver måned i semestrene.
				\item Økonomirapportering skal inneholde oversikt over:
				\begin {enumerate} 
					\item Internregnskap
					\item Kasse og bank/postgirobeholdning
					\item Ubetalte regninger på over kr. 2000
					\item Eventuelle annonseintekter
					\item Andre inntekter
					\item Overskridelser av budsjett på mer enn kr. 2000
				\end {enumerate}
				\item Inntekter og utgifter skal sammenstilles med periodiserte 
				budsjetttall.
			\end {enumerate}
			\item {Regnskap og budsjett}
			\begin {enumerate}
				\item Gjengene skal ha en egen økonomiansvarlig med ansvar for intern 
				regnskapsførsel.
				\item Bilag skal registreres og oversendes regnskapsfører etter avsluttet 
				regnskapsår og gjengen skal føre løpende regnskap over alle inn- og 
				utbetalinger.
				\item Gjengene skal levere budsjettforslag for påfølgende regnskapsår til 
				Gjengsekrtariatet innen primo november.
			\end {enumerate}
		\end{enumerate}

	\end{instruksledd}

	\begin{instruksledd}{Endring av instruksen}
		Endringer av denne instruksen skal skje i samsvar med Studentersamfundets lover og i samråd 
		med Gjengsjefkollegiet. Alle endringer skal forelegges Finanstyret for godkjenning.

	\end{instruksledd}

	\begin{instruksledd}{Formidling}
		Gjengsjefene plikter å gjøre nye medlemmer av sin gjeng kjent med denne instruks.

	\end{instruksledd}
 	
\end{instruks}


