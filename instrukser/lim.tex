
\begin{instruks}{Instruks for Layout Info Marked}{1. }{ }

    \begin{instruksledd}{Form�l}
        \begin{enumerate}
            \item Layout Info Marked (LIM) skal drive helhetlig markedsf�ring av Studentersamfundet i Trondhjem mot
									Studentersamfundets medlemmer, studenter, byens befolkning og potensielle samarbeidspartnere.
        \end{enumerate}
    \end{instruksledd}

    \begin{instruksledd}{Sammensetning}
        \begin{enumerate}
            \item LIM er delt opp i fire undergrupper: Administrasjon, Info, Marked, Layout og Web.
            \item LIM best�r av inntil 32 aktive medlemmer, av disse inntil 17 funksjon�rer.
            \item En funksjon�r i LIM er aktiv i 2 �r. Gjengmedlemmene i web er aktive i 1,5 �r, de resterende i 1 �r.
            \item LIM skal ha en gjengsjef og en �konomiansvarlig. Resterende stillinger fastsettes av LIM i felleskap.
									Gjengsjefen velges av LIM for 1 �r av gangen.
						\item Bortsett fra gjengsjef og �konomiansvarlig tilh�rer funksjon�rene en av undergruppene nevnt i punkt 2.a.
									Alle seksjoner utenom Layout har gjengmedlemmer. Arbeidet foreg�r i de enkelte gruppene, men det skal
									legges til rette for god kommunikasjon mellom gruppene for � best mulig h�ndtere prosjekter som spenner p�
									tvers av gruppene.
						\item LIM opprettholder ikke virksomheten under UKA.
        \end{enumerate}
     \end{instruksledd}

     \begin{instruksledd}{Ansvarsomr�de og plikter}
        \begin{enumerate}   
            \item  Medlemmer i LIM plikter � kjenne innholdet av og overholde f�lgende instrukser p� Huset:
                \begin{enumerate}
                    \item Generell gjenginstruks.
                \end{enumerate}
            \item LIM disponerer og har ansvaret for f�lgende rom med tilh�rende utstyr:
                \begin{enumerate}
                    \item Kontor
                    \item Nye Birkeland Bar
                    \item Plakatrommet p� Gjenganretningen
                    \item Det hemmelige rommet7
                    \item Bl�sten
                \end{enumerate}
 						\item LIM skal utgi et program for Studentersamfundets arrangementer hvert semester.
 						\item LIM er ansvarlig for at det produserers og distribueres n�dvendige plakater og bannere for
									Studentersamfundets arrangementer i semesteret. Disse avtales i samarbeid med Studentersamfundets
									arrangerende gjenger. I tillegg har LIM ansvaret for gesimsen.
						\item LIM har som ansvar � selge medlemskort, og � bemanne medlemskortluka under arrangementer p� Studentersamfundet.
						\item LIM skal formulere et m�ldokument for gjengen. M�ldokumentet skal revideres �rlig av ny gjengsjef.
						\item LIM skal ha et kommunikasjonsutvalg best�ende av lederne for hver gruppe, samt gjengsjef og
�konomiansvarlig. M�let med denne gruppen er � s�rge for god kommunikasjon mellom gruppene, og
forbrede saker til LIMs allm�te.
						\item LIM skal hvert v�rsemester gjennomf�re en unders�kelse blant studentene i Trondheim for � kartlegge
hvordan Studentersamfundets arrangementer og tilbud oppfattes. De skal hvert h�stsemester gjennomf�re en
intern unders�kelse for � kartlegge hvordan arbeidet p� Samfundet oppfattes av gjengmedlemmene.
        \end{enumerate}
    \end{instruksledd}
        
		\begin{instruksledd}{Formidling}
        \begin{enumerate}
            \item Gjengsjefen plikter � gj�re nye medlemmer av LIM kjent med denne instruks.
        \end{enumerate}
    \end{instruksledd}


\end{instruks}