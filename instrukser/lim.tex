
\begin{instruks}{Instruks for Layout Info Marked}{1. }{ }

    \begin{instruksledd}{Formål}
        \begin{enumerate}
            \item Layout Info Marked (LIM) skal drive helhetlig markedsføring av Studentersamfundet
i Trondhjem mot
								
Studentersamfundets medlemmer, studenter, byens befolkning og potensielle samarbeidspartnere.
        \end{enumerate}
    \end{instruksledd}

    \begin{instruksledd}{Sammensetning}
        \begin{enumerate}
            \item LIM er delt opp i fire undergrupper: Administrasjon, Info, Marked, Layout og Web.
            \item LIM består av inntil 32 aktive medlemmer, av disse inntil 17 funksjonærer.
            \item En funksjonær i LIM er aktiv i 2 år. Gjengmedlemmene i web er aktive i 1,5 år, de
resterende i 1 år.
            \item LIM skal ha en gjengsjef og en økonomiansvarlig. Resterende stillinger fastsettes
av LIM i felleskap.
								
Gjengsjefen velges av LIM for 1 år av gangen.
						\item Bortsett fra gjengsjef og
økonomiansvarlig tilhører funksjonærene en av undergruppene nevnt i punkt 2.a.
									Alle
seksjoner utenom Layout har gjengmedlemmer. Arbeidet foregår i de enkelte gruppene, men det skal
									legges til
rette for god kommunikasjon mellom gruppene for å best mulig håndtere prosjekter som spenner på
									tvers av
gruppene.
						\item LIM opprettholder ikke
virksomheten under UKA.
        \end{enumerate}
     \end{instruksledd}

     \begin{instruksledd}{Ansvarsområde og plikter}
        \begin{enumerate}   
            \item  Medlemmer i LIM plikter å kjenne innholdet av og overholde følgende instrukser på
Huset:
                \begin{enumerate}
                    \item Generell gjenginstruks.
                \end{enumerate}
            \item LIM disponerer og har ansvaret for følgende rom med tilhørende utstyr:
                \begin{enumerate}
                    \item Kontor
                    \item Nye Birkeland Bar
                    \item Plakatrommet på Gjenganretningen
                    \item Det hemmelige rommet7
                    \item Blæsten
                \end{enumerate}
		\item LIM skal utgi et program for
Studentersamfundets arrangementer hvert semester.
		\item LIM er ansvarlig for at det
produserers og distribueres nødvendige plakater og bannere for
		Studentersamfundets
arrangementer i semesteret. Disse avtales i samarbeid med Studentersamfundets
		arrangerende
gjenger. I tillegg har LIM ansvaret for gesimsen.
		\item LIM har som ansvar å selge medlemskort, og å
bemanne medlemskortluka under arrangementer på Studentersamfundet.
		\item LIM skal formulere et måldokument for
gjengen. Måldokumentet skal revideres årlig av ny gjengsjef.
		\item LIM skal ha et kommunikasjonsutvalg
bestående av lederne for hver gruppe, samt gjengsjef og
økonomiansvarlig. Målet med denne gruppen er å sørge for god kommunikasjon mellom gruppene, og
forbrede saker til LIMs allmøte.
		\item LIM skal hvert vårsemester gjennomføre en
undersøkelse blant studentene i Trondheim for å kartlegge
hvordan Studentersamfundets arrangementer og tilbud oppfattes. De skal hvert høstsemester
gjennomføre en
intern undersøkelse for å kartlegge hvordan arbeidet på Samfundet oppfattes av gjengmedlemmene.
        \end{enumerate}
    \end{instruksledd}
        
		\begin{instruksledd}{Formidling}
        \begin{enumerate}
            \item Gjengsjefen plikter å gjøre nye medlemmer av LIM kjent med denne instruks.
        \end{enumerate}
    \end{instruksledd}


\end{instruks}