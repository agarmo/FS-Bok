\begin{instruks}{Instruks for Markedsføringsgjengen}{20. november 2011}{13. desember 2011 }

    \begin{instruksledd}{Formål}
        \begin{enumerate}
            \item Markedsføringsgjengen (MG) skal drive helhetlig markedsføring av Studentersamfundet
                i Trondhjem mot
                Studentersamfundets medlemmer, studenter, byens befolkning og potensielle samarbeidspartnere.
        \end{enumerate}
    \end{instruksledd}

    \begin{instruksledd}{Sammensetning}
        \begin{enumerate}
            \item MG er delt opp i fire undergrupper: Administrasjon, Info, Marked, Layout og Web.
            \item MG består av inntil 37 aktive medlemmer, av disse inntil 22 funksjonærer.
            \item En funksjonær i MG er aktiv i 2 år. Gjengmedlemmene i web er aktive i 1,5 år, de
                resterende i 1 år.
            \item MG skal ha en gjengsjef, en nestleder, en økonomiansvarlig og en strategiansvarlig. Resterende stillinger fastsettes
                av MG i felleskap.

                Gjengsjefen velges av MG for 1 år av gangen.
            \item Bortsett fra gjengsjef, nestleder, økonomiansvarlig og strategiansvarlig tilhører funksjonærene en av undergruppene nevnt i punkt 2.a.
                Alle
                seksjoner utenom Layout har gjengmedlemmer. Arbeidet foregår i de enkelte gruppene, men det skal
                legges til
                rette for god kommunikasjon mellom gruppene for å best mulig håndtere prosjekter som spenner på
                tvers av
                gruppene.
            \item MG opprettholder ikke virksomheten under UKA.
        \end{enumerate}
    \end{instruksledd}

    \begin{instruksledd}{Ansvarsområde og plikter}
        \begin{enumerate}   
            \item  Medlemmer i MG plikter å kjenne innholdet av og overholde følgende instrukser på
                Huset:
                \begin{enumerate}
                    \item Generell gjenginstruks.
                \end{enumerate}
            \item MG disponerer og har ansvaret for følgende rom med tilhørende utstyr:
                \begin{enumerate}
                    \item Kontor
                    \item Nye Birkeland Bar
                    \item Plakatrommet på Gjenganretningen
                    \item Det hemmelige rommet
                \end{enumerate}
            \item MG skal utgi et program for
                Studentersamfundets arrangementer hvert semester.
            \item MG er ansvarlig for at det
                produserers og distribueres nødvendige plakater og bannere for
                Studentersamfundets
                arrangementer i semesteret. Disse avtales i samarbeid med Studentersamfundets
                arrangerende
                gjenger. I tillegg har MG ansvaret for gesimsen.
            \item MG har som ansvar å selge medlemskort, og å
                bemanne medlemskortluka under arrangementer på Studentersamfundet.
            \item MG skal formulere et måldokument for
                gjengen. Måldokumentet skal revideres årlig av ny gjengsjef.
            \item MG skal ha et kommunikasjonsutvalg
                bestående av lederne for hver gruppe, samt gjengsjef og
                økonomiansvarlig. Målet med denne gruppen er å sørge for god kommunikasjon mellom gruppene, og
                forbrede saker til MGs allmøte.
            \item MG skal ved behov gjennomføre en
                undersøkelse blant studentene i Trondheim for å kartlegge
                hvordan Studentersamfundets arrangementer og tilbud oppfattes. De skal hvert høstsemester
                gjennomføre en
                intern undersøkelse for å kartlegge hvordan arbeidet på Samfundet oppfattes av gjengmedlemmene.
        \end{enumerate}
    \end{instruksledd}

    \begin{instruksledd}{Formidling}
        \begin{enumerate}
            \item Gjengsjefen plikter å gjøre nye medlemmer av MG kjent med denne instruks.
        \end{enumerate}
    \end{instruksledd}


\end{instruks}

