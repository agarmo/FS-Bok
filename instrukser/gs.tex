
\begin{instruks}{Instruks for Gjengsekretariatet}{}{}

    \begin{instruksledd}{Form�l}
        \begin{enumerate}
            \item Gjengsekretariatet (GS) skal drive saksforberedelse og -oppf�lging for Finansstyret (FS) i forhold som vedr�rer
foreningsvirksomheten og gjennomf�re vedtak fattet av FS i samme saksomr�de. GS
skal s�rlig virke i administrative og felles anliggender for gjengene, og i saker som ikke fanges opp av
eksisterende gjengers ansvarsomr�der. GS er underlagt FS.
            \item  GS skal v�re koordinerende, samarbeidsskapende og arbeide for � bedre kommunikasjonen
mellom gjengene, forretningsdriften og FS. GS skal fungere som ressurs- og
kompetansesenter for gjengene, og bidra til erfaringsoverf�ring og kontinuitet i foreningsvirksomheten.
GS vil ogs� arbeide med organisatoriske oppdrag gitt av FS og etter eget initiativ.
        \end{enumerate}
    \end{instruksledd}

    \begin{instruksledd}{Sammensettning}
        \begin{enumerate}
            \item GS best�r av 4-10 aktive medlemmer, etter aktivitetsniv� og behov for � ivareta kompetanse.
Antallet medlemmer fastsettes av FS etter innstilling fra GS. GS
utlyser opptak av nye medlemmer ved behov. Medlemmene oppnevnes av FS etter innstilling fra
GS og forutg�ende behandling av innstillingen i Gjengsjefkollegiet. Bare medlemmer av
Studentersamfundet kan v�re medlemmer av GS.
            \item Medlemmer av GS skal normalt v�re aktive i minimum 1 �r. Ved innstilling av nye
medlemmer skal GS s�rge for kompetanseoverf�ring og at det sikres en kontinuerlig drift.
Medlemmene av GS kan innvilge et medlem permisjon, og kan ogs� frita et medlem for sitt
medlemskap.
            \item  GS' medlemmer skal normalt rekrutteres blant n�v�rende eller tidligere tillitsvalgte ved
Studentersamfundet. GS' sammensetning b�r gjenspeile bredden i foreningsvirksomheten og
medlemmene skal likestille alle aktiviteter som drives i Studentersamfundet.
            \item GS' kan sette ned undergrupper av begrenset varighet for s�rskilte form�l. Gjensekretariatet
oppnevner medlemmer av slike undergrupper etter behov.
            \item GS opprettholder virksomheten under UKA
        \end{enumerate}
    \end{instruksledd}

    \begin{instruksledd}{Ansvarsomr�der og plikter}
        \begin{enumerate}
            \item Medlemmer i GS plikter � kjenne innholdet av og overholde
	    f�lgende instrukser p� Huset:
                \begin{enumerate}
                    \item Generell gjenginstruks
                    \item Husorden
                    \item Branninstruks
                    \item Studentersamfundets lover
                \end{enumerate}
            \item Medlemmer av GS disponerer kontorplass etter avtale med daglig leder.
            \item Lederen av GS skal m�te i FS' m�ter. Et annet medlem kan v�re sekret�r i
Gjengsjefkollegiet.
            \item GS skal samarbeide med daglig leder og leder for Gjengsjefkollegiet.
        \end{enumerate}
    \end{instruksledd}

    \begin{instruksledd}{Formidling}
        \begin{enumerate}
            \item GS' leder plikter � gj�re nye medlemmer i GS kjent med denne instruks.
        \end{enumerate}
    \end{instruksledd}


\end{instruks}
