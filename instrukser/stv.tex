
\begin{instruks}{Instruks for Student-TV}{1. }{}

    \begin{instruksledd}{Formål}
        \begin{enumerate}
            \item Student-TV skal produsere TV for studentene i Trondheim og byens befolkning.
        \end{enumerate}
    \end{instruksledd}

    \begin{instruksledd}{Sammensetning}
        \begin{enumerate}
            \item Student-TV består av 85 medlemmer, hvorav 15 er funksjonærer. Nye medlemmer tas
                opp en gang i semesteret, med mulighet for suppleringsopptak utover dette. Bare medlemmer av
                Studentersamfundet kan tas opp i Student-TV.
            \item Et medlem av Student-TV er aktivt i 2 år. Et medlem kan innvilges permisjon av
                Styret i Student-TV, og kan også fritas fra sitt gjengmedlemskap.
            \item Ansvarlig redaktør innstilles av Student-TVs allmøte og velges av Styret i
                Mediastud AS. Redaktør velges for ett år av gangen. Redaktøren er ansvarlig for utformingen av det
                redaksjonelle innholdet samt kvaliteten på dette.
            \item Daglig leder velges av Student-TVs allmøte. Daglig leder er gjengsjef og
                har ansvar for den dagligedriften og gjengen internt på Samfundet. Daglig leder velges for ett år av
                gangen.
            \item Student-TV opprettholder normal produksjon under UKA.
        \end{enumerate}
    \end{instruksledd}

    \begin{instruksledd}{Ansvarsområde og plikter}
        \begin{enumerate}   
            \item  Medlemmer i Student-TV plikter å kjenne innholdet av og overholde følgende
                instrukser på Huset:
                \begin{enumerate}
                    \item Generell gjenginstruks.
                \end{enumerate}
            \item Student-TV disponerer og har ansvaret for følgende rom med tilhørende utstyr:
                \begin{enumerate}
                    \item Student-TV hybel Einar Førdes minne.
                    \item Kontrollrom for flerkameraproduksjon
                \end{enumerate}
            \item Student-TV plikter å utføre arbeid i henhold til punkt 1. i tidsrommene
                medio august til medio desember og medio januar til medio juni. Dato for arbeid og andre tidsfrister
                settes av redaktøren.
            \item Gjengsjefen er ansvarlig for Student-TV utstyr, og for at dette blir fagmessig
                betjent. Gjengsjefen er også ansvarlig for at Student-TVs utstyr blir forsvarlig sikret mot tyveri
                og skade, og at brannvernforskrifter og sikringsbestemmelser blir overholdt på Student-TVs område.
        \end{enumerate}
    \end{instruksledd}

    \begin{instruksledd}{Økonomi}
        \begin{enumerate}
            \item Gjengsjefen er ansvarlig for den biten av økonomien som ikke er knyttet til
                redaksjonell drift avStudent-TV.
        \end{enumerate}
    \end{instruksledd}

    \begin{instruksledd}{Endring av instruksen}
        \begin{enumerate}
            \item Endringer av denne instruksen skal skje i samsvar med Studentersamfundets
                lover og i samråd med Student-TVs medlemmer. Alle endringer skal forelegges Finansstyret til
                uttalelse og den nye instruksen er først gyldig når den er godkjent av styret i Mediastud AS.
        \end{enumerate}
    \end{instruksledd}

    \begin{instruksledd}{Formidling}
        \begin{enumerate}
            \item Gjengsjefen plikter å gjøre nye medlemmer av Student-TV kjent med denne
                instruks.
        \end{enumerate}
    \end{instruksledd}


\end{instruks}


