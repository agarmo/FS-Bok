\begin{instruks}{Instruks for Fotogjengen}{ }{ }

    \begin{instruksledd}{Formål}
        \begin{enumerate}
            \item Fotogjengens formål er å utarbeide fotografiske reportasjer fra Studentersamfundet i
                Trondhjems arrangementer, samt bestyre Samfundets bilde- og filmarkiv. Fotogjengen har i dag ansvar
                for å ta digitale miljøbilder til Samfundets nettsider. Fotogjengen plikter å sørge for fotografisk
                dokumentasjon av aktivitetene på Studentersamfundet.
        \end{enumerate}
    \end{instruksledd}
    
    \begin{instruksledd}{Sammensetning}
        \begin{enumerate} 
            \item Fotogjengen består av inntil 9 funksjonærer. Nye medlemmer tas opp hvert år av
                gjengsjef i Fotogjengen.
            \item Et medlem i Fotogjengen er aktivt i 2 ½ år, og har etter det mulighet for å beholde
                funksjonærstatusen som pangsjonist.
            \item Fotogjengen opprettholder sin virksomhet under Isfit og UKA. Fotogjengen kan utvides
                midlertidig til det antall nødvendig for å utføre FGs arbeid i forbindelse med UKA..
        \end{enumerate}
    \end{instruksledd}

    \begin{instruksledd}{Ansvarsområde og plikter}
        \begin{enumerate}
            \item Medlemmer i Fotogjengen plikter å kjenne innholdet av og overholde følgende instrukser
                på Huset:
                \begin{enumerate}
                    \item Generell gjenginstruks.
                \end{enumerate}
            \item Fotogjengen disponerer og har ansvaret for følgende rom med tilhørende utstyr
                \begin{enumerate}
                    \item Fotogjengen hybel.
                    \item Mørkerom og arbeidsrommet ``Kapellet''
                \end{enumerate}
            \item Gjengsjefen er ansvarlig for Fotogjengens utstyr, og for at dette blir fagmessig
                betjent. Gjengsjefen er også ansvarlig for at Fotogjengens utstyr blir forsvarlig sikret mot tyveri og skade, og at
                brannvernforskrifter og sikringsbestemmelser blir overholdt på Fotogjengens område.
        \end{enumerate}
    \end{instruksledd}
    
    \begin{instruksledd}{Formidling}
        \begin{enumerate}
            \item Gjengsjefen plikter å gjøre nye medlemmer av Fotogjengen kjent med denne
                instruks.
        \end{enumerate}
    \end{instruksledd}

\end{instruks}
