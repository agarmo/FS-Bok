
\begin{instruks}{Instruks for Klubbstyret}{ }{ }

    \begin{instruksledd}{Form�l}
        \begin{enumerate}
            \item Klubbstyret (KLST) har ansvaret for � planlegge og arrangere fredagskveldene p� Studentersamfundet i
Trondhjem og skal bidra til � gj�re Studentersamfundet til et senter for kulturelle arrangementer og
aktiviteter, samt tilby et utelivstilbud til Studentersamfundets medlemmer. KLST er ogs� ansvarlig for
husorden og drifter Studentersamfundets bibliotek.
        \end{enumerate}
    \end{instruksledd}

    \begin{instruksledd}{Sammensetning}
        \begin{enumerate}
            \item KLST best�r av inntil 20 faste medlemmer, hvorav 8 er funksjon�rer.
            \item En funksjon�r i KLST er aktiv i 2 �r og har etter dette mulighet for � beholde funksjon�rstatusen som
pangsjonist.
            \item Husmann er gjengsjef i KLST og velges av Klubbstyrets medlemmer, funksjon�rtiden er 1 �r. Bare den som
er/har v�rt medlem i Klubbstyret kan bli Husmann.
            \item KLST opprettholder ikke virksomheten under UKA.
        \end{enumerate}
     \end{instruksledd}

     \begin{instruksledd}{Ansvarsomr�de og plikter}
        \begin{enumerate}
            \item  Medlemmer i KLST plikter � kjenne innholdet av og overholde f�lgende instrukser p� Huset:
                \begin{enumerate}
                    \item Generell gjenginstruks
                \end{enumerate}
            \item KLST disponerer og har ansvaret for f�lgende rom med tilh�rende utstyr:
                \begin{enumerate}
                    \item Klubbstyrets hybel
                    \item Bibliotek med tilh�rende kontor, lager og kj�kken
                    \item Billettbodene ved inngang (i samarbeids med L�rdagskomiteen)
                \end{enumerate}
        \end{enumerate}
    \end{instruksledd}

    \begin{instruksledd}{Formidling}
        \begin{enumerate}
            \item Gjengsjefen plikter � gj�re nye medlemmer av KLST kjent med denne instruks.
        \end{enumerate}
    \end{instruksledd}


\end{instruks}


