\begin{instruks*}{Instruks for Forsterkerkomit\'een}

    \begin{instruksledd}{Formål}
        \begin{enumerate}
            \item Forsterkerkomiteen skal drive, vedlikeholde og sørge for eventuell utbygging av
            Studentersamfundet i
            Trondhjems lydtekniske utstyr, samt Husets interne telefonanlegg.
            \item Unntatt fra forholdene nevnt i pkt.1.a er Radio Revolts lydtekniske utstyr, samt drift
            og vedlikehold av telefonanlegg tilknyttet permanent utleide lokaler.
        \end{enumerate}
    \end{instruksledd}

    \begin{instruksledd}{Sammensettning}
        \begin{enumerate}
            \item Forsterkerkomiteem en består av inntil 22 aktive funksjonærer. Nye medlemmer tas opp
                hvert år av gjengsjefen.
            \item Et medlem i Forsterkerkomiteen er aktivt i 3 ½ år, og har etter det mulighet for å
                beholde funksjonærstatusen som pangsjonist.
            \item Gjengsjefen velges av Forsterkerkomiteens medlemmer. Funksjonstiden er 1 år. Bare den
                som er/har vært medlem i Forsterkerkomiten kan bli gjengsjef i FK.
            \item Forsterkerkomiteen opprettholder virksomheten under ISFiT og UKA.
                Forsterkerkomiteen kan utvides midlertidig til det antall nødvendig for å utføre FKs arbeid i forbindelse med UKA.
        \end{enumerate}
    \end{instruksledd}

    \begin{instruksledd}{Ansvarsområder og plikter}
        \begin{enumerate}
            \item Medlemmer i Forsterkerkomiteen plikter å kjenne innholdet av og overholde følgende
                instrukser på Huset:
                \begin{enumerate}
                    \item Generell gjenginstruks
                \end{enumerate}
            \item Forsterkerkomiteen disponerer og har ansvaret for følgende rom med utstyr:
                \begin{enumerate}
                    \item Sedelighetsdumpa
                     \item FK-gang
                    \item Dass
                    \item Psildre
                    \item Telefonsentralen
                    \item Byssa
                    \item Messa
                    \item Labben
                    \item FK-kjørerom
                    \item Studio, herunder opptaksrom og kontrollrom
                    \item Høytlager
                    \item Høyttalerlager
                    \item Printlab
                    \item Datarom
                    \item Kassererkontoret
                    \item FK-bibliotek
                    \item Kuppelen fra Labben til Studentradion
                    \item FK-sidescene
                    \item Konge-losjen
                \end{enumerate}
            \item Forsterkerkomiteen plikter ikke å utføre arbeid dersom det ikke gis beskjed
                før mandag klokken fire samme
                uke som arbeidet skal utføres. Ved telefonreparasjon og opprettelse av nye
                telefonlinjer gjelder egne frister
                satt av gjengsjefen.
            \item Forsterkerkomiteen har et spesielt ansvar for at personer som
                kommer inn gjennom dører til gjengens lokaler
                ikke volder skade eller annen ugagn på kuppelen.
            \item Forsterkerkomiteen har rett til å selv kjøre lyd på alle arrangement på
                Knaus. Eksterne teknikere får ikke
                bistå Forsterkerkomiteen.
        \end{enumerate}
    \end{instruksledd}

    \begin{instruksledd}{Formidling}
        \begin{enumerate}
            \item  Gjengsjefen plikter å gjøre nye medlemmer av Forsterkerkomiteen
                kjent med denne instruks.
        \end{enumerate}
    \end{instruksledd}


\end{instruks*}
