
\begin{instruks}{Instruks for Radio Revolt}{1. }{ }

    \begin{instruksledd}{Formål}
        \begin{enumerate}
            \item Radio Revolt, studentradioen i Trondheim skal drive radiosendinger for studentene
                i Trondheim og byens befolkning.
        \end{enumerate}
    \end{instruksledd}

    \begin{instruksledd}{Sammensetning}
        \begin{enumerate}
            \item Radio Revolt består av 90 medlemmer, hvorav 15 er funksjonærer. Nye medlemmer tas
                opp en gang i semesteret, med mulighet for suppleringsopptak utover dette. Bare medlemmer av
                Studentersamfundet kan tas opp i Radio Revolt.
            \item Et medlem av Radio Revolt er aktivt i 2 år. Et medlem kan innvilges permisjon av
                redaksjonen og kan også fritas fra sitt gjengmedlemskap.
            \item Ansvarlig redaktør innstilles av Radio Revolts allmøte og velges av Styret i
                Mediastud AS. Redaktør velges for ett år av gangen. Redaktøren er ansvarlig for utformingen av det
                redaksjonelle innholdet samt kvaliteten på dette.
            \item Daglig leder innstilles av redaktør og velges av Radio Revolts allmøte.
                Daglig leder er gjengsjef og har ansvar for den dagligedriften og gjengen internt på Samfundet.
                Daglig leder velges for ett år av gangen.
            \item Radio Revolt opprettholder normale sendinger under UKA.
        \end{enumerate}
    \end{instruksledd}
    
    
    \begin{instruksledd}{Ansvarsområde og plikter}
        \begin{enumerate}   
            \item  Medlemmer i Radio Revolt plikter å kjenne innholdet av og overholde følgende
                instrukser på Huset:
                \begin{enumerate}
                    \item Generell gjenginstruks.
                \end{enumerate}
            \item Radio Revolt disponerer og har ansvaret for følgende rom med tilhørende utstyr:
                \begin{enumerate}
                    \item Teknikerrom
                    \item Studio
                    \item Produksjonsrom
                    \item Radiogangen
                    \item Radio Revolts hybel
                    \item Klangrom m/kontoret
                \end{enumerate}
            \item Radio Revolt plikter å utføre arbeid i henhold til punkt 1. i tidsrommene
                medio august til medio desember og medio januar til medio juni. Dato for arbeid og andre tidsfrister
                settes av redaktøren.
            \item Redaktøren er ansvarlig for Radio Revolt utstyr, og for at dette blir fagmessig
                betjent. Gjengsjefen er også ansvarlig for at Radio Revolts utstyr blir forsvarlig sikret mot tyveri
                og skade, og at brannvernforskrifter og sikringsbestemmelser blir overholdt på Radio Revolts område.
        \end{enumerate}
    \end{instruksledd}


    \begin{instruksledd}{Økonomi}
        \begin{enumerate}
            \item Gjengsjefen er ansvarlig for den biten av økonomien som ikke er knyttet til
                redaksjonell drift avRadio Revolt.
        \end{enumerate}
    \end{instruksledd}
    
    \begin{instruksledd}{Endring av instruksen}
        \begin{enumerate}
            \item Endringer av denne instruksen skal skje i samsvar med Studentersamfundets
                lover og i samråd med Radio Revolts medlemmer. Alle endringer skal forelegges Finansstyret til
                uttalelse og den nye instruksen er først gyldig når den er godkjent av styret i Mediastud AS.
        \end{enumerate}
    \end{instruksledd}
    
    \begin{instruksledd}{Formidling}
        \begin{enumerate}
            \item Gjengsjefen plikter å gjøre nye medlemmer av Radio Revolt kjent med denne
                instruks.
        \end{enumerate}
    \end{instruksledd}

\end{instruks}


