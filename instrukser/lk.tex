
\begin{instruks}{Instruks for L�rdagskomiteen}{ }{ }

    \begin{instruksledd}{Form�l}
        \begin{enumerate}
            \item L�rdagskomiteen (L�K) har ansvaret for planlegging og gjennomf�ring av l�rdagenes kultur- og
utelivstilbud p� Studentersamfundet i Trondhjem, med unntak av Samfundsm�tet, og skal i samarbeid med
Styret videreutvikle Studentersamfundets tilbud til medlemmer og andre bes�kende.
        \end{enumerate}
    \end{instruksledd}

    \begin{instruksledd}{Sammensetning}
        \begin{enumerate}
            \item L�K best�r av inntil 16 faste medlemmer, hvorav 8 er funksjon�rer.
            \item En funksjon�r i L�K er aktiv i 2 �r og har etter dette mulighet for � beholde funksjon�rstatusen som
pangsjonist. Et gjengmedlem er aktivt i 1 �r.
            \item L�K opprettholder ikke virksomheten under UKA.
        \end{enumerate}
     \end{instruksledd}

     \begin{instruksledd}{Ansvarsomr�de og plikter}
        \begin{enumerate}
            \item  Medlemmer i L�K plikter � kjenne innholdet av og overholde f�lgende instrukser p� Huset:
                \begin{enumerate}
                    \item Generell gjenginstruks.
                \end{enumerate}
            \item L�K disponerer og har ansvaret for f�lgende rom med tilh�rende utstyr:
                \begin{enumerate}
                    \item Styrets hybel og kontor Friheten (i samarbeid med Styret)
                    \item Billettboden ved inngang (i samarbeid med Klubbstyret)
                    \item L�K har mulighet til � benytte Styrets hybel p� midlertidig basis, etter n�rmere avtale med Styret.
                \end{enumerate}
        \end{enumerate}
    \end{instruksledd}

    \begin{instruksledd}{Formidling}
        \begin{enumerate}
            \item Gjengsjefen plikter � gj�re nye medlemmer av L�K kjent med denne instruks.
        \end{enumerate}
    \end{instruksledd}

\end{instruks}