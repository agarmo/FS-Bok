
\addcontentsline{toc}{chapter}{Instrukser for Finansstyrets undergrupper og organer}

\begin{instruks}{Instruks for Arkivet}{1. januar 2008}{3. april 2008}

    \begin{instruksledd}{Formål}
        \begin{enumerate}
            \item Arkivet ved Studentersamfundet i Trondhjem skal ha som oppgave å
                arkivere all skriflig dokumentasjon av alle deler av aktiviteten i
                Studentersamfundet. Dette skal gjøres systematisk, oversiktlig og
                tilgjengelig. Arkivets overordnede oppgaver er definert ut fra
                Studentersamfundets lover, § 24. Arkivet har ansvar for å tilse at
                materiale blir samlet inn og vurdert. Arkivet er underlagt Finansstyret.
        \end{enumerate}
    \end{instruksledd}

    \begin{instruksledd}{Sammensetning}
        \begin{enumerate}
            \item Arkivet består av 4 aktive medlemmer. Nye medlemmer tas opp ved behov.
                Opptaket utlyses i rimelig tid før opptaksfristen. Bare
                Studentersamfundets medlemmer kan tas opp som nye medlemmer i Arkivet.
            \item Et medlem i Arkivet er aktiv i minimum 2 år, og kan deretter søke om å bli
                pangsjonist med mulighet for å forsette å virke så lenge som ønskelig. Et medlem kan innvilges permisjon av
                Hovedarkivar, og kan også fritas sitt medlemsskap.
            \item Hovedarkivar innstilles fra Arkivet og funksjonstiden er ett år.
            \item Arkivet opprettholder virksomheten under UKA.
        \end{enumerate}
    \end{instruksledd}

    \begin{instruksledd}{Ansvarsområde og plikter}
        \begin{enumerate}
            \item Medlemmer i Arkivet plikter å kjenne innholdet av og overholde følgende instrukser
                på Huset:
                \begin{enumerate}
                    \item Generell gjenginstruks.
                \end{enumerate}
            \item Arkivet disponerer og har ansvaret for følgende rom med tilhørende utstyr:
                \begin{enumerate}
                    \item Kontor
                    \item Arkivet i nordre rotunde
                    \item Arkivrommet på Trafoen (deles med Trafogjengen)
                \end{enumerate}
            \item Medlemmer i Arkivet disponerer nøkler etter nøkkelinstruks, og har i forbindelse
                med arbeid adgang til rotunden på Huset, og arkivrommet på Trafoen.
            \item Arkivet skal ved starten av semesteret orientere de andre gjengene om Arkivets
                funksjon samt regler, plikter og rutiner for arkivering av materiale, sentralt så vel som internt i gjengen.
            \item Hovedarkivaren er ansvarlig for Arkivets utstyr og at dette blir fagmessig
                betjent. Hovedarkivaren er også ansvarlig for at Arkivets utstyr blir forsvarlig sikret mot tyveri og skade, og at
                brannvernforskrifter og sikringsbestemmelser blir overholdt på Arkivets område.
        \end{enumerate}
    \end{instruksledd}

    \begin{instruksledd}{Økonomi og regnskap}
        \begin{enumerate}
            \item Arkivet kan få bevilget midler til drift, investeringer og forpleining over
                Studentersamfundets budsjett, og plikter å føre regnskap over disse midlene. Regnskapsperioden følger kalenderåret og
                regnskapet skal foreligge til revisjon senest en måned etter regnskapsperiodens slutt. Regnskapet
                skal godkjennes av Gjengsekretariatet.
        \end{enumerate}
    \end{instruksledd}

    \begin{instruksledd}{Endringer av instruksen}
        \begin{enumerate}
            \item Endringer av denne instruksen skal skje i samsvar med Studentersamfundets lover og
                i samråd med Arkivets medlemmer. Den nye instruksen er først gyldig når den er godkjent av Finansstyret.
        \end{enumerate}
    \end{instruksledd}

    \begin{instruksledd}{Formidling}
        \begin{enumerate}
            \item Hovedarkivar plikter å gjøre nye medlemmer i Arkivet kjent med denne instruks.
        \end{enumerate}
    \end{instruksledd}

\end{instruks}



