\begin{instruks}{Husorden for UKA-09}{12. september 2009}{12. september 2009}

Gjelder fra 13. september 2009 etter stengetid til og med 30. oktober 2009 kl 09.
Etter egen avtale skal UKErevyen disponere Storsalen og bakscenen etter samfundsmøtet 12.
september.


\begin{instruksledd}{Vedtak og gyldighet}
Husorden vedtas og eventuelt endres av Finansstyret etter forslag fra UKEstyret. Perioden
den gjelder fastsettes særskilt av Finansstyret. Husorden gjelder for alle lokaler UKA
disponerer, samt for personer som oppholder seg der.
\end{instruksledd}

\begin{instruksledd}{Romdisponering}
UKA disponerer Elgesetergate 1 (Samfundet-bygningen), Klæbuveien 1 (Trafo’n) og
brakker/telt på Fengselstomta. UKA disponerer alle lokaler i Samfundet-bygningen (inklusive
gjengområder ol), med unntak av lokaler som er utleid på kontrakt. De gjenger og
foreninger som ikke er UKEgjenger kan fortsatt disponere og fungere i sine lokaler, såfremt
disse ikke er rekvirert av UKA.
\end{instruksledd}

\begin{instruksledd}{Fullmakter og myndighet til UKA}
UKEstyret ved UKEsjef har myndighet til å fatte avgjørelser vedrørende UKAs drift. Dette i
samsvar med eventuelle rammer og vedtak fattet i Finansstyret. Studentersamfundets fast
ansatte vil i UKEperioden, samt der det for øvrig faller seg naturlig, samarbeide og jobbe for
UKA. UKA skal til gjengjeld refundere deler av deres lønn til driftregnskapet. Dette
spesifiseres nærmere i en egen avtale, der også de ansattes plikter overfor UKA kan
spesifiseres nærmere.
\end{instruksledd}

\begin{instruksledd}{Ro og orden samt brannvern}
I perioden UKAs husorden gjelder for Samfundet-bygningen skal UKEstyret sørge for at det
til enhver tid er en person med hovedansvar for ro og orden samt brannvern tilstede.
Vedkommende skal benevnes daghavende. For daghavende gjelder egen instruks. Ansvaret
for ro og orden samt brannvern vil under selve festivalen bli ivaretatt av Vertskapet. For
Vertskapet gjelder egne instrukser. Studentersamfundets branninstruks er gyldig til enhver
tid, og gjelder foran andre instrukser. Vertskapet under UKA skal kjenne
Studentersamfundets branninstruks i detalj. Vaktmester er til enhver tid øverste myndighet
når det gjelder ro og orden samt brannvern.
\end{instruksledd}

\begin{instruksledd}{Kontrollører}
Under UKA får UKEstyret, deres nestledere og vertskapsstyret utlevert Samfundets
kontrollørkort. UKEstyret, deres nestledere, vertskapsstyret samt Samfundets ordinære
kontrollører har kontrollørplikt under UKA i henhold til instruksen for kontrollører.
Samfundets ordinære kontrollører har ingen rettigheter i form av gratis inngang på
arrangementer, med mindre dette er avtalt spesifikt med UKEsjef eller daghavende. Kun
UKEsjef, UKAs daghavende, vertskapsstyrets daghavende, Samfundets vaktmester og SG-
daghavende (sistnevnte i skjenkesaker) har anledning til å inndra gjeng- og UKEkort under
UKA. Medlemmer av UKEstyret og vertskapsstyret har anledning til å inndra innslepp.
\end{instruksledd}

\begin{instruksledd}{Hybler}
UKEsjef, UKAs daghavende, vertskapsstyrets daghavende og Samfundets vaktmester har
myndighet til å tømme og stenge gjenghybler dersom det er nødvendig. Vanlig hybelinstruks
gjelder i perioden.
\end{instruksledd}

\begin{instruksledd}{Utøvelse av husorden}
Bemyndigede personer i henhold til denne husorden skal under utøvelse være skikket til å
ivareta husorden korrekt.
\end{instruksledd}

\begin{instruksledd}{Formidling}
UKEsjefen er ansvarlig for å formidle husorden til den det måtte angå. I praksis gjøres dette
gjennom UKAs salgssjef og vertskapssjef.
\end{instruksledd}

\begin{instruksleddd}{Annet}
Utstyr som eies av Samfundet skal være forsikret av Samfundet. Eventuelle bygningstekniske
ødeleggelser faller også under Samfundets forsikring. UKA er ansvarlig for tredjeparts utstyr,
inklusive privat utstyr. Daglig leder er også under UKA innehaver av skjenkebevilling.
\end{instruksledd}

\end{instruks*}
