
\begin{instruks}{Instruks for Profilgruppa}{29. januar 2008}{29. januar 2008}

    \begin{instruksledd}{Formål}
        \begin{enumerate}
            \item Profilgruppa skal ha det overordnede ansvaret for profilen på Samfundets utestedslokaler. Den skal bidra til å
gjøre Studentersamfundet i Trondhjem til et samlingssted for Studentersamfundets medlemmer, studenter og
for folk i byen, jfr. Formålsparagrafen. De skal tilby å videreutvikle et konkurransedyktig utelivstilbud til
Trondheims befolkning.
            \item  Profilgruppa skal sette opp en profilplan for alle utstedslokalene på Samfundet. Profilplanen skal utarbeides i
samarbeid med Nybygg, Samfundets byggekomité, UKAs Byggeprosjekt, Sikringskomiteen, Klubbstyret,
Lørdagskomiteen, Regi, Kulturutvalget, Diversegjengen, Forsterkerkomiteen, IT-komiteen, Kafégjengen og
Serveringsgjengen, samt den faste staben. Planen skal inneholde en langsiktig investeringsplan for inventar.
            \item Profilgruppa skal sette opp budsjett for alle sine prosjekter. I samarbeid med regnskapsfører skal gruppa se til
at det ikke benyttes mer enn det som er bevilget fra Finansstyret (FS).
            \item Profilgruppa skal i samarbeid med Layout Info Marked og de arrangerende gjengene se på muligheter for å
markedsføre de ulike lokalprofilene og konsepter i lokalene.
        \end{enumerate}
    \end{instruksledd}

    \begin{instruksledd}{Sammensetning}
        \begin{enumerate}
            \item Profilgruppa skal bestå av fem medlemmer. Styret har i tillegg rett til å ha ett medlem i gruppa.
            \item To av medlemmene skal rekrutteres internt på Samfundet, disse bør ha minimum ett år på Huset. To skal
fortrinnsvis rekrutteres eksternt fra relevant fagmiljø, og det siste medlemmet kommer fra FS.
            \item  FS skal ha en representant i Profilgruppa. Representanten fra FS skal være fast i ett år.
            \item Gruppa skal ha en leder og en økonomiansvarlig. Disse har en bindingstid på to år mens de øvrige sitter i ett
år. Alle medlemmer i Profilgruppa er funksjonærer. Leder og økonomiansvarlig har mulighet for å bli
Funksjonærpangsjonist etter endt bindingstid. Dette gjelder også andre medlemmer som sitter i to år.
            \item Gruppas leder innstiller selv medlemmene til FS for godkjenning. Bare medlemmer av Studentersamfundet
kan tas opp i Profilgruppa. Opptak utlyses i rimelig tid før opptaksfristen.
            \item Ved stemmelikhet avgjør leder.
        \end{enumerate}
    \end{instruksledd}

    \begin{instruksledd}{Ansvarsområder og plikter}
        \begin{enumerate}
            \item Medlemmer i Profilgruppa plikter å kjenne innholdet av og overholde følgende instrukser på Huset:
                \begin{enumerate}
                    \item Husorden
                    \item Branninstruks
                    \item Rømningsinstruks
                    \item Studentersamfundets lover
                \end{enumerate}
            \item Medlemmer av Profilgruppa disponerer kontorplass etter avtale med daglig leder.
            \item Leder plikter å møte for Finansstyret med oppdateringer om gruppas arbeid.
            \item Profilgruppa skal til enhver tid samarbeide med gjengene. Dette gjøres gjennom kontinuerlig oppdatering av
GSK om arbeidet, slik at GSK har anledning til å komme med innspill. Større endringer på lokaler skal
informeres om et semester før endringen utføres.
        \end{enumerate}
    \end{instruksledd}

    \begin{instruksledd}{Formidling}
        \begin{enumerate}
            \item Leder av Profilgruppa skal sørge for at medlemmene i gruppa kjenner innholdet i denne instruksen.
        \end{enumerate}
    \end{instruksledd}

    \begin{instruksledd}{Varighet}
        \begin{enumerate}
            \item Profilgruppa er et prøveprosjekt på to år, med årlig evaluering om å gjøre grupperingen permanent. I løpet av
prøveperioden skal det utarbeides en detaljert arbeidsinstruks.
        \end{enumerate}
    \end{instruksledd}
    
\end{instruks}
