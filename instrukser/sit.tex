
\begin{instruks}{Instruks for Studentersamfundets Interne Teater}{17. oktober 2010}{ }

    \begin{instruksledd}{Formål}
        \begin{enumerate}
            \item Studentersamfundets Interne Teater (SIT) er en politisk og religiøst uavhengig
                amatørteatergruppe tilknyttet Studentersamfundet i Trondhjem som kunstnerisk gjeng. SIT har som mål
                å fremme interesse for et vidt spekter av teaterformer, og å legge forholdene til rette for sine
                medlemmer for praktisk arbeid og skolering innenfor skuespillerteknikk, sang, scenografi-, kulisse-
                og kostymeutforming, regi/instruksjon, skriving og lesing av dramatikk, bl.a. med tanke på UKA.
        \end{enumerate}
    \end{instruksledd}

    \begin{instruksledd}{Sammensetning}
        \begin{enumerate}
            \item SIT består av ca. 70 medlemmer, hvorav 8 funksjonærer (styret) og resten
                gjengmedlemmer. Nye medlemmer tas opp én gang i semesteret. Opptak utlyses i rimelig tid før
                opptaksfristen. Bare medlemmer av Studentersamfundet i Trondhjem kan tas opp i SIT.
            \item Et medlem av SIT er aktivt i 2 år.
            \item Styret i SIT består av 8 medlemmer. Styret er SITs administrative organ og kan
                gjøre vedtak i alle saker. Teatersjefen fungerer som SITs daglige leder (gjengsjef), og som leder
                for styret. Medlemmer av styret må alltid ha styrevervet som sin primære oppgave i SIT (gjelder ikke
                i UKE-semester).
            \item SIT er delt i tre undergrupper; kostymegjengen, kulissegjengen og
                skuespillergjengen.
            \item  Generalforsamlingen er SITs øverste organ. Alle tilstedeværende medlemmer
                av SIT har stemmerett på generalforsamlingen.
        \end{enumerate}
    \end{instruksledd}

    \begin{instruksledd}{Ansvarsområde og plikter}
        \begin{enumerate}   
            \item  Medlemmer i SIT plikter å kjenne innholdet av og overholde følgende instrukser på
                Huset:
                \begin{enumerate}
                    \item Generell gjenginstruks.
                \end{enumerate}
            \item Studentersamfundets Interne Teater disponerer og har ansvaret for følgende rom med
                tilhørende utstyr:
                \begin{enumerate}
                    \item Hybelen
                    \item Kostymen
                    \item Lille Øvre
                    \item Kostymelager i rotunden ved Edgar
                    \item Kulisselager i 4. kvadrant
                    \item Rekvisittbua innenfor Store Øvre
                \end{enumerate}
        \end{enumerate}
    \end{instruksledd}

    \begin{instruksledd}{Økonomi og regnskap}
        \begin{enumerate}
            \item Generalforsamlingen er SITs øverste organ. Alle tilstedeværende
                medlemmer av SIT har stemmerett pågeneralforsamlingen. 
            \item Styret skal i forbindelse med aktivitetsplanen legge frem et
                internt oversiktsbudsjett for SITs aktiviteter.
            \item Store avvik fra budsjett og instruks skal rapporteres til
                Gjengsekretariatet.
        \end{enumerate}  
    \end{instruksledd}

    \begin{instruksledd} Endring av instruksen
        \begin{enumerate}
            \item Endringer av denne instruksen skal skje i samsvar med
                Studentersamfundets lover og i samråd med gjengens medlemmer. Alle endringer skal forelegges
                Finansstyret til uttalelse og godkjennelse.		
        \end{enumerate}  
    \end{instruksledd}

    \begin{instruksledd}{Formidling}
        \begin{enumerate}
            \item Teatersjefen (Gjengsjefen) plikter å gjøre nye medlemmer av SIT kjent med
                denne instruks.
        \end{enumerate}
    \end{instruksledd}

    \begin{instruksledd}{Kompetanseoverføring}
        \begin{enumerate}
            \item Styret har ansvaret for at det blir foretatt kompetanseoverføring til det
                nyvalgte styret etter generalforsamlingen. Kompetanseoverføringen bør inneholde de forskjellige
                styrevervenes ansvarsområder, oppgaver og utfordringer, og utføres ved skriftlig erfaringsskriv og
                muntlig overføring.
            \item Personer som besetter produksjonsapparatstillinger i SITs produksjoner bør
                skrive et erfaringsskriv som er tilgjengelig for senere stillingstakere, samt være tilgjengelige for
                spørsmål utover dette skrivet	
        \end{enumerate}
    \end{instruksledd}


\end{instruks}


