
\begin{instruks}{Instruks for Studentersamfundets Interne Teater}{17. oktober 2010}{ }

    \begin{instruksledd}{Form�l}
        \begin{enumerate}
            \item Studentersamfundets Interne Teater (SIT) er en politisk og religi�st uavhengig amat�rteatergruppe tilknyttet Studentersamfundet i Trondhjem som kunstnerisk gjeng. SIT har som m�l � fremme interesse for et vidt spekter av teaterformer, og � legge forholdene til rette for sine medlemmer for praktisk arbeid og skolering innenfor skuespillerteknikk, sang, scenografi-, kulisse- og kostymeutforming, regi/instruksjon, skriving og lesing av dramatikk, bl.a. med tanke p� UKA.
                    \end{enumerate}
    \end{instruksledd}

    \begin{instruksledd}{Sammensetning}
        \begin{enumerate}
            \item SIT best�r av ca. 70 medlemmer, hvorav 8 funksjon�rer (styret) og resten gjengmedlemmer. Nye medlemmer tas opp �n gang i semesteret. Opptak utlyses i rimelig tid f�r opptaksfristen. Bare medlemmer av Studentersamfundet i Trondhjem kan tas opp i SIT.
            \item Et medlem av SIT er aktivt i 2 �r.
            \item Styret i SIT best�r av 8 medlemmer. Styret er SITs administrative organ og kan gj�re vedtak i alle saker. Teatersjefen fungerer som SITs daglige leder (gjengsjef), og som leder for styret. Medlemmer av styret m� alltid ha styrevervet som sin prim�re oppgave i SIT (gjelder ikke i UKE-semester).
		\item SIT er delt i tre undergrupper; kostymegjengen, kulissegjengen og skuespillergjengen.
		\item  Generalforsamlingen er SITs �verste organ. Alle tilstedev�rende medlemmer av SIT har stemmerett p� generalforsamlingen.
        \end{enumerate}
     \end {instruksledd}
	\begin{instruksledd}{Ansvarsomr�de og plikter}
        \begin{enumerate}   
            \item  Medlemmer i SIT plikter � kjenne innholdet av og overholde f�lgende instrukser p� Huset:
                \begin{enumerate}
                    \item Generell gjenginstruks.
                \end{enumerate}
            \item Studentersamfundets Interne Teater disponerer og har ansvaret for f�lgende rom med tilh�rende utstyr:
                \begin{enumerate}
                		\item Hybelen
				\item Kostymen
				\item Lille �vre
				\item Kostymelager i rotunden ved Edgar
				\item Kulisselager i 4. kvadrant
				\item Rekvisittbua innenfor Store �vre
                \end{enumerate}
        \end{enumerate}

	  \begin{instruksledd}{�konomi og regnskap}
		\begin{enumerate}
			\item Generalforsamlingen er SITs �verste organ. Alle tilstedev�rende medlemmer av SIT har stemmerett p�generalforsamlingen. 
			\item Styret skal i forbindelse med aktivitetsplanen legge frem et internt oversiktsbudsjett for SITs aktiviteter.
			\item Store avvik fra budsjett og instruks skal rapporteres til Gjengsekretariatet.
		\end{enumerate}  
	   \end{instruksledd}
	  \begin{instruksledd} Endring av instruksen
		\begin{enumerate}
			\item Endringer av denne instruksen skal skje i samsvar med Studentersamfundets lover og i samr�d med gjengens medlemmer. Alle endringer skal forelegges Finansstyret til uttalelse og godkjennelse.		
      	\end{enumerate}  
	\begin{instruksledd}{Formidling}
            \begin{enumerate}
                \item Teatersjefen (Gjengsjefen) plikter � gj�re nye medlemmer av SIT kjent med denne instruks.
            \end{enumerate}
        \end{instruksledd}
\begin{instruksledd}{Kompetanseoverf�ring}
            \begin{enumerate}
                \item Styret har ansvaret for at det blir foretatt kompetanseoverf�ring til det nyvalgte styret etter generalforsamlingen. Kompetanseoverf�ringen b�r inneholde de forskjellige styrevervenes ansvarsomr�der, oppgaver og utfordringer, og utf�res ved skriftlig erfaringsskriv og muntlig overf�ring.
                \item Personer som besetter produksjonsapparatstillinger i SITs produksjoner b�r skrive et erfaringsskriv som er tilgjengelig for senere stillingstakere, samt v�re tilgjengelige for sp�rsm�l utover dette skrivet	
		\end{enumerate}
        \end{instruksledd}


\end{instruks}


