
\begin{instruks}{Instruks for Studentersamfundets Orkester}{ }{ }

    \begin{instruksledd}{Navn og formål}
        \begin{enumerate}
            \item Orkesterets navn er ``Studentersamfundets Orkester'', heretter kalt Orkesteret.
                Orkesteret består av et antall
                orkestre og band, heretter kalt grupper, og har 150-200 medlemmer.
            \item Orkesterets formål er å legge til rette for at flest mulig studenter kan bedrive aktiv
                musikkutøvelse i et bredt
                og inkluderende miljø.
        \end{enumerate}
    \end{instruksledd}

    \begin{instruksledd}{Sammensetning}
        \begin{enumerate}
            \item \textbf{Faste grupper}

                Orkesterets faste grupper skal være av en permanent art. Med permanent menes i denne
                instruksen at en
                gruppe viser evne og vilje til å eksistere over en lengre periode, og greier å drive aktiv
                rekruttering av nye
                medlemmer. De faste grupperingene, bortsett fra Snaustrinda Spelemannslag, Salongorkesteret og
                Leisure
                Suit Lovers er regnskapspliktige ovenfor Orkesterets generalforsamling. De regnskapspliktige
                grupperingene
                har rett til funksjonærstatus til leder og kasserer. Studentersamfundets Symfoniorkester har i
                tillegg rett til
                funksjonærstatus for sin sekretær og materialforvalter.

            \item \textbf{Uavhengige grupper}

                Orkesterets uavhengige grupper skal ha medlemskap som er av en permanent art. De er ikke
                regnskapspliktige og har heller ikke rett til funksjonærstillinger.

            \item \textbf{Midlertidige grupper}

                Orkesterets midlertidige grupper skal ikke være av en permanent art. De er ikke
                regnskapspliktige og har
                ikke rett til funksjonærstillinger.

            \item Opptak av nye grupper må godkjennes av Generalforsamlingen, etter innstilling fra
                Studentersamfundets
                Orkesters Styre, heretter kalt Orkesterstyret.

            \item Midlertidige grupper skal vurderes for fortsatt medlemskap ved hver ordinære
                generalforsamling ut i fra
                følgende retningslinjer:
                \begin{enumerate}
                    \item Om gruppen fortsatt er i drift. Dette må dokumenteres av leder for gruppen ved
                        siste styremøte før
                        Generalforsamlingen slik at Orkesterstyret kan innstille overfor Generalforsamlingen.
                    \item Om gruppen kan endre status til uavhengig eller fast gruppe. Dette kan
                        gjennomføres hvis gruppen
                        gjennom flere år har vært av permanent art og følger opp sine plikter på linje med de andre
                        gruppene.
                \end{enumerate}
            \item Alle medlemmer av Orkesteret har de samme plikter og rettigheter.
        \end{enumerate}
    \end{instruksledd}

    \pagebreak
    \begin{instruksledd}{Orkesterets medlemmer}
        \begin{enumerate}
            \item Medlemskap i Studentersamfundet i Trondhjem er en betingelse for medlemskap i
                Orkesteret. Unntak kan
                gjøres for interesserte musikere og andre som ikke har adgang til medlemskap i Studentersamfundet i
                Trondhjem. Disse kan under forutsetning av godkjennelse fra Styret i Studentersamfundet i Trondhjem
                få
                innvilget adgang til Studentersamfundet for ett semester av gangen.
            \item Spillende studenter som ønsker medlemskap i Orkesteret men som ikke kan gis fast
                tilknytning til en
                eksisterende gruppe kan søke Orkesterets styre om opptak. Slikt opptak kan skje etter skriftlig
                søknad og
                opptaksprøve. Utover dette fastsetter Orkesterstyret retningslinjer for opptaket. Orkesterstyrets
                vedtak kan
                overprøves av Generalforsamlingen.
        \end{enumerate}
    \end{instruksledd}

    \begin{instruksledd}{Orkesterets plikter}
        \begin{enumerate}
            \item Alle Orkesterets medlemmer må inneha medlemskap i Studentersamfundet i Trondhjem.
            \item Medlemmene har plikt til å møte presis til alle øvelser og oppdrag. Forfall må meldes
                styret i de respektive
                grupper i rimelig tid.
            \item Medlemmene er erstatningspliktig ansvarlig for noter og alt materiell som eies eller
                disponeres av Orkesteret.
                Erstatningsplikten gjelder også privat eiendom som oppbevares i Orkesterets lokaler, heretter kalt
                Musikerlåfte'.
            \item Medlemmene skal rette seg etter de instrukser som til enhver tid gjelder for bruken av
                Musikerlåfte'.
                Medlemmene plikter å bidra til rydding og renhold av Musikerlåfte'. Medlemmene kan bli pålagt ulike
                arbeidsoppgaver av Orkesterstyret.
            \item Medlemmene plikter å holde seg orientert gjennom oppslag, beskjeder og liknende.
            \item Medlemmene plikter å betale den fastsatte kontingent til sine respektive
                medlemsorkestre til rett tid.
            \item Medlemmene blir pangsjonister etter seks semestres aktiv tjeneste i Orkesteret.
                Pangsjonistene blir tildelt
                Orkesterets hedersbevisning, G-nøkkelen. Medlemmer som har gjort en spesiell innsats for Orkesteret
                kan
                også tildeles Orkesterets orden, Polyhymnia. Hvem som tildeles Polyhymnia bestemmes av
                Ordenskapittelet,
                etter innstilling fra Orkesterstyret.
        \end{enumerate}
    \end{instruksledd}

    \begin{instruksledd}{Styret av Studentersamfundets Orkester}
        \begin{enumerate}
            \item Styret av Studentersamfundets Orkesteret består av 14 funksjonærer og har følgende
                sammensetning:

                Tale og stemmeberettigede:
                \begin{itemize}
                    \item Med møteplikt: Orkestersjef (gjengsjef), Nestleder, Kasserer, Materialforvalter og
                        Intendant
                    \item Med møterett: leder, eller den leder bemyndiger, fra hver faste gruppering.
                \end{itemize}

                Observatører med talerett, men uten stemmerett:
                \begin{itemize}
                    \item Med møterett: kontaktperson for midlertidige grupperinger godkjent av styret, samt
                        Orkesterets
                        funksjonærer og dirigenter.
                \end{itemize}

                Orkesterstyrets sammensetning skal så langt som mulig representere medlemmene av Orkesteret,
                med
                særskilt hensyn på grupperingenes størrelse, medlemmenes kjønn og etniske bakgrunn/nasjonalitet.
                Orkesterstyret ledes av Orkestersjefen, som er gjengsjef. Orkestersjefen velges av Generalforsamling
                og må
                godkjennes av Storsalen.

            \item \textbf{Stemmefordeling}

                Ved dissens fordeles stemmene etter denne nøkkelen:

                \begin{itemize}
                    \item Orkestersjef: 2 stemmer
                    \item Orkesterets Nestleder, Kasserer, Materialforvalter og Intendant: Henholdsvis 1
                        stemme hver
                    \item Representant fra faste grupperinger: 1 stemme per 20 registrerte/faste medlemmer.
                    \item Representanter fra grupperinger som ikke har 20 registrerte/faste medlemmer
                        tildeles en stemme.
                \end{itemize}

            \item \textbf{Orkesterstyrets funksjonstid}

                Orkestersjef og Orkesterets Kasserer, Materialforvalter og Intendant velges av Generalforsamling for
                ett år.
                Orkesterets Nestleder velges av styret og godkjennes av påfølgende Generalforsamling. Et styremedlem
                kan
                av de andre medlemmene av Orkesterstyret innvilges permisjon eller fritas fra sitt styreverv.
                Musikerlåfte\'.

            \item \textbf{Orkesterstyrets plikter}

                Orkesterstyret skal representere Orkesteret og sørge for at Orkesterets rettigheter og plikter blir
                ivaretatt.

                Orkesterstyret skal se til at de enkelte gruppene drives på en sunn måte. Dette innebærer oppfølging
                av blant
                annet de enkelte gruppenes semesterplaner og økonomi.

                Orkestersjefen, eller den Orkestersjefen bemyndiger, skal innkalle til styremøte når det er
                nødvendig,
                minimum en gang per uke. Det skal også avholdes styremøte dersom minimum tre styremedlemmer krever
                det og kaller inn resten av styret på ordinær måte. I månedene juni, juli, august og desember holdes
                det
                normalt ikke styremøter.

                Ved grov pliktforsømmelse kan styremedlemmer avsettes ved ekstraordinær generalforsamling.

            \item \textbf{Styremedlemmenes ansvarsområde}

                Det skal til enhver tid foreligge instrukser for Orkesterstyret. Det enkelte styremedlem får der
                sitt
                ansvarsområde definert. Orkestersjefen skal påse at instruksene blir fulgt.

            \item \textbf{Disiplinærmyndighet}

                Orkesterstyret har myndighet til å idømme medlemmer av Orkesteret straff for brudd på Orkesterets
                instruks,
                illojal opptreden overfor Orkesteret eller andre forseelser som kun angår Orkesteret. Straffen kan
                være
                utvisning for kortere eller lengre tid eller annen passende straff. I disiplinærsaker hvor det har
                forekommet
                brudd på Samfundets lover, er Styret ved Studentersamfundet i Trondhjem disiplinærmyndighet.

        \end{enumerate}
    \end{instruksledd}
    \pagebreak
    
    \begin{instruksledd}{Orkesteret og Studentersamfundet}
        \begin{enumerate}
            \item \textbf{Rettigheter}

                Orkesteret disponerer Musikerlåfte' under forutsetning av at instruksen for bruk av Musikerlåfte'
                blir fulgt.

                Studentersamfundet bevilger hvert år tilskudd til Orkesterets drift og kan bevilge midler til
                Orkesterets
                investeringer. Orkesteret og de faste gruppene kan også søke om ekstrabevilgninger fra Finansstyret
                for
                spesielle utgiftsposter. Orkesteret får utbetalt eventuelle godtgjørelser for utførte tjenester i
                Huset etter avtale
                i hvert enkelt tilfelle.

                Orkesteret kan etter avtale med Kontrollkontoret legge beslag på lokaler i Huset når dette er
                nødvendig for
                Orkesterets arbeid. Arbeidet skal utføres på en slik måte at det, så fremt det er mulig, ikke er til
                ulempe for
                annen utleie eller andre tilstelninger på Huset.

                Medlemmene av Orkesterstyret som har både tale- og stemmerett er funksjonærer ved Studentersamfundet
                i Trondhjem.

            \item \textbf{Plikter}

                Orkesteret plikter:
                \begin{enumerate}
                    \item Å bidra til den musikalske underholdningen ved Studentersamfundet, så som egne
                        konserter, spilling til
                        dans, musikkmøter, kunstnerisk osv.
                    \item Å bidra til å markere Studentersamfundet som en kulturinstitusjon i Trondheim.
                    \item Å stille revyorkester til UKA.
                    \item Orkesteret skal føre nøyaktig inventaroversikt. Ansvarlig for dette er
                        Materialforvalteren.
                \end{enumerate}

            \item \textbf{Regnskap og budsjett}

                Orkesteret plikter å levere budsjett og regnskap til Gjengsekretariatet og Finansstyret årlig og i
                rett tid.
                Regnskapsperioden følger kalenderåret og regnskapet blir revidert av Studentersamfundets revisor.
                Orkesteret får hvert år et tilskudd fra Finansstyret. Disse pengene fordeles på grupperingene i
                Orkesteret etter
                en foredelingsnøkkel utarbeidet av Orkesterstyret. Orkesteret får i tillegg forpleiningspenger i
                henhold til
                antall funksjonærer.

                De faste gruppene plikter å føre og levere regnskap, dette gjelder ikke Snaustrinda
                Spelemannslag, Salongorkesteret og Leisure Suit Lovers. Regnskapet revideres av minst èn annen
                kasserer i
                Orkesteret, og godkjennes av Generalforsamlingen etter innstilling fra Styret. Hvis ønskelig kan de
                faste
                gruppene benytte Studentersamfundets revisor for ytterligere revisjon.

                Orkesteret plikter å levere budsjett- og revidert budsjettforslag til Gjengsekretariatet og
                Finansstyret årlig.

                De faste gruppene plikter å levere budsjettforslag for drift og investeringer årlig, og så tidlig at
                Orkesterstyret
                kan ta stilling til budsjettforslaget før budsjettårets start. Budsjettene skal godkjennes av
                Orkesterstyret og
                endelig vedtas av Generalforsamlingen.

                Dersom ordinære bevilgninger til Orkesteret over Studentersamfundets budsjett ikke er tilstrekkelig
                til
                forsvarlig drift og vedlikehold, plikter lederne i de enkelte gruppene å fremføre dette for
                Orkesterets styre.

                De uavhengige og de midlertidige gruppene har ikke regnskapsplikt.
        \end{enumerate}
 \end{instruksledd}

 \begin{instruksledd}{Generalforsamlingen ved Studentersamfundes Orkester}
     \begin{enumerate}
         \item \textbf{Myndighet}

             Orkesteret samlet til generalforsamling representerer Orkesterets høyeste myndighet.

         \item \textbf{Innkalling}

             Innkalling til generalforsamling skal finne sted ved kunngjøring i de enkelte gruppene via
             egnede medier og
             på Musikerlåfte' minst en uke i forveien. Oppslaget skal inneholde dagsorden, sakspapirer og
             undertegnes av
             gjengsjefen på vegne av Orkesterstyret, eller minst to styremedlemmer.

         \item \textbf{Valg av styre}

             Generalforsamlingen velger styrets fellesverv Orkestersjef, Kasserer, Materialforvalter og Intendant
             og skal
             godkjenne de faste gruppenes valg av styremedlemmer. Valget skjer ved simpelt flertall. Ved
             stemmelikhet
             gjennomføres omvalg. Ved fortsatt stemmelikhet har Orkestersjefen dobbeltstemme. Ved valg av nytt
             styre
             skal hver enkelt gruppe legge fram årsrapport, regnskap og budsjett. Styrets sammensetning skal
             meldes
             Gjengsekretariatet, styret av Studentersamfundet i Trondhjem og Kontrollkontoret så snart som mulig.

         \item \textbf{Ekstraordinær generalforsamling}

             Ekstraordinær generalforsamling kan holdes når spesielle forhold gjør dette nødvendig.
             Orkesterstyret eller
             minst halvparten av Orkesterets medlemmer kan innkalle til ekstraordinær generalforsamling. Ved
             ekstraordinær generalforsamling kan bare de saker behandles som har foranlediget innkallingen. Det
             gjelder
             de samme voteringsregler som ved ordinær generalforsamling, likeledes gjelder de samme regler for
             innkalling.

         \item \textbf{Instruksen}

             Innkalling til generalforsamling skal finne sted ved kunngjøring i de enkelte gruppene via
             egnede medier og
             på Musikerlåfte' minst en uke i forveien. Oppslaget skal inneholde dagsorden, sakspapirer og
             undertegnes av
             gjengsjefen på vegne av Orkesterstyret, eller minst to styremedlemmer.
         \item \textbf{Inkalling}

             Forandring av Hovedinstruksen skal skje i samsvar med Studentersamfundets lover og må vedtas av
             ordinær
             eller ekstraordinær generalforsamling med 2/3 flertall.

             Det skal til enhver tid foreligge interne instrukser for styrets medlemmer og bruken av
             Musikerlåfte'.
             Forandring av interne instrukser skjer etter samme retningslinjer som for Hovedinstruksen.

             Lederne i de enkelte gruppene plikter å gjøre nye medlemmer i gruppen kjent med instruksene.
     \end{enumerate}
 \end{instruksledd}

\end{instruks}


