\begin{instruks}{Instruks for Sikringskomiteen}{}{}
    \begin{instruksledd}{Formål}
        \begin{enumerate}
            \item  Sikringskomiteen er opprettet av Finansstyret. Sikringskomiteen skal
                være rådgivende og utøvende organ
                innen brann og sikringsarbeid på Samfundet. Sikringskomiteen har ansvar
                for materiell og rutiner for person-
                og verdisikring på Huset, herunder ansvar for nøkler og adgangskontroll.
            \item Sikringskomiteen er engasjert av daglig leder, og arbeider med oppdrag
                gitt av denne, samt behov avdekket
                på egenhånd.
        \end{enumerate}
    \end{instruksledd}

    \begin{instruksledd}{Sammensetning}
        \begin{enumerate}
            \item Sikringsvervet bør bestå av 2 aktive medlemmer. Dette antallet kan etter behov
                endres. Dette bestemmes
                direkte av daglig leder, etter forespørsel fra Sikringssjefen ved hvert enkelt
                tilfelle.
            \item Medlemmene av Sikringskomiteen bør være erfarne Husfolk, og ha en god oversikt over
                Studentersamfundets lokaler samt Husets drift. Medlemmene skal jobbe nøytralt, og
                likestille alle Husets
                gjenger/områder og andre interessenter.
            \item Nye medlemmer tas opp av daglig leder. Sittende Sikringssjef lyser ut og intervjuer
                for stillingene. Bare
                nåværende eller tidligere medlemmer av noen av Studentersamfundets gjenger kan taes
                opp som sikringssjef.
            \item Medlemmer av sikringskomiteen skal være engasjert i 3 semestre. Opptak av nye
                medlemmer skjer en gang i
                året for å skape overlapp. Ved utskifting av medlemmene skal Sikringssjefen sørge for
                overlapping, slik at
                kompetansen bevares, og at det sikres en kontinuerlig drift. Sikringssjefen kan ved
                behov frita et
                gruppemedlem fra sitt verv. Permisjon kan ikke innvilges i dette engasjementet.
        \end{enumerate}
    \end{instruksledd}

    \begin{instruksledd}{Ansvarsområde og plikter}
        \begin{enumerate}
            \item Medlemmer i Sikringskomiteen plikter å kjenne innholdet av og overholde
                følgende instrukser på Huset:
                \begin{enumerate}
                    \item  Husorden
                    \item Branninstruks
                    \item Rømningsinstruks
                    \item Kontrollørinstruks
                    \item Instruks for Vedlikeholdsgruppa
                    \item Sikringsvervets instruks
                    \item Studentersamfundets lover.
                \end{enumerate}
            \item Sikringssjefen har møterett i Studentersamfundets byggekomite og
                Vedlikeholdsgruppa.
            \item Sikringskomiteen skal jobbe for daglig leder med alle oppgaver knyttet
                til sikringsarbeid på Huset.
                Sikringskomiteen skal gjennomføre nødvendige tiltak for å gi gjester,
                gjengmedlemmer og ansatte
                tilstrekkelig sikkerhet, og som også tar vare på Husets materielle
                verdier.
            \item Sikringssjefen har ansvaret for å holde instrukser for brann og
                brannforebygging, evakuering og
                adgangskontroll à jour. Endringer i instrukser skal godkjennes av
                Finansstyret.
            \item Sikringskomiteen skal påse at Husmann og LK leder, har nødvendig
                opplæring på brannsentralen,
                evakueringsrutiner, samt Husets sikkerhetsrutiner.
            \item Sikringskomiteen skal samarbeide nært med Husmann. Husmannen skal ivareta
                intern opplæring, mens
                sikringskomiteen tar seg av alle eksterne kontakter. Husmannen skal
                formidle lover og regler gitt av
                sikringsgruppa til resten av Husets gjengmedlemmer. Sikringssjefen
                samarbeider med Husmann ved brannøvelser.
            \item Sikringskomiteen har ansvar for kontakt med brannvesen.
            \item Sikringskomiteen skal påse at det på private arealer foreligger
                tilstrekkelig med brannutstyr og sørge for at
                hyblene følger normale normer i forhold til generell brannsikkerhet.
            \item Sikringskomiteen skal/må jobbe i samarbeid med vaktmester med tanke på
                erfaringer, vedlikehold osv.
            \item Ved utleie av Huset skal sikringssjefen, i samarbeid med daglig leder,
                påse at arrangør er kjent med
                retningslinjer for brannsikkerhet og evakuering av Huset.
            \item Sikringssjefen kan av hensyn til sikkerheten til besøkende,
                gjengmedlemmer eller ansatte stenge
                arrangementer i samråd med Daglig leder eller Finansstyrets leder.
            \item Sikringskomiteen skal forvalte Husets nøkler, nøkkelkort og låsmateriell.
                Vedlikehold av dører gjøres av
                vaktmester og Diverse-gjengen.
            \item Sikringskomiteen disponerer kontorplass ved heisrommet
                (gjenganretningen).
            \item Sikringskomiteen disponerer nøkler og nøkkelkort etter nøkkelinstruks,
                og har i forbindelse med
                sikringsarbeid adgang til alle rom i Huset, også gjengarealer.
            \item Sikringskomiteen skal forvalte Husets videoovervåkningsutstyr, og sørge
                for at dette vedlikeholdes.
                Sikringsgruppa plikter videre å påse at informasjon fra anlegget ikke
                misbrukes, og at regler for personvern i
                forbindelse med dette overholdes.
            \item Sikringskomiteen plikter å utføre arbeid i henhold til denne instruks i
                tidsrommene primo august til medio
                desember og medio januar til medio mai. Dato for arbeid og andre
                tidsfrister settes av sikringssjefen.
            \item Sikringskomiteen kan tildele midlertidig funksjonærstatus til personer
                som gjør stor innsats for midlertidige
                prosjekter. Dette er begrenset oppad til 1 år.
            \item Sikringssjefen har ansvar for å dokumentere sikringskomiteen sin
                virksomhet.
        \end{enumerate}
    \end{instruksledd}

    \begin{instruksledd}{Godtgjørelse og rettigheter}
        \begin{enumerate}
            \item Sikringskomiteens aktive medlemmer, studerende pensjonister og pensjonister som
                avtjener førstegangstjeneste eller siviltjeneste har rett til funksjonærkort.
        \end{enumerate}
    \end{instruksledd}

    \begin{instruksledd}{Økonomi og regnskap}
        \begin{enumerate}
            \item Sikringskomiteen tildeles forpleiningspenger på samme grunnlag som andre aktive
                funksjonærer.
            \item Sikringskomiteen benytter ellers driftsbudsjettet etter løpende avtaler med
                Daglig Leder.
        \end{enumerate}
    \end{instruksledd}

    \begin{instruksledd}{ Endringer av instruksen}
        \begin{enumerate}
            \item Endringer av denne instruksen skal skje i samsvar med Studentersamfundets lover
                og i samråd med
                Sikringskomiteens medlemmer. Alle endringer skal godkjennes av Finansstyret før
                den er gjeldende.
        \end{enumerate}
    \end{instruksledd}

    \begin{instruksledd}{Formidling}
        \begin{enumerate}
            \item Sikringssjefen plikter å gjøre nye medlemmer i Sikringskomiteen kjent med denne
                instruks.
        \end{enumerate}
    \end{instruksledd}

\end{instruks}





