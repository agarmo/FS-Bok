\begin{instruks}{Husorden for UKA-11}{12. august 2011}{16. august 2011}



\begin{instruksledd}{Omfang og varighet}
Husorden for UKA-11 (Husorden) gjelder isteden for gjeldende ”Husorden for
Studentersamfundet i Trondheim”.\\

Husorden vedtas og endres av Finansstyret etter forslag fra UKEstyret. Husorden
gjelder for alle lokaler UKA disponerer, samt for personer som oppholder seg der.
Styret, FS-AU, GS, og Sikringskomiteen har uhindret tilgang til huset i embeds medfør.\\

Husorden gjelder for alle gjengmedlemmer og funksjonærer ved
Studentersamfundet i Trondhjem, enten de er tatt opp av UKA eller ikke. UKA og
dens funksjonærer plikter å påse at gjester også overholder Husorden.\\

Husorden gjelder fra 18. september 2011 kl. 03:00 til og med 4. november kl
09:00, dog slik at UKA disponerer Storsalen og bakscenen fra Samfundsmøtets
slutt, lørdag 17. september.\\

Bestemmelsene i Husorden gjelder også for arealer utenfor Samfundet, som
Fengselstomta og Dødens dal.
\end{instruksledd}

\begin{instruksledd}{Romdisponering, åpningstid mm}
UKA disponerer Elgesetergate 1 (Samfundsbygningen), Klæbuveien 1 (Trafo’n)
og brakker/telt på Fengselstomta. Plassering av brakker og telt anvises av
Daglig leder.\\

Under Husorden råder UKA over alle offentlige lokaler, gjengområder i
Samfundsbygningen, og Trafoen. Unntatt fra UKA sin rådighet er tekniske rom,
arealer som disponeres av Samfundets administrasjon og ansatte, samt lokaler
omfattet av Sesam AS leieavtale.\\

Arealer som disponeres eksklusivt av gjenger og foreninger, disponeres av UKA i
ukeperioden, uavhengig av vedtatte gjenginstrukser.\\

Sikringssjef leverer ut nøkler etter fordeling fastsatt av UKA. Bruk av nøkler skal
skje i henhold til Samfundets ordinære nøkkelinstruks.\\

De gjenger og foreninger som er tatt opp som UKEgjenger er under UKA
underlagt UKEstyret.\\

Samfundets åpningstider er fastsatt av Finansstyret. For enkelt arrangementer
kan annen åpningstid avtales med Daglig leder.\\

Vaktsjef er ansvarlig for åpning og stenging av Huset. Ved stengetid skal alle
aktiviteter i offentlige lokaler opphøre og lokalene tømmes. Vaktsjef skal sørge
for utlåsing ved stengetid. Kun de personer UKA har gitt adgang til gjengenes
hybler, kan være i Huset etter stengetid.
\end{instruksledd}

\begin{instruksledd}{Fullmakter og myndighet til UKA}
UKEstyret ved UKEsjef fatter avgjørelser vedrørende UKAs drift i samsvar med
eventuelle rammer og vedtak fattet i Finansstyret.\\

Studentersamfundets fast ansatte vil i UKEperioden, samt der det for øvrig faller
seg naturlig, samarbeide og jobbe for UKA. Samfundet kan kreve refundert deler
av deres lønn fra UKA. UKA svarer for alle kostnader til renhold, samt vakthold
gjennom Samfundet vaktselskap.
\end{instruksledd}

\begin{instruksledd}{Ro, orden, sikkerhet og brannvern}
Studentersamfundets branninstruks og øvrige bestemmelser som HMS står
uendret gjennom UKA, og gjelder foran alle andre instrukser.\\

Daglig leder, Samfundets Vaktsjef, og Brannvernleder er til enhver tid øverste
myndighet når det gjelder ro, orden, sikkerhet og brannvern. Pålegg fra Daglig
leder, Samfundets Vaktsjef, og Brannvernleder kan ikke overprøves. UKA plikter
å følge de påbud Brannvernleder og Sikringssjefen fastsetter for brann og
sikringsarbeid.\\

Daglig leder, Samfundets Vaktsjef, og Brannvernleder har ansvar for publikums
sikkerhet under UKA. UKAs egne vakter innordnes Vaktsjef. Vaktsjef og den han
bemyndiger har ansvar for bortvisning av gjester fra offentlige arealer. UKA kan
bortvise egne medarbeidere.\\

I perioden Husorden gjelder for Samfundsbygningen skal UKEstyret sørge for at
det til enhver tid er en Daghavende med hovedansvar for ro og orden samt
brannvern tilstede. For Daghavende gjelder egen instruks som forelegges daglig
leder for godkjenning.\\

Det er UKEstyrets ansvar å opprettholde ro og orden, sikkerhet og brannvern
under den periode Husorden gjelder. Pliktene kan overlates UKAs Vertskap.
Innleid vakthold underlegges Vaktsjefen.\\

For Vertskapet gjelder egne instrukser som forelegges daglig leder for
godkjenning. Vertskapet under UKA skal kjenne Studentersamfundets
branninstruks og HMS regelverk i detalj.
\end{instruksledd}

\begin{instruksledd}{Avvik og rapportering ved avvik}
Alle avvik rapporteres daglig til Daglig leder og Vaktsjef. Større avvik hvor det
kan være fare for betydelig skade for person- eller bygningssikkerhet,
Samfundets renommé eller økonomi skal rapporteres umiddelbart.\\

Alle skjenkekontroller og avvik ved serveringsvirksomheten skal straks
rapporteres skjenkebestyrer.
\end{instruksledd}

\begin{instruksledd}{Beredskaps- og kriseplan}
UKA plikter å utarbeide egen beredskaps- og kriseplan med klare handlings- og
varslingsrutiner. Planen skal foreligges daglig leder, vaktsjef og brannvernleder
for godkjenning.
\end{instruksledd}

\begin{instruksledd}{Servering}
Skjenkebestyrer er innehaver av Samfundets skjenkebevilling, også under UKA.
UKA plikter til enhver tid å ha en daghavende servering med kunnskapsprøve.
UKA plikter å ha egen daghavende servering i Dødens dal.\\

UKA plikter å følge de retningslinjer og påbud skjenkebestyrer fastsetter for
serveringsvirksomheten. Pålegg fra skjenkebestyrer, stedfortreder og
daghavende servering kan ikke overprøves.
\end{instruksledd}

\begin{instruksledd}{Kontrollører}
Under UKA får UKEstyret, deres nestledere og UKAs Vertskapsstyre utlevert
Samfundets kontrollørkort. UKEstyret, deres nestledere, Vertskapsstyret samt
Samfundets ordinære kontrollører har kontrollørplikt under UKA i henhold til
instruksen for kontrollører.\\

UKEsjef, Daghavende, Vertskapsstyrets daghavende, Husmann, Daglig leder,
Samfundets vaktsjef eller stedfortreder og daghavende servering har anledning
til å inndra medlems-, gjeng- og UKEkort under UKA, og myndighet til å tømme
og stenge gjenghybler dersom de finner det nødvendig.\\

Husmann opprettholder sin virksomhet og myndighet under UKA, og er
overordnet UKAs funksjonærer. Inndratte gjengkort leveres Husmann for videre
oppfølgning i henhold til Samfundets instrukser.\\

Medlemmer av UKEstyret og Vertskapsstyret har anledning til å inndra innslepp.
Samfundets ordinære kontrollører har ikke rett til gratis inngang på
arrangementer, med mindre dette på forhånd er særskilt avtalt med UKEsjef eller
Daghavende.\\
\end{instruksledd}

\begin{instruksledd}{Hybler}
Vanlig hybelinstruks gjelder i perioden. UKA har ansvar for at hyblene og
toaletter på sideloft til enhver tid er tilfredsstillende renholdt.
\end{instruksledd}

\begin{instruksledd}{Utøvelse av husorden}
De personer som bemyndiges i henhold til Husorden skal under sin utøvelse
være skikket til å ivareta Husorden på en korrekt måte.
\end{instruksledd}

\begin{instruksledd}{Formidling}
UKEsjefen er ansvarlig for å formidle husorden til den det måtte angå.
\end{instruksledd}

\begin{instruksleddd}{Annet}
Utstyr som eies av Samfundet skal være forsikret av Samfundet. Eventuelle
bygningstekniske ødeleggelse faller inn under Samfundets forsikring. UKA er
ansvarlig for tredjeparts utstyr, inklusive privat utstyr.
\end{instruksledd}

\begin{instruksledd}[Tilbakelevering}
Ved utløp av Husordens periode plikter UKA å sørge for at alle lokaler som
omfattet av Husorden tilbakeleveres i samme stand som ved Husordens
ikrafttredelse.
\end{instruksledd}

\end{instruks*}
