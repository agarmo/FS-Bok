
\begin{instruks}{Instruks for Studentersamfundets byggekomité}{}{}

    \begin{instruksledd}{Formål}
        \begin{enumerate}
            \item Studentersamfundets byggekomite (SBK) har som formål å koordinere, planlegge,
                budsjettere og
                gjennomføre alle større byggeprosjekter ved Studentersamfundet i Trondhjem. Videre
                skal SBK koordinere
                alle andre større oppussings- og ombyggingsprosjekter, herunder også prosjekter i
                regi av UKA.
        \end{enumerate}
    \end{instruksledd}

    \begin{instruksledd}{Sammensetning}
        \begin{enumerate}
            \item SBK skal bestå av 5 faste medlemmer. Gruppen skal alltid ha en leder og en
                økonomiansvarlig.
            \item Daglig leder, utpeker leder til SBK, og denne har en bindingstid på 2 år.
            \item Medlemmene av SBK bør være erfarne Husfolk, og ha en god oversikt over
                Studentersamfundets lokaler
                samt Husets drift. Medlemmene skal jobbe nøytralt, og likestille alle Husets
                gjenger/områder og andre
                interessenter.
            \item Nye medlemmer tas opp av daglig leder. Sittende SBK sjef lyser ut og intervjuer
                for stillingene. Bare
                nåværende eller tidligere medlemmer av noen av Studentersamfundets gjenger kan
                taes opp som medlemmer
                av SBK. Medlemmer i SBK har en bindingstid på 2 år
            \item Leder for SBK har innstillingsrett til medlemmer av gruppen overfor daglig
                leder.
            \item Sikringssjef har møterett i SBK.
            \item Daglig leder har møterett i SBK.
            \item Leder for profilgruppen har møterett i SBK.
            \item P/G sjef inngår som medlem i SBK før, under og etter planleggingen av UKA-prosjekt.
        \end{enumerate}
    \end{instruksledd}

    \begin{instruksledd}{ Ansvarsområde og arbeidsoppgaver}
        \begin{enumerate}
            \item Medlemmer i SBK plikter å kjenne innholdet av og overholde følgende instrukser på
                Huset:
                \begin{enumerate}
                    \item Husorden
                    \item Branninstruks
                    \item Rømningsinstruks
                    \item Instruks for Studentersamfundets byggekomite
                    \item Instruks for Vedlikeholdsgruppa
                    \item Studentersamfundets lover
                \end{enumerate}
            \item Gruppen skal i samråd med daglig leder utarbeide langtidsplaner og
                budsjetter for alle større bygge-,
                oppussings- og ombyggingsarbeider ved Studentersamfundet.
            \item SBK skal i samarbeid med daglig leder og profilgruppen, koordinere,
                planlegge, budsjettere og gjennomføre
                alle enkeltstående større byggeprosjekter ved Studentersamfundet.
            \item SBK skal i samarbeid med Daglig leder og profilgruppen koordinere alle
                andre større oppussings- og
                ombyggingsprosjekter, herunder også prosjekter i UKA-regi.
            \item SBK skal forut for alle større bygge-, oppussings- og
                ombyggingsarbeider, levere en prosjektskisse og
                budsjett for godkjenning i Finansstyret (FS).
            \item I forkant av bygge-, oppussings- og ombyggingsarbeider skal gruppen
                utarbeide en risikoanalyse i samråd
                med daglig leder.
            \item  Forut for alle bygge-, oppussings- og ombyggingsarbeider skal SBK sørge
                for at det foreligger en
                prosjektplan/tidsplan med milepæler.
            \item Under alle bygge-, oppussings- og ombyggingsarbeider, skal de sørge for
                at all informasjon om prosjektet er
                tilgjengelig hos daglig leder.
            \item Ved prosjekt slutt skal SBK levere en evaluering av dette, som skal inngå
                i en erfaringsdatabase for
                byggeprosjekter ved Studentersamfundet i Trondhjem. Denne
                prosjektevalueringen skal påpeke eventuelle
                mangler ved overtakelse, og plassere ansvar for oppfølging av prosjektet.
            \item
                Leder for SBK har ansvar for å dokumentere gruppens virksomhet.
            \item SBK plikter å utføre arbeid i henhold til denne instruks i tidsrommene
                primo august til medio desember og
                medio januar til medio mai. Dato for arbeid og andre tidsfrister settes av
                SBK i samråd med daglig leder.
        \end{enumerate}
    \end{instruksledd}

    \begin{instruksledd}{Godtgjørelse og rettigheter}
        \begin{enumerate}
            \item SBKs medlemmer får være med å planlegge og gjennomføre alle større bygge-,
                oppussings- og
                vedlikeholdsprosjekter ved Studentersamfundet, videre får de være med på
                langtidsplanlegging og
                budsjettering for alle større bygge-, oppussings- og vedlikeholdsprosjekter.
            \item Deltakelse i SBK gir ikke rett til funksjonærkort på Samfundet.
        \end{enumerate}
    \end{instruksledd}

    \begin{instruksledd}{Økonomi og regnskap}
        \begin{enumerate}
            \item SBK får bevilget penger fra Finansstyret, men utgiftene føres av
                regnskapsmessige årsaker som en del av
                driftsbudsjettet ved Studentersamfundet i Trondhjem. Økonomiansvarlig i SBK har
                ansvar for å føre
                regnskap, og for at budsjettene ikke overskrides.
            \item SBK plikter å sende inn budsjettforslag for hvert kalenderår. I tillegg skal
                SBK oppdatere langtidsbudsjettet
                for Bygge-, oppussings- og vedlikeholdssaker innen 01.12 hvert år.
        \end{enumerate}
    \end{instruksledd}

    \begin{instruksledd}{Endring av instruksen}
        \begin{enumerate}
            \item Endringer av denne instruksen skal skje i samsvar med Studentersamfundets lover
                og i samråd med SBKs
                medlemmer. Den nye instruksen er først gyldig når den er godkjent av Finansstyret.
        \end{enumerate}
    \end{instruksledd}

    \begin{instruksledd}{Formidling}
        \begin{enumerate}
            \item Leder for SBK plikter å gjøre nye medlemmer av gruppen kjent med denne
                instruksen.
        \end{enumerate}
    \end{instruksledd}


\end{instruks}


