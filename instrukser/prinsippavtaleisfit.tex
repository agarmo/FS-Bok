
\begin{instruks}{Prinsippavtale Mellom Styret for stiftelsen den Internasjonale Studentfestivalen i
        Trondheim (ISFiT) og Studentersamfundet i Trondhjem (Samfundet) ved Finansstyret
        (FS).}{}{}

        I avtalen betegnes den enkelte festival og dens festivalstyre "festivalen".

    \begin{instruksledd}{Bakgrunn}
        Stiftelsen ISFiT ble stiftet av Studentersamfundet i Trondhjem, Trondheim Kommune, Høgskolen
        i Sør-Trøndelag, Norges Teknisk-Naturvitenskapelige Universitet og Studentsamskipnaden i Trondheim
        som hver har bidratt med 20\% av grunnkapitalen. Stiftelsen sørger for at det opprettes et
        festivalstyre som arrangerer en internasjonal studentfestival i Trondheim. Festivalen ble første
        gang arrangert i 1990 og arrangeres nå i vårsemesteret, annethvert år oddetallsår.
    \end{instruksledd}

    \begin{instruksledd}{Formål}
        Denne avtalen fastlegger noen hovedprinsipper for samarbeidet mellom Samfundet og ISFiT.
        Dette gjøres for å skape et mer permanent grunnlag for samarbeid mellom ISFiT og Samfundets øvrige
        gjenger, slik at man slipper å gå gjennom en forhandlingsrunde med festivalstyret foran hver
        festival. En avtale som denne sikrer stabile forutsetninger og forutsigbarhet i samarbeidet mellom
        ISFiT og Samfundet.
    \end{instruksledd}        

    \begin{instruksledd}{Avtalens gyldighet og virkeområde}
        Avtalen løper til den erstattes av en ny slik avtale; fremforhandlet, godkjent og
        vedtatt av FS og ISFiT. Stiftelsen ISFiT plikter å sørge for at dens til en hver tid valgte
        festivalstyre er kjent med avtalen og avtalens innhold.
    \end{instruksledd}

    \begin{instruksledd}{Økonomi}
        ISFiTs og Samfundets gjensidige økonomiske forpliktelser er regulert i en egen avtale, som
        forhandles frem forut for hver festival. For den økonomiske avtalen er daglig leder ved Samfundet å
        betrakte som avtalepart.
    \end{instruksledd}

    \begin{instruksledd}{Servering}
        All offentlig servering skjer i regi av Serveringsgjengen eller KiSS ved Samfundet.
        Serveringsgjengen og KiSS sine økonomier er selvstendige, og atskilt fra festivalen. For eventuelle
        spesifikke serveringsavtaler vil Serveringsgjengen og Serveringsgjengens gjengsjef eller KiSS og
        KiSS sin gjengsjef være å betrakte som avtalepart(er).
    \end{instruksledd}

    \begin{instruksledd}{Arrangementsgjennomføring}
        ISFiT vil iløpet av festivalperiodene gjennomføre en rekke ulike arrangement med
        svært ulik profil på Samfundet. For å sikre gjennomføringen vil ISFiT sørge for å knytte til seg
        personer med arrangementserfaring fra Samfundet. 

        ISFiT er ansvarlige for koordinering, teknisk koordinering og sikkerhet, fortrinnsvis gjennom egne
        funksjonærer, dernest gjennom bruk av aktive eller tidligere aktive husmennesker med
        arrangementskompetanse.
        Arrangementssikkerheten skal være gjenstand for nøye gjennomgang, og ISFiT plikter i forkant av
        festivalen å
        konsultere Husmann, Daglig Leder og Sikringsansvarlige ved Samfundet vedrørende dette. Enhver
        anmerkning fra en
        eller flere av disse, er å betrakte som et kontraktfestet pålegg og må utbedres før arrangementet
        kan gjennomføres.

        Arrangementsgjennomføring skjer helt og holdent for festivalens egen regning og risiko, og Samfundet
        bærer ingen
        økonomiske forpliktelser i forbindelse med disse.

        Ethvert arrangement i regi av ISFiT på Samfundet skal holde like høy, eller høyere,
        arrangementsmessig kvalitet som
        Samfundets ordinære aktiviteter.
    \end{instruksledd}

    \begin{instruksledd}{Lokaler}
        Ved fordeling av offentlige areal i regi av RomBookingsKomiteen (RBK) har ISFiT førsteprioritet i
        festivalperioden.
        ISFiT melder inn arealbehov til Samfundet på ordinært vis gjennom RBK. Festivalen skal ha meldt inn
        sine arealbehov innen 10. januar i festivalsemesteret. Kostnader ved bruk av offentlige lokaler
        reguleres i den økonomiske
        avtalen mellom FS og festivalen. ISFiT kan bruke tilgjengelige private arealer uten kostnader.
    \end{instruksledd}

    \begin{instruksledd}{Teknisk utstyr}
        ISFiT plikter å finne gode samarbeidsformer med aktuelle gjenger på Samfundet. Særlig viktig er ett
        teknisk
        informasjonspunkt mellom Studentersamfundet og ISFiT. Økonomiske retningslinjer i forbindelse med
        bruk av
        teknisk utstyr, klargjøres i egen økonomisk avtale mellom FS og festivalen.
    \end{instruksledd}

    \begin{instruksledd}{Møte og kultur}
        Ved bruk av husinterne krefter avtales dette mellom festivalen og aktuelle bidragsytere. Eventuell
        honorering avtales
        mellom partene, uavhengig av denne avtalen.

        Ved bruk av eksterne krefter er det viktig at disse til enhver tid er kjent med at ISFiT (og ikke
        Samfundet) står som
        arrangør. Både økonomisk og arrangementsmessig er det svært viktig at et slikt skille er
        velfungerende og tydelig.
    \end{instruksledd}

    \begin{instruksledd}{Konsert og fest}
        Fest- og konsertarrangement på fredager og lørdager i festivalperioden gjennomføres etter nærmere
        avtale mellom
        festivalen og Klubbstyret (fredager) og Lørdagskomiteen og Styret (lørdager). Eventuelt lån eller
        leie av
        arrangementsteknisk utstyr, lokaler og nøkler reguleres i egen avtale mellom festivalen og disse
        gjengene.

        Økonomiske retningslinjer for disse og øvrige arrangement klargjøres i egen økonomisk avtale mellom
        FS og
        Festivalen.
    \end{instruksledd}

    \begin{instruksledd}{ISFiTs funksjonærer} 
        ISFiTs funksjonærer har status som gjengmedlemmer uten følgerett på Samfundet, mens ISFiTs
        mellomledere
        (begrenset oppad til maksimalt 35, trettifem personer) har status som gjengmedlemmer med følgerett
        på Samfundet.
        Festivalstyret (begrenset oppad til maksimalt 9, ni, personer) har status som funksjonær på
        Samfundet. Husmann kan i
        henhold til husorden og hybelinstrukser inndra følgerett til Sideloftene dersom husmann finner det
        nødvendig.

        Både ISFITs festivalstyre og funksjonærer må være medlemmer av Samfundet. Det forventes at ISFiTs
        funksjonærer
        har kjennskap til husorden og andre relevante instrukser som, men ikke begrenset til, brann-,
        nøkkel- og
        hybelinstrukser. Festivalstyret har som husfunksjonærer et særskilt ansvar for å sørge for opplæring
        og oppfølging av
        ISFiT-funksjonærer.

        Dersom en av ISFiTs funksjonærer forbryter seg mot Samfundets lover eller FSs instrukser følges
        vanlig
        disiplinærsaksprosedyre, som nedfelt i Samfundets lover §34.

        ISFiT får gratis ID-kort fra Samfundet og skaffer selv oblater som skiller mellomledere tydelig fra
        funksjonærer.
        Funksjonæroblater til festivalstyret får ISFiT av Samfundet

        De av Samfundets gjenger som bidrar under festivalen har ISFiT-status etter nærmere avtale med de
        respektive
        gjenger. Serveringsgjengen og KISS sine rettigheter og plikter under festivalen spesifiseres nærmere
        i avtale med
        disse.

        Under festivalen har personer med funksjonærstatus og gjengmedlemstatus på Samfundet de samme
        rettighetene som
        praksis er i vanlig drift. Festivalen har, på lik linje med arrangerende gjenger i vanlig drift,
        rett til å inndra
        funksjonærenes følgerett til arrangementer i henhold til funksjonærinstruksens punkt sju. Dette skal
        varsles i rimelig
        tid før arrangementet.

        Endringer i Samfundets husorden og instrukser vil også gjelde for ISFiT-funksjonærer.
    \end{instruksledd}


    \begin{instruksledd}{Endringer i prinsippavtalen}
        Endringer i prinsippavtalen kan kun gjøres etter felles vedtak i FS og hos Stiftelsen ISFiT.
    \end{instruksledd} 


\end{instruks}


