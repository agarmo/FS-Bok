\begin{instruks}{Instruks for Vedlikeholdsgruppa}{}{}

    \begin{instruksledd}{Formål}
        \begin{enumerate}
            \item Formålet med Vedlikeholdsgruppa (VG) er å vurdere og koordinere alle
                bygge-, oppussings-, og
                vedlikeholdsarbeider ved Samfundet. Videre skal VG prioritere og
                gjennomføre disse bygge-, oppussings- og
                vedlikeholdsarbeidene, samt kontrollere at oppgavene er utført.
            \item VG skal fungere som et arbeidsutvalg for virksomhetsområdet og er
                underlagt daglig leder. Utvalget arbeider
                etter økonomiske rammer for drift.
        \end{enumerate}
    \end{instruksledd}

    \begin{instruksledd}{ Sammensetning}
        \begin{enumerate}
            \item VG har følgende sammensetning:
                \begin{enumerate}
                    \item Daglig leder
                    \item Vaktmester
                    \item Leder for Diversegjengen
                    \item Leder for SBK, eller dennes representant
                    \item Medlem(mer) med spesiell interesse av/kunnskap om vedlikeholdet på Huset
                \end{enumerate}
            \item Etter behov innkalles en representant fra følgende:
                \begin{enumerate}
                    \item Sikringskomiteen
                    \item IT-komiteen, Videokomiteen og Forsterkerkomiteen
                    \item Internvaktmester i Klostergt. 35 i saker som angår vedlikeholdsarbeidet i
                        Klostergata 35
                \end{enumerate}
            \item Leder for vedlikeholdsgruppa skal fortrinnsvis ha erfaring fra Diversegjengen.
                Sittende leder intervjuer og innstiller kandidat til ny leder, overfor daglig leder.
        \end{enumerate}
    \end{instruksledd}

    \begin{instruksledd}{Ansvarsområder og plikter}
        \begin{enumerate}
            \item Medlemmer i vedlikeholdsgruppen plikter å kjenne innholdet av og overholde
                følgende instrukser på Samfundet:
                \begin{enumerate}
                    \item Husorden
                    \item Branninstruks
                    \item Rømningsinstruks
                    \item Sikringsinstruks
                    \item Instruks for Vedlikeholdsgruppa
                    \item Studentersamfundets lover
                \end{enumerate}
            \item VG skal vurdere og prioritere alle bygge og vedlikeholdsarbeider ved
                Studentersamfundet.
            \item VG skal fordele alle bygge og vedlikeholdsarbeider, og sørge for at
                nødvendig vedlikehold i forbindelse med
                normal drift gjennomføres.
            \item VG skal to ganger i året, medio oktober og medio mars, gjennomføre
                vedlikeholdsrunder på alle Samfundets
                arealer. I UKA-år gjennomføres vedlikeholdsrunden i høstsemesteret i
                etterkant av UKA. Disse rundene skal
                i størst mulig grad avdekke vedlikeholdsoppgaver.
            \item VG kan fordele oppgaver til vaktmester, Diversegjengen, samt andre
                gjenger på huset, og hente inn ekstern
                kompetanse hvis nødvendig, for å få gjennomført vedlikeholdsoppgaver.
            \item Gruppen har ansvar for å påse at vedlikeholdsarbeider den prioriterer
                 gjennomført, utføres i henhold til offentlige krav og forskrifter
            \item Leder for VG har ansvaret for å administrere arbeidet med meldinger om
                vedlikeholdsoppgaver, som
                kommer inn fra gjengene.
        \end{enumerate}
    \end{instruksledd}

    \begin{instruksledd}{ Godtgjørelse og rettigheter}
        \begin{enumerate}
            \item VG gruppens medlemmer får være med på å planlegge, prioritere, fordele og følge
                opp vedlikeholdsarbeider
                og ombygging/utskiftninger, bygningsmessige og tekniske forandringer nødvendig for
                normal drift.
            \item Leder for Vedlikeholdsgruppa har rettigheter til funksjonærkort på Samfundet.
                Deltakelse for andre i VG gir
                ikke rettighet til funksjonærkort på Samfundet.
        \end{enumerate}
    \end{instruksledd}

    \begin{instruksledd}{Økonomi og regnskap}
        \begin{enumerate}
            \item VG får bevilget penger over driftsbudsjettet, og regnskapsføres som en del av
                dette. Leder av VG i samarbeid
                med daglig leder har ansvaret for å føre regnskap, og for at budsjettene ikke
                overskrides.
        \end{enumerate}
    \end{instruksledd}


    \begin{instruksledd}{Endring av instruksen}
        \begin{enumerate}
            \item Endringer av denne instruksen skal skje i samsvar med Studentersamfundets lover og
                i samråd med
                Vedlikeholdsgruppens medlemmer. Den nye instruksen er først gyldig når den er godkjent
                av Finansstyret.
        \end{enumerate}
    \end{instruksledd}

    \begin{instruksledd}{Formidling}
        \begin{enumerate}
            \item Leder for Vedlikeholdsgruppa plikter å gjøre nye medlemmer av gruppen kjent med
                denne instruksen.
        \end{enumerate}
    \end{instruksledd}

\end{instruks}



