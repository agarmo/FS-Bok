\begin{instruks*}{Instruks for Informasjonsteknologikomit\'een}

    \begin{instruksledd}{Formål}
        \begin{enumerate}
            \item Informasjonsteknologikomiteen, heretter ITK, skal drive, vedlikeholde og sørge for
                eventuell utbygging av
                Studentersamfundets datanett - herunder nettverk og nettverkselektronikk, tellesystemet i
                samråd med daglig
                leder, samt vedlikeholde og eventuelt utbygge Samfundets nøkkelkortsystem, herunder
                medlemsdatabasen i
                samarbeid med Samfundets vaktmester og sikringskomiteen.
            \item  ITK skal søke å koordinere bruken av datautstyr for Studentersamfundets øvrige
                enheter. Dette gjøres
                gjennom kontinuerlig kontakt med, og oppfølging av deres dataansvarlige.
            \item  ITK skal ha ansvar for dokumentasjon av kabling på Studentersamfundet. ITK skal
                samarbeide med andre
                enheter ved Studentersamfundet som har ansvaret for forskjellige kablingsanlegg
                (Student-TV, Videokomiteen, Forsterkerkomiteen og Diversegjengen) for å få til dette.
            \item  ITK skal tilby sentraliserte IT-tjenester til Samfundets gjenger, herunder drift av
                Samfundets tjenermaskiner
                og de tjenester som benytter disse. IT-komiteen står for den praktiske kommunikasjonen
                mot NTNUs IT-
                seksjon (ITEA) der det er nødvendig.
        \end{enumerate}
    \end{instruksledd}

    \begin{instruksledd}{Sammensettning}
        \begin{enumerate}
            \item ITK består av 8-12 aktive medlemmer og et varierende antall pensjonister.
            \item Et medlem i ITK er aktivt i 2,5 år.
            \item  Under UKA kan ITK utvides midlertidig for å utføre det arbeid som er nødvendig før og
            under UKA.
        \end{enumerate}
    \end{instruksledd}

    \begin{instruksledd}{Ansvarsområder og plikter}
        \begin{enumerate}
            \item Medlemmer i ITK plikter å kjenne innholdet av og overholde følgende instrukser på Huset:
                \begin{enumerate}
                    \item  Generell gjenginstruks.
                \end{enumerate}
            \item ITK disponerer og har ansvaret for følgende rom med tilhørende utstyr:
                \begin{enumerate}
                    \item Serverrommet i 4. etasje
                    \item Kontoret i 4. etasje
                    \item Telematikkrommet på Nordre sideloft
                \end{enumerate}
            \item ITK har mulighet til å benytte Fotogjengens hybel på midlertidig basis, etter nærmere
                 avtale med Fotogjengen.
        \end{enumerate}
    \end{instruksledd}

    \begin{instruksledd}{Formidling}
        \begin{enumerate}
            \item Gjengsjefen plikter å gjøre nye medlemmer av gjengen kjent med denne
                instruks.
        \end{enumerate}
    \end{instruksledd}


\end{instruks*}
