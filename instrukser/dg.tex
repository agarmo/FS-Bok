
\addcontentsline{toc}{chapter}{Gjenginstrukser}

\begin{instruks}{Instruks for Diversegjengen}{9. februar 2011}{24. febraur 2011}

    \begin{instruksledd}{Formål}
        \begin{enumerate}
            \item Diversegjengens (DG) oppgaver er å arbeide med bygningsmessig vedlikehold og eventuelle
                ombygginger på Samfundets faste eiendommer, i samråd med Finansstyret, vaktmester og Vedlikeholdsgruppa.
                Arbeidet forutsettes utført i nært samarbeid med vaktmester .
        \end{enumerate}
    \end{instruksledd}

    \begin{instruksledd}{Sammensetning}
        \begin{enumerate}
            \item DG består av inntil 11 faste medlemmer, hvorav 8 er vanlige funksjonærer
                og inntil 3 er faglærte elektrikere og funksjonærer.
            \item Et medlem i DG er aktivt i 2 år og har etter det mulighet til å beholde
                funksjonærstatusen som pangsjonist.
            \item DG opprettholder virksomheten under UKA, og utvider gjengen med det nødvendige antall
                som må til for å løse gjengens oppgaver før og under UKA. Elektriker(ene) til DG går før UKA ut av DG for å
                danne egen gjeng, Elektriker-gjengen (EG).
        \end{enumerate}
    \end{instruksledd}

    \begin{instruksledd}{Ansvarsområde og plikter}
        \begin{enumerate}   
            \item  Medlemmer i DG plikter å kjenne innholdet av og overholde følgende instrukser på Huset:
                \begin{enumerate}
                    \item Generell gjenginstruks.
                \end{enumerate}
            \item DG disponerer og har ansvaret for følgende rom med tilhørende utstyr:
                \begin{enumerate}
                    \item Kontor
                    \item Hybel (Gufsejuvet)
                    \item Verksted
                    \item Gang med tilhørende skap
                    \item Toalett (1 av 2, DGs til venstre)
                    \item  Dusj
                \end{enumerate}
        \end{enumerate}
    \end{instruksledd}

    \begin{instruksledd}{Formidling}
        \begin{enumerate}
            \item Gjengsjefen plikter å gjøre nye medlemmer av DG kjent med denne instruks.
        \end{enumerate}
    \end{instruksledd}


\end{instruks}


