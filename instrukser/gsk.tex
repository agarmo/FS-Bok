
\begin{instruks}{Instruks for Gjengsjefskollegiet}{27. november 2009}{3. Desember 2009}

    \begin{instruksledd}{Fomål}
        GSK skal sikre bedre kommunikasjon mellom de forskjellige virksomheter i
        Studentersamfundet.
        GSK er et høringsorgan for Finansstyret (FS) i saker av prinsipiell og
        administrativ art som berører
        foreningsdriften. GSK er først og fremst et orienteringsforum, dernest et forum
        som skal fatte tiltak.
    \end{instruksledd}

    \begin{instruksledd}{Sammensetning}
        \begin{enumerate}
            \item Medlemmer:
                Medlemmene er gjengsjefer i alle gjenger og tilsluttede foreninger ved
                Studentersamfundet i Trondhjem.
            \item Observatører:
                Styret og FS ved gjengrepresentant. Gjengenes representant i FS har møteplikt i
                GSK. Rådet kan delta med
                en observatør etter eget ønske.
            \item Nye grupperinger:
                Nye grupperinger kan søke om plass i GSK. Dette godkjennes med 3/4 flertall.
        \end{enumerate}
    \end{instruksledd}

    \begin{instruksledd}{Forretningsorden til Gjengsjefkollegiet}
        \begin{enumerate}
            \item Både medlemmer og observatører har forslags- og talerett, mens bare medlemmene
                har stemmerett. Hvis et
                medlem eller en observatør ikke kan møte er det tillatt å sende stedfortreder.
                Stedfortreder erverver da
                medlemmets/observatørens rettigheter.
            \item Møte avholdes minst en gang i måneden i semesteret eller ved behov. Alle
                medlemmer og observatører av
                GSK kan innkalle til ekstraordinært GSK-møte.
            \item Det står GSK fritt å innkalle andre enn de som er nevnt under punkt 2 hvis det
                anses formålstjenlig.
            \item Gjengsjefkollegiet skal vurdere alle innkomne saker. Alle medlemmer og
                observatører av GSK kan fremme
                saker.
            \item I løpet av vårsemesteret velger GSK ny leder som velges for et år. Lederen er
                ansvarlig for innkallelse,
                saksforberedelse, saksliste og referat, samt retningslinjer gitt i punkt j).
                Lederen kan etter eget ønske velge
                om han/hun vil være medlem av gjengsekretariatet (GS).
            \item
                I løpet av vårsemesteret velger GSK gjengrepresentant til Finansstyret. Denne
                fungerer som medlem av FS
                påfølgende studieår. Såfremt mulig bør kandidater til gjengrepresentantstillingen
                være ferdig med sitt virke i
                en av gjengene.
            \item Vedtak i GSK kan kun fattes hvis over halvparten av det totale antall
                stemmeberettigede i GSK stemmer for.
                Dette gjelder uansett antall fremmøtte. Eneste unntak er i følgende situasjon:
                Hvis votering over et
                vedtaksforslag må utsettes grunnet for få fremmøtte, kan neste GSK-møte votere
                over det samme
                vedtaksforslaget uavhengig av antall fremmøtte. Vedtaksforslaget blir vedtatt hvis
                over halvparten av de
                fremmøtte stemmer for.
            \item Medlemmene i GSK skal lojalt følge opp de vedtak GSK fatter.
            \item Det skal føres protokoll fra møtene som oversendes Finansstyret. Originale
                sakspapirer og utskrevne
                protokoller skal oversendes Samfundets arkivar.
            \item Retningslinjer for innkallelse, sakspapirer og saksgang.
                \begin{enumerate}
                    \item GSK-leder skal sende ut innkallelse minimum en uke før møtet skal
                        avholdes. Saksliste med tilhørende
                        sakspapirer skal sendes ut minimum tre dager før. All informasjon sendes
                        ut via e-postlisten gsk@samfundet.no.
                    \item Vedtaksforslag må være sendt ut minimum en uke på forhånd. Dette for at
                        gjengsjefene skal ha mulighet
                        til å sette seg inn i forslaget, samt kunne drøfte det med sin egen gjeng.
                    \item Innspill rundt innhold og ordlyd
                        ønskes sendt til forfatter av vedtaksforslaget før møtet. Dette for å
                        redusere diskusjonen rundt innhold og
                        ordlyd på møtet.
                    \item Medlemmene i GSK plikter å sette seg inn i de utsendte sakspapirene og
                        eventuelle vedtaksforslag før
                        møtet.
                    \item GSK-leder skal være bevisst sin rolle som møteleder og holde en tydelig
                        møtestruktur.
                    \item GSK kan definere arbeidsgrupper (som for eksempel virksomhetsområder (VO))
                        for videre behandling
                        av saker. Dette gjelder spesielt saker der kun deler av GSK er direkte
                        involvert. Når arbeidsgruppen har
                        behandlet saken, skal resten av GSK informeres. GSK-leder er ansvarlig for
                        å følge opp
                        arbeidsgruppene.
                    \item Møtet skal gjennomføres med følgende saksgang:
                        \begin{enumerate}
                            \item Godkjenning av referatet og møteinnkalling
                            \item Påmelding til eventuelt
                            \item Runde rundt bordet
                            \item Orienteringssaker
                            \item Vedtakssaker
                            \item Diskusjonssaker
                            \item Eventuelt
                            \item Åpent punkt
                            \item Kritikk av møtet
                        \end{enumerate}
                \end{enumerate}
        \end{enumerate}
    \end{instruksledd}

    \begin{instruksledd}{Saker som skal behandles}
        Hovedoppgaven til GSK er å sørge for bedre samarbeid og kommunikasjon innad i
        Studentersamfundet. Generelt
        gjelder dette:
        \begin{enumerate}
            \item Drøfte konkrete planer vedrørende gjengenes drift.
            \item Koordinere større oppgaver som berører flere gjenger.
            \item Gjengrepresentanten i Finansstyret skal informere om bakgrunnen for
                vedtak i FS og om fremtidige saker i FS.
            \item Generelt informere om virksomheten i de enkelte foreninger.
            \item I tillegg skal GSK behandle saker av prinsipiell og administrativ art
                som berører foreningsdriften. GSK skal
                spesielt behandle saker som angår gjengenes rettigheter og plikter.
            \item Behandle andre saker som blir oversendt fra Finansstyret eller
                forretningsdriften.
        \end{enumerate}

    \end{instruksledd}
\end{instruks}







