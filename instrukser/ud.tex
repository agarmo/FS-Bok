
\begin{instruks}{Instruks for Under Dusken}{1. }{ }

    \begin{instruksledd}{Formål}
        \begin{enumerate}
            \item Under Dusken skal gi ut avis minst annenhver uke for studentene i Trondheim og byens befolkning. Under Dusken er et selvstendig organ for studenter utgitt i Trondheim av A/S Mediastud. Under Dusken blir delt ut gratis p� l�resteder i Trondheim med medlemsrett i Studentersamfundet i Trondhjem og kommer ut annenhver tirsdag, �tte ganger i semesteret.
        \end{enumerate}
    \end{instruksledd}

    \begin{instruksledd}{Sammensetning}
        \begin{enumerate}
            \item Under Dusken best�r av 74 medlemmer, hvorav 14 er funksjon�rer. Nye medlemmer tas opp en gang i semesteret, med mulighet for suppleringsopptak utover dette. Bare medlemmer av Studentersamfundet kan tas opp i Under Dusken.
            \item Et medlem av Under Dusken er aktivt i 1 �r. Et medlem kan innvilges permisjon av redaksjonen og kan ogs� fritas fra sitt gjengmedlemskap.
            \item Ansvarlig redakt�r innstilles av Under Duskens allm�te og velges av Styret i Mediastud AS. Redakt�r velges for ett �r av gangen. Redakt�ren er ansvarlig for utformingen av det redaksjonelle innholdet samt kvaliteten p� dette.
		\item Daglig leder innstilles av redakt�r og velges av Under Duskens allm�te. Daglig leder er gjengsjef og har ansvar for den dagligedriften og gjengen internt p� Samfundet. Daglig leder velges for ett �r av gangen.
		\item Under Dusken opprettholder utgivelsene under UKA.
\end{enumerate}
     \end{instruksledd}{Ansvarsområde og plikter}
        \begin{enumerate}   
            \item  Medlemmer i Under Dusken plikter å kjenne innholdet av og overholde følgende instrukser på Huset:
                \begin{enumerate}
                    \item Generell gjenginstruks.
                \end{enumerate}
            \item Under Dusken disponerer og har ansvaret for følgende rom med tilhørende utstyr:
                \begin{enumerate}
                    \item Under Dusken hybel
                \end{enumerate}
		\item Under Dusken plikter � utf�re arbeid i henhold til punkt 1. i tidsrommene medio august til medio desember og medio januar til medio juni. Dato for arbeid og andre tidsfrister settes av redakt�ren.
        	\item Redakt�ren er ansvarlig for Under Dusken utstyr, og for at dette blir fagmessig betjent. Gjengsjefen er ogs� ansvarlig for at Under Duskens utstyr blir forsvarlig sikret mot tyveri og skade, og at brannvernforskrifter og sikringsbestemmelser blir overholdt p� Under Duskens omr�de.
		\end{enumerate}
        \begin{instruksledd}{�konomi}
            \begin{enumerate}
                \item Gjengsjefen er ansvarlig for den biten av �konomien som ikke er knyttet til redaksjonell drift avUnder Dusken.
            \end{enumerate}
        \end{instruksledd}
        \begin{instruksledd}{Endring av instruksen}
            \begin{enumerate}
                \item Endringer av denne instruksen skal skje i samsvar med Studentersamfundets lover og i samr�d med Under Duskens medlemmer. Alle endringer skal forelegges Finansstyret til uttalelse og den nye instruksen er f�rst gyldig n�r den er godkjent av styret i Mediastud AS.
            \end{enumerate}
        \end{instruksledd}
        \begin{instruksledd}{Formidling}
            \begin{enumerate}
                \item Gjengsjefen plikter å gjøre nye medlemmer av Under Dusken kjent med denne instruks.
            \end{enumerate}
        \end{instruksledd}


\end{instruks}


