\begin{instruks}{Instruks for arealdisponering på Samfundet}{3. desember 2009}{3. desember 2009}

    Studentersamfundet i Trondhjem eier alle lokalene i Elgesetergate 1. Disse disponeres
    av Studentersamfundets gjenger og administrasjon etter retningslinjer fra Finansstyret.

    Alle henvendelser som gjelder omdisponering og endring av offentlige og private
    arealer ved Studentersamfundet,
    skal rettes til Gjengsekretariatet (GS). Med dette menes behov for mer areal,
    frigjorte arealer, oppussing, skader,
    mangler og oppgradering av lokalene.

    Saker som omfatter behov for mer areal og frigjort areal behandles av GS, som
    orienterer de berørte parter. Følgende
    instanser skal høres under saksbehandlingen;

    \begin{itemize}
        \item Administrasjonen
        \item Samfundets byggekomiteen (SBK)
        \item Berørte gjenger/organer
        \item Styret ved endringer på offentlige lokaler
    \end{itemize}

    GS fatter et vedtak på grunnlag av høringsuttalelsene og informerer de berørte parter
    om vedtaket. Administrasjonen
    og SBK har ankemulighet. Anker skal meldes inn til GS innen 14 dager etter
    offentliggjøring av vedtak.

    I saker preget av hastverk og av mindre størrelsesorden, gis GS og SBK sammen fullmakt
    til å fatte vedtak uten
    høring.

    Vedtaket fra GS sendes til Finansstyret (FS) for godkjenning. Dersom FS underkjenner
    vedtaket, skal saken behandles
    på nytt.

    Saker som omhandler oppussing, skader, mangler og oppgradering videresendes SBK. Saken
    behandles og sendes
    tilbake til GS, som videreformidler utfallet. Utgifter tilknyttet endring på
    bygningsmasse behandles av SBK.

\end{instruks}

