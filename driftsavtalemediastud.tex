\begin{instruks}{Driftsavtale For Mediastud AS}{1. november 1999}{}

	\begin{instruksledd}{ }
		Studentsamskipnaden i Trondheim (SiT) og Studentersamfundet i
		Trondhjem (Samfundet) er enige om å inngå
		følgende avtale om driftsvilkår for selskapet Mediastud AS.
	\end{instruksledd}

	\begin{instruksledd}{Selskapet}
		\begin{enumerate}
		\item Aksjeselskapet Mediastud AS skal drive nærradioen
		``Studentradio'n'', tidsskriftet ``Studentersamfundets
		prenta blad Under Dusken'' heretter kalt ``Under Dusken'',
		nærfjernsynet ``Student-TV'n'', samt delta i andre
		former for mediavirksomhet.
		\item Det skal velges fem medlemmer til styret i selskapet. To
		medlemmer velges etter forslag fra SiT, to
		medlemmer velges etter forslag fra Samfundet, mens et
		eksterntmedlem velges av generalforsamlingen etter
		forslag fra styret i Mediastud AS. Det eksterne medlemmet skal
		bidra med mediefaglig kompetanse og
		vedkommende skal ikke ha tilknytning til SiT eller Samfundet.
		Partene står fritt til å velge varamedlemmer til
		styret.
		\item Styret velger selv leder og nestleder. Styreleder har en
		funksjonstid på ett år. Vervet som styreleder skal
		fortrinnsvis alternere annethvert år mellom de to partene. Den
		partsom står for tur til å ha ledervervet, må i så
		fall sette fram krav om dette på selskapets generalforsamling.
		Dersom slikt krav ikke blir framsatt, står styret
		fritt til å konstituere seg selv med den begrensning at leder-
		og nestlederverv skal fordeles på de to partene.
		Ved stemmelikhet har styrets leder dobbelstemme.
		\item Avtalepartene har selv ansvar for kompensasjon av sine
		styrerepresentanter. Mediastud AS har selv ansvar
		for kompensasjon av det eksterne styremedlemmet.
		\end{enumerate}
	\end{instruksledd}

	\begin{instruksledd}{Eierforhold}
		\begin{enumerate}
		\item Selskapets aksjekapital er på kr.120.000.-.
		Studentsamskipnaden i Trondheim og Studentersamfundet i
		Trondhjem står hver tegnet med 60 aksjer à kr 1000.-.
		\end{enumerate}
	\end{instruksledd}

	\begin{instruksledd}{Drift}
		\begin{enumerate}
		\item Avtalepartene forplikter seg til å yte et årlig tilskuddet
		til Mediastud AS på 750000 kroner per 1999.
		Tilskuddet utbetales årlig. Tilskuddet er gjenstand for
		justering etter konsumprisindeks i forutgående 12
		månedersperiode. Av dette tilskuddet skal 40\% betales av
		Samfundet og 60\% av SiT. Beløpet overføres
		innen tre uker etter oversendelse av faktura fra Mediastud AS.
		Ved forsinkelse kan Mediastud AS beregne
		renter og gebyrer etter de til enhver tid gjeldende lover og
		forskrifter.
		\item Hvis en av eierne krever det kan selskapets samlede
		overskudd, eller en andel av dette, gå til reduksjon av det
		samlede driftstilskudd. Krav om slik reduksjon skal fremmes i
		forbindelse med ordinær generalforsamling,
		og kan tidligst gjøres gjeldende for neste års utbetaling av
		tilskudd.
		\item Studentersamfundet skal stille egnede lokaler til rådighet
		for Studentradioen og Under Dusken.
		Studentersamfundet har ansvar for vedlikehold og drift av disse.
		Standard og størrelse på lokalene reguleres i
		en egen avtale.
		\item StudentTV skal så sant det er mulig få lokaler i Samfundet
		under de samme betingelser som Under Dusken
		og Studentradio'n.
		\item Utleier skal få utgifter av administrasjon, energi og
		andre indirekte og direkte kostnader refundert.
		Refusjonsbeløpet utgjør 35.000 kroner for 1999 og belastes
		Mediastud. Dette beløpet er gjenstand for
		regulering etter konsumprisindeks på samme måte som nevnt i
		3.1.
		\item Samfundet skal videre gi funksjonærstatus til medlemmer i
		de ulike virksomhetene på linje med andre
		undergrupper av Samfundet.
		\end{enumerate}
	\end{instruksledd}

	\begin{instruksledd}{Varenavn}
		\begin{enumerate}
		\item Navnet og logoen til "Under Dusken" er Samfundets eiendom
		og kan brukes vederlagsfritt av Mediastud AS.
		Øvrige varenavn tilhører Mediastud AS med mindre annet framgår.
		\end{enumerate}
	\end{instruksledd}

	\begin{instruksledd}{Avtalens Varighet}
		\begin{enumerate}
		\item Avtalen løper til 31.12 2002, etter dette kan avtalen sies
		opp. Oppsigelse skal varsles skriftlig minst 12
		måneder i forveien.
		\end{enumerate}
	\end{instruksledd}

	\begin{instruksledd}{Avvikling}
		\begin{enumerate}
		\item Dersom ikke en ny avtale er frem forhandlet etter
		oppsigelse skal Mediastud AS avvikles dersom en av
		partene ønsker dette.
		\item Ved avvikling av Mediastud AS skal resterende aktiva og
		passiva etter utløsning av eventuelle fordringer i
		selskapet fordeles etter aksjeparten. Dette med unntak
		avpartenes eventuelle pant i boet.
		\end{enumerate}
	\end{instruksledd}

	\begin{instruksledd}{Ikrafttredelse}
		\begin{enumerate}
		\item Denne avtale og de retningslinjer gitt herunder får
		virkning fra 1.7.1999, og erstatter alle tidligere avtaler.
		Tilskudd gjøres gjeldende fra og med 2. halvår 1999. Allerede
		innbetalte tilskudd for 1. halvår 1999 endres
		ikke av denne avtalen.
		\end{enumerate}
	\end{instruksledd}

	\begin{instruksledd}{Underskrifter}
		\begin{enumerate}
		\item Avtalen signeres i 3 - tre - eksemplarer av styreleder i
		SiT og leder i Finansstyret i Samfundet. SiT,
		Samfundet og Mediastud AS beholder 1 - ett - eksemplar hver.
		\end{enumerate}
		\end{instruksledd}

\vfill
For Studentsamskipnaden i Trondheim 	\hfill	For Studentersamfundet i Trodnhjem

Konserstyreleder \hfill		Finansstyrets leder


\end{instruks}

