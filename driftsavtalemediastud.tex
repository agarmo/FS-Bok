\begin{instruks}{Aksjonæravtale mellom Studentersamfundet i Trondhjem
(``Samfundet'') og
Studentsamskipnaden i Trondheim (``SiT'') sammen benevnt som Partene for
eierskap til selskapet Mediastud AS (``Mediastud'')}{2. mars 2010}{2. mars 2010}

	\begin{instruksledd}{Bakgrunn og formål }
Aksjeselskapet Mediastud AS er opprettet av Partene for å drive studentradioen
``Radio Revolt'', studenttidsskriftet ``Studentersamfundets prenta blad Under
Dusken'' heretter kalt ``Under Dusken'', studentfjernsynet ``Student-TV’n'', samt
delta i andre former for mediavirksomhet.

Denne avtale erstatter ``Driftsavtale for Mediastud AS'' av november 1999.
	\end{instruksledd}

	\begin{instruksledd}{Aksjekapital, eierforhold og vedtekter}
	Mediastud har utstedt 120 aksjer pålydende kr. 1.000,00, hvorav 60
	aksjer eies av Samfundet og 60 aksjer eies av SiT.
	
	Aksjene er bare omsettelige mellom Partene.

	Ved utstedelse av nye aksjer plikter Partene å tegne et likt antall
	aksjer slik at eierforholdet opprettholdes med lik eierandel. 
	
	Partene er enige om at det ikke skal utbetales utbytte fra Mediastud.
	
	Endring av Mediastuds vedtekter krever enighet mellom Partene. 
	\end{instruksledd}

	\begin{instruksledd}{Partenes forpliktelser}
	Partene er forpliktet til å bidra til utvikling av Mediastud gjennom
	lojalt samarbeid til beste for Mediastuds utvikling. 

	Partene forplikter seg til å yte et årlig tilskudd til Mediastud AS på
	kr. 900.000 (``Tilskuddet''). Tilskuddet skal justeres i henhold til
	endringen i konsumprisindeksen fra 1. januar året før til 1. januar
	gjeldende år. Utgangspunkt for indeksregulering er kr. 900.000 per
	01.01.2009.

	Av Tilskuddet skal 40\% betales av Samfundet og 60\% betales av SiT. 
	
	Styret i Mediastud oversender faktura til Partene sammen med fjorårets
	regnskap og årets budsjett. Partene har da en betalingsfrist på 3 uker.
	Ved forsinket betaling kan Mediastud AS kreve lovbestemt
	forsinkelsesrente. 

	Hvis Partene er enige om det, kan det foretas en endring av Tilskuddet.
	Krav om endring av Tilskuddet skal fremmes i forbindelse med Mediastuds
	ordinære generalforsamling.  Tilskuddet kan i et slikt tilfelle tidligst
	endres for Tilskuddet som utbetales for påfølgende år.
	
	Navnet og logoen til ``Under Dusken'' er Samfundets eiendom og kan brukes
	vederlagsfritt av Mediastud AS. Øvrige varenavn tilhører Mediastud AS
	med mindre annet er avtalt.
	\end{instruksledd}

	\begin{instruksledd}{Forholdet til SiT og Samfundet}
	Alle avtaler mellom Partene og Mediastud skal inngås på
	forretningsmessige betingelser og til virkelig verdi. Driftstekniske
	avtaler som angår forholdet mellom mediegjengene og Samfundet reguleres
	i egen avtale mellom Mediastud og Samfundet. Slike avtaler skal på
	forhånd forelegges og godkjennes av den annen Part. 
	
	Samfundet skal gi funksjonærstatus til studenter som er medarbeidere i
	Mediastuds ulike virksomheter. Omfanget og innholdet av funksjonærstatus
	reguleres i en egen avtale mellom Samfundet og Mediastud.
	
	Samfundet skal stille hybel til rådighet for medarbeidere i Mediastud
	med funksjonærstatus.  Medarbeiderne i Mediastud har ansvar for
	vedlikehold og drift av denne. Standard og størrelse på lokalene
	reguleres i en egen avtale.
	\end{instruksledd}

	\begin{instruksledd}{Bestemmelser vedrørende Mediastuds ledelse}
		Partene setter ned en felles valgkomité på totalt tre personer
		som skal innstille på medlemmer til Mediastud. Valgkomiteen
		består av styreleder i SiT, leder for Finansstyret på Samfundet,
		og et uttredende styremedlem valgt av styret i Mediastud.
		Valgkomiteen ledes av uttredende styremedlem.

		Styret i Mediastud bør ha kompetanse innenfor disse områdene
		\begin{enumerate}
			\item Ledelse av studentorganisasjoner
			\item Mediefaglig kompetanse
			\item Strategisk økonomistyring
			\item Styrekompetanse
		\end{enumerate}

		Alle styremedlemmer skal bidra til å styrke Mediastuds evne til
		å oppnå sine mål og strategier og er forpliktet til å utøve sitt
		verv for Mediastuds beste og i overensstemmelse med prinsipper
		for god virksomhetsstyring og aksjeloven.

		Alle styremedlemmer skal være bosatt i Trondheim under sin
		funksjonstid.

		Styret ansetter daglig leder. Styret skal utarbeide instruks for
		sitt arbeide og for daglig leder. Styreinstruksen godkjennes av
		Partene.
	\end{instruksledd}

	\begin{instruksledd}{Ikrafttredelse, varighet og endring av avtale}
	Avtalen trer i kraft ved signering og er i kraft så lenge Partene er aksjonærer
	i Mediastud, eller frem til Mediastud er oppløst. 

	Alle endringer til denne avtale skal være skriftlige, inntas som eget vedlegg
	til avtalen og være signert av Partene. 

	Avtalen kan avtalen sies opp med minst 12 måneder skriftlig varsel.
	Dersom ikke en ny avtale er fremforhandlet ved oppsigelsestidens utløp, skal
	Mediastud avvikles etter skriftlig påkrav fra en av Partene.

	Ved avvikling av Mediastud skal selskapets aktiva fordeles som bestemt i
	vedtektene.

	Partene skal sammen beslutte hvorledes vedtektenes bestemmelser om oppløsning
	best oppfylles og hvilke formål som skal tilgodeses.
	\end{instruksledd}

	\begin{instruksledd}{Tvistløsning}
	Dersom det oppstår tvist mellom Partene om tolkningen eller
	gjennomføringen av denne avtale, tilleggs- eller endringsavtaler, tvist
	om krav som er knyttet til denne avtale eller tvist om gyldigheten av
	denne avtale skal tvisten mekles etter tvisteloven § 8-1. Meklingmannen
	oppnevnes av Sør-Trøndelag tingrett.   

	Dersom mekling ikke fører frem, avgjøres tvisten ved alminnelig
	rettergang. Mediastuds forretningskontor vedtas som avtalens verneting.
	\end{instruksledd}

	\vfill
For Studentsamskipnaden i Trondheim 	\hfill	For Studentersamfundet i Trodnhjem

Konserstyreleder \hfill		Finansstyrets leder


\end{instruks}

