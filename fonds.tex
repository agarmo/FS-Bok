\chapter{Fond}

\begin{fond}{Klubbstyrets Fond}
    \begin{fondsparagraf}{Fondets Formål}
        Fondets formål er i følgende prioriterte rekkefølge:
        \begin{enumerate}
            \item å utgjøre likviditetsreserve for Klubbstyrets drift.
            \item å kunne benyttes til investeringer som kommer Klubbaften til gode.
        \end{enumerate}
    \end{fondsparagraf}

    \begin{fondsparagraf}{Bevilgninger fra fondet}
        Bevilgninger eller lån fra fondet vedtas i Finansstyret kun etter søknad fra
        Klubbstyret. Slike søknader skal behandles på en generalforsamling i Klubbstyret.
    \end{fondsparagraf}

    \begin{fondsparagraf}{Overføring til fondet}
Klubbstyret kan etter at regnskapet er godkjent av revisor, fatte vedtak om det skal foretas overføring til fondet eller
ikke. Denne beslutningen skal bygge på en realistisk vurdering av nødvendig likviditet for Klubbstyrets drift de neste
regnskapsår.
    \end{fondsparagraf}
    
    \begin{fondsparagraf}{Fondets tilhørighet}
Fondet inngår i Finansstyrets forvaltning av øvrige fond. Finansstyret forplikter seg til å sørge for best mulig
avkastning av innestående midler innenfor de gjeldene fullmakter for pengeplassering. Avkastningen skal i sin helhet
tilfalle fondet.
    \end{fondsparagraf}
    
    
    


\end{fond}



\begin{fond}{Statutter for Kampfond}

  \begin{fondsparagraf}
Alle medlemmer av Studentersamfundet har rett til å komme med bevilgningsforslag. Det bør være levert til Styret
minst 96 timer før det lørdagsmøtet der forslaget skal realitetsbehandles.
  \end{fondsparagraf}

  \begin{fondsparagraf}
Forslag om bevilgning skal behandles som en resolusjon, se §18b i Studentsamfundets lover.
  \end{fondsparagraf}
  
  \begin{fondsparagraf}
Kampfond skal ikke brukes til drift av faste organisasjoner og lag, men til streiker, aksjoner og andre saker som
trenger rask støtte.
  \end{fondsparagraf}
  
  \begin{fondsparagraf}
Hvert forslag om bevilgning er begrenset oppover til 10\% av det totale fondet.
  \end{fondsparagraf}
  
  \begin{fondsparagraf}
Styret plikter å gjøre forslagsgiveren oppmerksom på at vedtatte bevilgning må søkes utbetalt innen 6 uker.
  \end{fondsparagraf}  

\end{fond}

\begin{fond}{Student Georg Sverdrups fond til utdeling av gratis medlemskort}
\emph{Vedtatt 22/4 1922 med senere endringer og tilføyelser.

Se forøvrig lovenes §6, vedtatt 31/10 1931.
}

  \begin{fondsparagraf}
  Fondet bestyres av Studentersamfundet i Trondhjem's finansstyre.
  \end{fondsparagraf}
  
  \begin{fondsparagraf}
Fondets midler kan bare brukes til å bestride ubemidlede studenters medlemskontingent til Studentersamfundet i
Trondhjem.
  \end{fondsparagraf}
  
  \begin{fondsparagraf}
Fondet skal være urørlig, og bare rentene må brukes til utdeling i samsvar med §2.

Blir de disponible midler helt eller delvis ikke utdelt, kan fondets styre vedta om de unyttede midler skal legges til det
urørlige fond eller legges til en senere utdeling.
  \end{fondsparagraf}

  \begin{fondsparagraf}
Utdelingen blir foretatt semestervis av finansstyret etter innstilling fra Studentersamfundets styre, som søknadene
stilles til. I alminnelighet forutsettes beløpet fordelt i hele semesterkontingenter.Men finansstyret kan dele ut porsjoner
som motsvarer 1/2 semesterkontingent.
  \end{fondsparagraf}
  
  \begin{fondsparagraf}
Enhver søknad må inneholde opplysninger om søkerens økonomiske forhold, bilagt med attester. Under ellers like
forhold gir utvist interesse for Samfundet eller noen gren av dets virksomhet (teater, orkester, frivillig undervisning
etc.) fortrinnsrett. Særvilkår kan ikke oppstilles.
  \end{fondsparagraf}
  
  \begin{fondsparagraf}
Skulle Studentersamfundet i Trondhjem stanse sin virksomhet, skal nærværende fond etter de regler som er fastsatt i
disse statutter deles ut til studiehjelp for trengende studenter.
  \end{fondsparagraf}
  
  \begin{fondsparagraf}
Enhver forandring av disse statutter kan bare Studentersamfundet i Trondhjem foreta, og da på den måte som er
bestemt for endringer i Samfundets alminnelige lover. §1, 2, 5 og 6 kan ikke forandres.
  \end{fondsparagraf}
  
\end{fond}



\begin{fond}{Sceneteknisk Fond}
    \begin{fondsparagraf}{Fondets Formål}
        Fondet er opprettet av Finansstyret med formål å finansiere investering i teknisk
        utstyr til Studentersamfundets scener.
    \end{fondsparagraf}

    \begin{fondsparagraf}{Fondsstyre}
        Fondets styre består av fire personer, derav en leder. Finansstyret oppnevner fire
        medlemmer, to av disse etter innstilling fra Gjengsekretariatet. Finanssytyret
        utpeker en leder. Representater i styret sitter til ny oppnevnes.
    \end{fondsparagraf}

    \begin{fondsparagraf}{Sekretær}
        Gjengsekretariatets medlem i fondsstyret er sekretær. Sekretæren har ansvaret for å
        løpende rapportere bevilgninger til
        FS i form av referat med signatur fra fondsstyremøter og investeringsoverslag til
        regnskapsfører.
    \end{fondsparagraf}

    \begin{fondsparagraf}{Investeringsplan}
        Fondsstyret har ansvaret for revisjon av investeringsplanen. Investeringsplanen fremlegges
        av arrangerende gjenger samtidig med utarbeidelse av budsjett for førstkommende regnskapsår. Den skal inneholde
        konkrete ønsker og behov knyttet til investeringene gjengene ønsker å gjøre. Fondsstyret utarbeider et felles
        budsjett for investeringer ut fra de respektive gjengers budsjett. Budsjettet videresendes Gjengsekretariatet (og Finansstyret)
        for godkjenning. Budsjettet kan utarbeides på grunnlag av tildelte midler fra Arrangørfondet, men kan også søke om
        ytterligere midler fra Finansstyret. Investeringene skal være i henhold til fondets formål.
    \end{fondsparagraf}

    \begin{fondsparagraf}{Bevilgninger fra fondet}
        Fondsstyret har fullmakt til å behandle bevilgninger, men må løpende rapportere til
        Finansstyret. Bevilgningene skjer
        med utgangspunkt i investeringsplanen som skal legges frem ved starten av regnskapsåret.
        Ytterligere bevilgninger
        kan vedtas av fondsstyret i forbindelse med revidert budsjett. Ved vesentlige endringer må
        revidert budsjett godkjennes av Finansstyret.
    \end{fondsparagraf}

    \begin{fondsparagraf}{Overføring til fondet}
        Klubbstyret, Lørdagskomiten og Kulturutvalget kan overføre eventuelle overskudd til
        fondet. De arrangerende gjenger
        kan etter regnskapet er godkjent av revisor, fatte vedtak om midler skal overføres fondet
        eller ikke. Denne
        beslutningen skal bygge på en realistisk vurdering av nødvendig likviditet for drift de
        neste regnskapsår.
    \end{fondsparagraf}

    \begin{fondsparagraf}{Forvaltning av fondet}
        Fondet forvaltes av Finansstyret på linje med øvrige fond. Finansstyret skal sørge for
        best mulig avkastning av
        innestående midler innenfor de gjeldende fullmakter for pengeplassering. Avkastningen av
        fondets midler skal i sin
        helhet tilfalle fondet.

        Om nødvendig gis finansstyret fullmakt til å gi likviditetslån til arrangerende gjenger
        etter oppfordring fra fondsstyret,
        lånemidler kan bare brukes til å finansiere driftsutgifter. Det kan stilles garanti fra
        fondet i form av at fondsstyret
        forelegger Finansstyret skriftlig dokumentasjon på hva garantiforholdet gjelder.
    \end{fondsparagraf}

    \begin{fondsparagraf}{Endring av fondets statutter, avvikling}
        Fondets statutter kan endres av fondsstyret i samarbeid med arrangerende gjengers
        medlemmer, og må godkjennes av
        Finansstyret. Fondet kan avvikles av Finansstyret i samarbeid med fondsstyret. Ved
        avvikling disponeres fondets
        midler av Finansstyret.
    \end{fondsparagraf}

\end{fond}
