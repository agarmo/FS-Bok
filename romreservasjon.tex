
\begin{instruks}{Retningslinjer for reservering av offentlige arealer
    ved Studentersamfundet i Trondhjem}{16. desember 2010}{1. mars 2011}

    \begin{instruksledd}{Formål og intensjon}
        Retningslinjene for reservering av offentlige arealer ved Studentersamfundet i Trondhjem
        (Samfundet) skal fremme
        Samfundets mål om å være det naturlige samlingssted for trondheimsstudentene gjennom et
        bredt sosialt og kulturelt tilbud.


        Retningslinjene gjelder både for internt bruk, utlån og utleie, slik at Samfundets lokaler
        kan benyttes av eksterne
        grupper når interne ikke har behov. Retningslinjene skal sikre hensiktsmessig
        arealfordeling, god utnyttelse av
        lokalene og jevn arrangementsspredning. De skal bidra til nødvendig informasjonsflyt
        mellom brukerne og føre til at
        tekniske gjenger involveres i arrangementsplanleggingen.
    \end{instruksledd}

    \begin{instruksledd}{Grupper som kan reservere rom}
        I tillegg til at Samfundets gjenger kan reservere rom til eget bruk,
        tilrettelegges det for utlån og til en viss grad utleie.
        Dette er for at Samfundet skal oppfylle målsetningen om å være et
        allaktivitetshus. Et bredt arrangementspekter øker
        medlemstilbudet, lån av lokaler er et medlemsgode, arrangementene kan rekruttere
        medlemmer og brukere/overskudd
        til Samfundets serveringstilbud.

        \begin{enumerate}
            \item \textbf{Studentersamfundets egne gjenger}

                Interne gjenger/grupperinger på Samfundet trenger å benytte de offentlige
                arealene av
                Samfundet for å få gjennomført sitt arbeid, enten det er arrangementer,
                vedlikehold, kurs,
                møter eller annet. Representanter for gjengene deltar på rombookingsmøtet
                og har førsterett
                på romreservasjoner (etter prioriteringsnormene under punkt 6.2). Oppmøte
                på
                rombookingsmøtet er en forutsetning for å få beholde reservasjonsønsker.

            \item \textbf{ Låntakere: Eksterne idealistiske grupper med åpne arrangement}

                Eksterne idealistiske studentgrupper som ønsker å legge sine arrangementer
                til Samfundet
                slipper å betale for romreservasjon under forutsetning av at
                arrangementene er åpne for alle
                Samfundets medlemmer og at arrangør ikke har økonomisk inntjening på
                arrangementet.
                Med ”eksterne idealistiske grupper” forstås i denne instruksen
                studenter/grupper av studenter
                som ikke er tilknyttet gjengene på Samfundet, og som ønsker å legge
                arrangementer av
                kulturell, politisk eller faglig art til Samfundet. Låntakeren skal
                fortrinnsvis være medlem av
                Samfundet, arrangementene skal være åpne for alle våre medlemmer og
                medlemmer skal ha
                rabatt ved eventuell inngangsbillett. Linjeforeningsfester og andre
                lukkede arrangement faller
                inn under utleie (se punkt 3.3 og 3.4).

            \item \textbf{Leietakere: Studentgrupper med lukkete arrangement}

                Eksterne idealistiske grupper og andre studentgrupper som ønsker å bruke
                Samfundets lokaler
                til lukkete arrangement, må betale leie på lik linje med kommersielle
                aktører.

            \item \textbf{ Leietakere: Kommersielle aktører med åpne/lukkete arrangement}

                Kommersielle aktører som vil leie rom ved Samfundet til åpne eller lukkete
                arrangementer,
                har anledning til det dersom ikke interne gjenger/grupper trenger
                lokalene. Ved åpne
                arrangement forutsettes det at arrangøren gir prisrabatt til Samfundets
                medlemmer.
        \end{enumerate}
    \end{instruksledd}

    \begin{instruksledd}{Forholdet mellom Samfundet og lån-/leietakere}
        Utlåns- og utleieregler fastsettes av Finansstyret. Daglig leder har anledning
        til å forhandle
        leiepris innenfor rammer gitt av Finansstyret.
        Daglig leder er ansvarlig for all kontakt og kontraktsfullbyrdelse med
        leietakerne. Dette innebærer blant annet
        ivaretakelse av Samfundets regler og koordinering med interne gjenger.

        Styret er ansvarlig for Samfundets låntakere. Ansvaret innebærer blant annet
        kontakt koordinering mot berørte parter
        internt, rom-/nøkkelansvar og tilstedeværelse under arrangementet.

        Tekniske gjenger skal ha anledning til å uttale seg om sin kapasitet og
        eventuelt si nei til å stille teknisk assistanse
        under både utleie og utlån.

        Dersom et tildelt rom ikke vil benyttet plikter lån-/leietaker å umiddelbart
        gi beskjed til Kontrollkontoret (kk@samfundet.no).
    \end{instruksledd}

    \begin{instruksledd}{Reservasjons- og romfordelingsprosess}
        Daglig leder er ansvarlig for romfordeling på Samfundet. Administrasjonen skal
        planlegge romfordeling i samarbeid
        med gjengene, oppdatere databasen for booking (robokop.samfundet.no) og
        registrere nye arrangementer fortløpende
        gjennom semesteret. Ansvar for booking av rom i forkant av semesteret ligger
        hos Gjengsekrteriatet.


        Arealbehov på fredager og lørdager avtales direkte med henholdsvis Klubbstyret
        og Lørdagskomiteen.

        \begin{enumerate}
            \item \textbf{Romreservasjon i forkant av semesterstart}
                \begin{enumerate}
                    \item Alle gjenger med behov for offentlige arealer utenom fredag
                        og lørdag må melde sine behov inn til
                        Huskoordinator i Styret før rombookingsmøtet i forkant av hvert
                        semester.
                    \item Styret har ansvaret for å legge romønskene inn i databasen,
                        systematisere informasjonen og sette opp forslag til
                        prioriteringer etter normene i denne instruks.
                    \item Semesterplan settes opp på rombookingsmøte i god tid før
                        hvert semester.
                    \item Tekniske gjenger skal formidle hvor lang tid de trenger til
                        forberedelse foran et arrangement og arrangør må
                        ta hensyn til dette.
                    \item Semesterplanen som er bestemt på rombookingsmøtet kan kun
                        endres ved enighet mellom partene. Nærmere
                        beskrivelse under punkt 5.3.
                    \item Dersom et tildelt rom ikke vil benyttet plikter låntaker å
                        umiddelbart gi beskjed til Kontrollkontoret
                        (kk@samfundet.no).
                \end{enumerate}

            \item \textbf{Romreservasjon i løpet av semesteret}

                I etterkant av semesterets rombookingsmøte kan Kontrollkontoret
                fortløpende reservere rom
                til interesserte.
                \begin{enumerate}
                    \item  Ledige lokaler tildeles i henhold til normer for
                        prioritering.
                    \item Utlån/-leie skal vanligvis ikke gå på bekostning av interne
                        romreservasjoner. Punkt 5.3 beskriver
                        retningslinjer ved overstyring av reserverte lokaler.
                    \item Romutlån til eksterne arrangementer som ikke står i
                        semesterprogrammet fastsettes endelig to uker før
                        arrangementet.
                    \item Romutleie som ikke står i semesterprogrammet kan fastsettes
                        to måneder før arrangementet dersom særlige
                        grunner taler for det.
                    \item Endringer og informasjon om ledige arealer skal til enhver
                        tid ligge oppdatert på robokop.samfundet.no.
                \end{enumerate}

            \item \textbf{Overstyring av reserverte lokaler}

                Dersom tungtveiende grunner taler for det kan Samfundet benytte
                arealer som er allerede er reservert. Med
                tungtveiende grunner menes åpne arrangement som er forventet å trekke
                tilnærmet fullt lokale. Dersom det er behov
                for overstyring av eksempelvis orkesterøvinger skal
                Studentersamfundets Symfoniorchester forespørres mens det ennå
                er tid til å skaffe et godt øvingslokale. Det samme gjelder korøvinger
                og annen gjengvirksomhet. Dersom det ikke
                finnes egnet internt rom skal Samfundet dekke merutgiftene til leie av
                annet lokale.
        \end{enumerate}

    \end{instruksledd}

    \begin{instruksledd}{Prioriteringsnormer}
        Prioriteringsnormene utelukker ikke at arealene benyttes til andre typer
        arrangementer enn det som går frem av disse
        retningslinjene, dersom lokalene er ledige og Studentersamfundet kan godkjenne
        bruken.
        \begin{enumerate}
            \item \textbf{Generell romprioritering}
                \begin{enumerate}
                    \item Arrangementer som er åpne for alle medlemmer har prioritet
                        foran lukkede arrangementer.
                    \item Faste konsepter har prioritet foran enkeltstående
                        arrangementer.
                    \item Arrangementer som er utlyst i semesterprogrammet kan ikke
                        endre dato, men eventuelt rom dersom
                        involverte parter enes om dette.
                    \item Arrangementer hvor det er inngått bindende avtaler kan ikke
                        endres.
                    \item Øvingene til Studentersamfundets Symfoniorchester har
                        prioritet foran andre arrangementer i Storsalen (se
                        også punkt 5.3)
                    \item Interne møter som ikke kan holdes på private områder skal i
                        størst mulig grad benytte Trafoens og NTNUs
                        lokaler. Under UKA og ISFiT benytter disse organisasjonene
                        Trafoen.
                \end{enumerate}
            \item \textbf{ Prioritering for enkeltarealer}
                FK og Regi benytter Knaus, Klubben og Storsalen på mandager.
                Aktiviteter på Knaus, i Klubben og i Storsalen på
                mandager må avklares med disse gjengene. Dersom mandagsarrangementer i
                andre lokaler krever lys- og/eller
                lydassistanse, må også dette avtales spesielt med de tekniske
                gjengene.
                \begin{enumerate}
                    \item Klubben
                        \begin{enumerate}
                            \item Åpne, faste arrangement\footnote{Med ``åpne, faste
                                arrangement'' menes regelmessige arrangementer/konsepter som er åpne for alle og som
                                arrangeres av en av Samfundets gjenger.}
                            \item Fotball
                            \item Åpne, enkeltstående arrangement \footnote{Med ``åpne,
                                enkeltstående arrangement'' menes arrangementer som er åpne for alle og som arrangeres av en
                        av Samfundets gjenger, men som ikke faller inn under noe fast,
                        jevnlig konsept.}
                            \item Åpne arrangement av eksterne idealistiske arrangører (utlån)
                            \item Øvinger
                            \item Faste interne arrangement\footnote{For eksempel Mingle
                                Vingle}
                            \item Åpne arrangement av eksterne kommersielle arrangører (utleie)
                            \item Lukkete arrangement av eksterne arrangører (utleie)
                            \item  Interne møter
                            \item Interne fester
                        \end{enumerate}
                   \item Knaus
                        \begin{enumerate}
                            \item  SIT-forestilling med tre generalprøvedager
                            \item  Åpne, faste arrangement
                            \item  Åpne, enkeltstående arrangement
                            \item  Åpne arrangement av eksterne idealistiske arrangører
                            \item  Øving SIT
                            \item Annen øving
                            \item Interne møter
                            \item Åpne arrangement av eksterne kommersielle arrangører
                            \item Lukkete arrangementer av eksterne arrangører (utleie)
                            \item Interne fester
                        \end{enumerate}
                    \item Sangerhallen
                        \begin{enumerate}
                            \item  Åpne, faste arrangement
                            \item Korøvelse
                            \item  Åpne, enkeltstående arrangement
                            \item Åpne arrangement av eksterne idealistiske arrangører (utlån)
                            \item Andre øvinger
                            \item Åpne arrangement av eksterne arrangører (utleie)
                            \item Lukkete arrangement av eksterne arrangører (utleie)
                            \item Interne møter
                            \item Interne fester
                        \end{enumerate}
                    \item Prakthønerommet
                        \begin{enumerate}
                            \item Åpne, faste arrangement
                            \item Åpne, enkeltstående arrangement
                            \item Åpne arrangement av eksterne idealistiske arrangører
                            \item Interne arrangement
                            \item Øvinger
                            \item Åpne arrangement av eksterne kommersielle arrangører (utleie)
                            \item Interne møter
                            \item Lukkete arrangement av eksterne arrangører (utleie)
                        \end{enumerate}
                    \item Storsalen
                        \begin{enumerate}
                            \item Åpne, faste arrangementer/Studentersamfundets
                                Symfoniorchester
                            \item Fotballkamper på landskampnivå
                            \item Konserter
                            \item Andre åpne, enkeltstående arrangement
                            \item SIT-forestilling med tre generalprøvedager
                            \item Åpne arrangementer av Samfundets kunstneriske
                                gjenger
                            \item Andre øvinger
                            \item Åpne arrangementer av eksterne idealistiske
                                arrangører (utlån)
                            \item Faste interne arrangement4
                            \item Åpne arrangementer av eksterne kommersielle
                                arrangører (utleie)
                            \item Lukkete arrangement av eksterne arrangører
                                (utleie)
                            \item Interne fester
                        \end{enumerate}
                        4 For eksempel Samfundets Interne Grand Prix
                    \item Bodegaen
                        \begin{enumerate}
                            \item Åpne, faste arrangement
                            \item Åpne, enkeltstående arrangement
                            \item Åpne arrangement av eksterne idealistiske
                                arrangører (utlån)
                            \item Åpne arrangement av eksterne kommersielle
                                arrangører (utleie)
                            \item Faste interne arrangement5
                            \item Lukkete arrangement av eksterne arrangører (utleie)
                        \end{enumerate}
                        5 For eksempel stikkefest.
                    \item Strossa
                        \begin{enumerate}
                            \item  Åpne, faste arrangement
                            \item Åpne, enkeltstående arrangement
                            \item Åpne arrangement av eksterne idealistiske arrangører
                                (utlån)
                            \item Åpne arrangement av eksterne kommersielle arrangører
                                (utleie)
                            \item Interne faste arrangement
                            \item Lukkete arrangement av eksterne arrangører (utleie)
                        \end{enumerate}
                    \item Trafoen: Lite og stort rom nede
                        \begin{enumerate}
                            \item Øvinger
                            \item Interne møter
                            \item Kurs
                        \end{enumerate}
                    \item Rundhallen
                        \begin{enumerate}
                            \item Åpne, faste arrangement
                            \item Åpne, enkeltstående arrangement
                            \item Åpne arrangement av eksterne idealistiske arrangører (utlån)
                            \item Åpne arrangement av eksterne kommersielle arrangører (utleie)
                            \item Interne faste arrangement
                            \item Lukkete arrangement av eksterne arrangører (utleie)
                        \end{enumerate}
                \end{enumerate}
        \end{enumerate}
    \end{instruksledd}

    \begin{instruksledd}{Oppdatering av retningslinjene}
        Denne instruksen er vedtatt av Finansstyret 22. mars 2007. Den kan evalueres
        av Gjengsjefkollegiet/Gjengsekretariatet og endres av Finansstyret ved behov.
    \end{instruksledd}

\end{instruks}


