\chapter{Statutter for Studentersamfundets plateselskap}

\begin{lovkapittel}{Statutter for Studentersamfundets plateselskap}
  
  \begin{lovparagraf}
    Studentersamfundets plateselskap skal stå for produksjon og spredning av musikkutgivelser fra Studentersamfundet i 
    Trondhjem, og ta vare på Samfundets opphavsrettigheter.
  \end{lovparagraf}
  
  \begin{lovparagraf} 
    En representant for FK og en representant for SIT står for den daglige driften av plateselskapet. De blir oppnevnt av
    Finansstyret for et tidsrom på 2 år etter innstilling fra vedkommende gjeng. Tjenestetiden bør helst være fasedreid 1 år
    for å sikre kontinuiteten.
  \end{lovparagraf}
  
  \begin{lovparagraf}
    Plateselskapet er underlagt Finansstyret i alle saker som har stor økonomisk rekkevidde. Det som blir produsert må på
    forhånd godkjennes av Finansstyret på grunnlag av et framlagt budsjett.
  \end{lovparagraf}
  
  \begin{lovparagraf}
    Plateselskapet kan tegne kontrakt om produksjon med medlemmer eller institusjoner innen Studentersamfundet i
    Trondhjem. En slik kontrakt må være godkjent på forhånd av Finansstyret.
  \end{lovparagraf}

  \begin{lovparagraf}
    Regnskap for hvert driftsår, revidert av Samfundets revisor, skal legges fram for Finansstyret til godkjenning snarest
    mulig i neste vårsemester. Et eventuelt driftsoverskudd blir avsatt til et driftsfond for å sikre Plateselskapets
    virksomhet. Finansstyret råder over midlene til driftsfondet. Midlene kan også, så langt en finner det nødvendig,
    benyttes til andre formål innen Samfundet, men likevel først og fremst på en slik måte at de indirekte vil komme
    Plateselskapet til gode. Ingen kan kreve andel i et eventuelt produksjonsoverskudd uten at det på forhånd foreligger
    skriftlig avtale om dette.
  \end{lovparagraf}

\end{lovkapittel} 
