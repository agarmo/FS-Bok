\begin{forord}
Studentersamfundet er en gammel organisasjon med lange tradisjoner. Listen over lover, instrukser og statutter er
nesten like lang som Samfundets histore. Disse retningslinjene er til for at alle vi som jobber på Samfundet skal kunne
leve sammen i harmoni, og for at arbeidet vi gjør skal gå så smertefritt som mulig.

Å skaffe seg en oversikt over alt dette er et femårsstudium i seg selv. Hovedpensum for faget er FS-boka, som gis ut
av Finansstyret. Målet er at denne trykksaken, som du sitter med i hendene nå, skal fungere som et oppslagsverk for
alle aktive på Huset, der alt som kan eller bør være av interesse er samlet.

Studentersamfundet er imidlertid en organisasjon i kontinuerlig endring. Nye gjenger opprettes, regler kommer til, og
regler forsvinner. Derfor jobbes det hele tiden med å oppdatere og revidere FS-boka. Gjengsekretariatet har fått i
oppgave av Finansstyret å foreslå endringer, som deretter blir behandlet av sistnevnte. I denne utgaven har blant annet
Hybelinstruksen og enkelte gjenginstrukser blitt oppdatert.

For at FS-boka hele tiden skal være så riktig som mulig, er vi som arbeider med boka avhengig av tilbakemelding.
Dersom du har innspill eller spørsmål om innholdet, er du hjertelig velkommen til å kontakte Gjengsekretariatet.



\vfill


Finansstyrets leder \hfill Gjengsekretariatet


\end{forord}

